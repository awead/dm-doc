\appendix
\chapter{Calculations}

Here are the mathemcatical details used to arrive at the ratios for each fret.  At the end of
each relevant passage, there will a series of calculations showing how I arrived at a particular
ratio for a given fret.  Each instruction method focuses on where to place the fret on the neck
of the instrument, but in order to determine the ratio we must find the vibrating length.
Therefore, each passage will have two calculations associated with it.  One denoted \textit{F}
which represents the location of the fret and another denoted \textit{V} which represents the
vibrating length.  In all cases, the mensur length of the string will be represented by the
constant \textit{m}.

The calculation of fret placement is determined according to the given passage from the
instructions, while the vibrating lenth is calculated as the difference between the fret distance
subracted from the mensur length.

\begin{eqnarray*}
    V_x = m - F_x
\end{eqnarray*}

As calculations proceed through a given source, the source often refers back to frets that have
already been placed.  For this, each fret calculation is referred to by name, such as
$F_{3}$ which would refer to the fret placement calcultation for the third fret.

\section{Hans Gerle}

\textbf{Fret 12}

\begin{eqnarray*}
F_{12} =
    \frac{m}{2} \\
V_{12} =
    m - F{12} =
    m - \frac{m}{2} =
    \frac{2m}{2} - \frac{1m}{2} =
    \frac{1m}{2}
    \to 2:1
\end{eqnarray*}

\textbf{Fret 7}

\begin{eqnarray*}
F_{7} =
    2 * ( \frac{F_{12}}{3} ) =
    2 * ( \frac{\frac{m}{2}}{3} ) =
    \frac{\frac{2m}{2}}{3} =
    \frac{m}{3} \\
    V_{7} = m - \frac{m}{3} = \frac{2m}{3} \to 3:2
\end{eqnarray*}

\textbf{Fret 1}

\begin{eqnarray*}
F_{1} =
    2 * ( \frac{F_{7}}{11} ) =
    2 * ( \frac{\frac{m}{3}}{11} ) =
    \frac{\frac{2m}{3}}{11} =
    \frac{2m}{33} \\
V_{1} =
    m - \frac{2m}{33} =
    \frac{31m}{33}
    \to 33:31
\end{eqnarray*}

\textbf{Fret 2}

\begin{eqnarray*}
    F_{2}
        &=& \frac{F_{7}}{3}
        = \frac{\frac{m}{3}}{3}
        = \frac{m}{9} \\
    V_{2}
        &=& m - F_{2}
        = m - \frac{m}{9}
        = \frac{8m}{9}
        \to 9:8
\end{eqnarray*}

\textbf{Fret 5}

\begin{eqnarray*}\label{Gr-5}
    F_{5}
        &=& \frac{F_{12}}{2}
        = \frac{\frac{m}{2}}{2}
        = \frac{m}{4} \\
    V_{5}
        &=& m - F_{5}
        = m - \frac{m}{4}
        = \frac{3m}{4}
        \to 4:3
\end{eqnarray*}

\textbf{Fret 6}

\begin{eqnarray*}
    F_{6}
        &=& \frac{F_{5} + F_{7}}{2}
        = \frac{\frac{m}{4} + \frac{m}{3}}{2}
        = \frac{\frac{7m}{12}}{2}
        = \frac{7m}{24} \\
    V_{6}
        &=& m - F_{6}
        = m - \frac{7m}{24}
        = \frac{17m}{24}
        \to 24:17
\end{eqnarray*}

\textbf{Fret 3}

\begin{eqnarray*}
F_{3} =
    (3 + 5) * (\frac{F_{1}}{3}) =
    8 * (\frac{\frac{2m}{33}}{3} =
    8 * \frac{2m}{99} =
    \frac{16m}{99} \\
V_{3} =
    m - F_{3} =
    m - \frac{16m}{99} = \frac{83m}{99}
    \to 99:83
\end{eqnarray*}

\textbf{Fret 4}

\begin{eqnarray*}
    F_{4}
        &=& \frac{F_{3} + F_{5}}{2}
        = \frac{\frac{16m}{99} + \frac{m}{9}}{2}
        = \frac{\frac{64m}{396} + \frac{99m}{396}}{2}
        = \frac{\frac{163m}{396}}{2}
        = \frac{163m}{792} \\
    V_{4}
        &=& m - F_{4}
        = m - \frac{163m}{792}
        = \frac{629m}{792}
        \to 792:629
\end{eqnarray*}

\section{John Dowland}

All frets are identical to Gerle's ratios except:

\textbf{Fret 3}

\begin{eqnarray*}
    F_{3Dowland}
        &=& ( 3 + 4 + \frac{1}{2} ) * ( \frac{F_{1Dowland}}{3} ) \\
        &=& ( 7 + \frac{1}{2} ) * ( \frac{\frac{2m}{33}}{3} )
        = ( 7 + \frac{1}{2} ) * ( \frac{2m}{99} ) \\
        &=& \frac{14m}{99} + \frac{2m}{198}
        = \frac{28m}{198} + \frac{2m}{198}
        = \frac{30m}{198} \\
    V_{3Dowland}
        &=& m - F_{3Dowland}
        = m - \frac{30m}{198}
        = \frac{168m}{198}
        \to 198:168
\end{eqnarray*}

\textbf{Fret 4}

\begin{eqnarray*}
    F_{4Dowland}
        &=& \frac{F_{2Dowland} + F_{5Dowland}}{2} \\
        &=& \frac{\frac{30m}{198} + \frac{m}{4}}{2}
        = \frac{\frac{120m}{792} + \frac{198m}{792}}{2}
        = \frac{\frac{318m}{792}}{2}
        = \frac{318m}{1584} \\
    V_{4Dowland}
        &=& m - F_{4Dowland}
        = m - \frac{318m}{1584}
        = \frac{1266m}{1584}
        \to 1584:1266
\end{eqnarray*}

\textbf{Frets 8, 9 and 10}

\begin{eqnarray*}
    F_{8Dowland}
        &=& \frac{m-F_{1Dowland}}{3}
        = \frac{m - \frac{2m}{33}}{3}
        = \frac{\frac{31m}{33}}{3}
        = \frac{31m}{99} \\
    F_{9Dowland}
        &=& \frac{m-F_{2Dowland}}{3}
        = \frac{m - \frac{m}{9}}{3}
        = \frac{\frac{8m}{9}}{3}
        = \frac{8m}{27} \\
    F_{10Dowland}
        &=& \frac{m-F_{3Dowland}}{3}
        = \frac{m - \frac{30m}{198}}{3}
        =\frac{\frac{168m}{198}}{3}
        =\frac{168m}{594} \\
    V_{8Dowland}
        &=& m - F_{8Dowland}
        = m - \frac{31m}{99}
        = \frac{68m}{99}
        \to 99:68 \\
    V_{9Dowland}
        &=& m - F_{9Dowland}
        = m - \frac{8m}{27}
        = \frac{19m}{27}
        \to 27:19 \\
    V_{10Dowland}
        &=& m - F_{10Dowland}
        = m - \frac{168m}{594}
        = \frac{426m}{594}
        \to 594:426
\end{eqnarray*}

\section{Silvestro Gnassi}

\textbf{Fret 1}

\begin{eqnarray*}
    F_{1Gnassi}
        &=& \frac{F_2}{2}
        = \frac{\frac{m}{9}}{2}
        = \frac{m}{18} \\
    V_{1Gnassi}
        &=& m - F_1
        = m - \frac{m}{18}
        = \frac{17}{18}
        \to 18:17
\end{eqnarray*}

\textbf{Fret 3}

\begin{eqnarray*}
    F_{3Gnassi}
        &=& F_{1Gnassi} + F_{2Gnassi}
        = \frac{m}{18} + \frac{m}{9}
        = \frac{m}{18} + \frac{2m}{18}
        = \frac{3m}{18}
        = \frac{m}{6} \\
    V_{3Gnassi}
        &=& m - F_{3Gnassi}
        = m - \frac{m}{6}
        = \frac{5m}{6}
        \to 6:5
\end{eqnarray*}

\textbf{Fret 4}

\begin{eqnarray*}
    F_{4Gnassi}
        &=& \frac{F_{3Gnassi} + F_{5Gnassi}}{2}
        = \frac{\frac{m}{6} + \frac{m}{4}}{2}
        = \frac{\frac{4m}{24} + \frac{6m}{24}}{2}
        = \frac{\frac{10m}{24}}{2}
        = \frac{10m}{48} \\
    V_{4Gnassi}
        &=& m - F_{4Gnassi}
        = m - \frac{10m}{48}
        = \frac{38m}{48}
        \to 48:38
\end{eqnarray*}

\textbf{Fret 8}

\begin{eqnarray*}
    F_{8Gnassi}
        &=& F_{7Gnassi} + (F_{6Gnassi} - F_{5Gnassi}) \\
        &=& \frac{m}{3} + \frac{7m}{24} - \frac{m}{4}
        = \frac{m}{3} + \frac{7m}{24} - \frac{6m}{24}
        = \frac{m}{3} + \frac{m}{24} \\
        &=& \frac{8m}{24} + \frac{m}{24}
        = \frac{9m}{24}
        = \frac{3m}{8} \\
    V_{8Gnassi}
        &=& m - F_{8Gnassi}
        = m - \frac{3m}{8}
        = \frac{5m}{8}
        \to 8:5
\end{eqnarray*}
