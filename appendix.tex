\appendix

\chapter{Complete Fretting Diagrams}

\begin{figure}[ht]
\centering
\setlength{\unitlength}{0.5mm}
\begin{picture}(60,365)
% Draw fingerboard edges
\color{black}
\linethickness{0.075mm}
\put(0,0){\line(0,1){360}}
\put(60,0){\line(0,1){360}}

% Draw strings
% 6th course
\color{strings}
\linethickness{0.5mm}
\put(5,0){\line(0,1){360}}
\linethickness{0.25mm}
\put(7,0){\line(0,1){360}}
% 5th course
\put(15,0){\line(0,1){360}}
\put(17,0){\line(0,1){360}}
% 4th course
\put(25,0){\line(0,1){360}}
\put(27,0){\line(0,1){360}}
% 3rd course
\put(35,0){\line(0,1){360}}
\put(37,0){\line(0,1){360}}
% 2nd course
\put(45,0){\line(0,1){360}}
\put(47,0){\line(0,1){360}}
% 1st course
\put(56,0){\line(0,1){360}}
% Insert string pitch names for lute in G
% 6th
\color{black}
\put(2,365){\small{G}}
% 5th
\put(14,365){\small{c}}
% 4th
\put(24,365){\small{f}}
% 3rd
\put(34,365){\small{a}}
% 2nd
\put(43,365){\small{d'}}
% 1st
\put(53,365){\small{g'}}
\color{black}
\linethickness{1mm}
\put(0,281){\line(1,0){60}}
\color{black}
\put(60,280){\small{\textemdash  2nd (chromatic)}}
\color{black}
\linethickness{1mm}
\put(0,123.1){\line(1,0){60}}
\color{black}
\put(60,122.1){\small{\textemdash  7th (diatonic)}}
\color{black}
\linethickness{1mm}
\put(0,240.3){\line(1,0){60}}
\color{black}
\put(60,239.3){\small{\textemdash  3rd (diatonic)}}
\color{black}
\linethickness{1mm}
\put(0,73.5){\line(1,0){60}}
\color{black}
\put(60,72.5){\small{\textemdash  9th (chromatic)}}
\color{black}
\linethickness{1mm}
\put(0,309.6){\line(1,0){60}}
\color{black}
\put(60,308.6){\small{\textemdash  1st (diatonic)}}
\color{black}
\linethickness{1mm}
\put(0,5){\line(1,0){60}}
\color{black}
\put(60,4){\small{\textemdash  12th (diatonic)}}
\color{black}
\linethickness{1mm}
\put(0,214.7){\line(1,0){60}}
\color{black}
\put(60,213.7){\small{\textemdash  4th (chromatic)}}
\color{black}
\linethickness{1mm}
\put(0,155.5){\line(1,0){60}}
\color{black}
\put(60,154.5){\small{\textemdash  6th (chromatic)}}
\color{black}
\linethickness{1mm}
\put(0,92.7){\line(1,0){60}}
\color{black}
\put(60,91.7){\small{\textemdash  8th (diatonic)}}
\color{black}
\linethickness{1mm}
\put(0,46.4){\line(1,0){60}}
\color{black}
\put(60,45.4){\small{\textemdash  10th (diatonic)}}
\color{black}
\linethickness{1mm}
\put(0,178.4){\line(1,0){60}}
\color{black}
\put(60,177.4){\small{\textemdash  5th (diatonic)}}
\color{black}
\linethickness{1mm}
\put(0,355){\line(1,0){60}}
\color{black}
\put(60,354){\small{\textemdash  Nut}}
\color{black}
\linethickness{1mm}
\put(0,29.3){\line(1,0){60}}
\color{black}
\put(60,28.3){\small{\textemdash  11th (chromatic)}}
\color{black}
\put(2,332.3){\small{A$\flat$}}
\put(12,332.3){\small{d$\flat$}}
\put(22,332.3){\small{g$\flat$}}
\put(32,332.3){\small{b$\flat$}}
\put(42,332.3){\small{e$\flat$'}}
\put(52,332.3){\small{a$\flat$'}}
\color{black}
\put(2,196.5){\small{c}}
\put(12,196.5){\small{f}}
\put(22,196.5){\small{b$\flat$}}
\put(32,196.5){\small{d'}}
\put(42,196.5){\small{g'}}
\put(52,196.5){\small{c''}}
\color{black}
\put(2,59.9){\small{f}}
\put(12,59.9){\small{b$\flat$}}
\put(22,59.9){\small{e$\flat$'}}
\put(32,59.9){\small{g'}}
\put(42,59.9){\small{c''}}
\put(52,59.9){\small{f''}}
\color{black}
\put(2,260.6){\small{B$\flat$}}
\put(12,260.6){\small{e$\flat$}}
\put(22,260.6){\small{a$\flat$}}
\put(32,260.6){\small{c'}}
\put(42,260.6){\small{f'}}
\put(52,260.6){\small{b$\flat$'}}
\color{black}
\put(2,295.3){\small{A}}
\put(12,295.3){\small{d}}
\put(22,295.3){\small{g}}
\put(32,295.3){\small{b}}
\put(42,295.3){\small{e'}}
\put(52,295.3){\small{a'}}
\color{black}
\put(2,227.5){\small{B}}
\put(12,227.5){\small{e}}
\put(22,227.5){\small{a}}
\put(32,227.5){\small{c$\sharp$'}}
\put(42,227.5){\small{f$\sharp$'}}
\put(52,227.5){\small{b'}}
\color{black}
\put(2,17.1){\small{g}}
\put(12,17.1){\small{c'}}
\put(22,17.1){\small{f'}}
\put(32,17.1){\small{a'}}
\put(42,17.1){\small{d''}}
\put(52,17.1){\small{g''}}
\color{black}
\put(2,166.9){\small{c$\sharp$}}
\put(12,166.9){\small{f$\sharp$}}
\put(22,166.9){\small{b}}
\put(32,166.9){\small{d$\sharp$'}}
\put(42,166.9){\small{g$\sharp$'}}
\put(52,166.9){\small{c$\sharp$''}}
\color{black}
\put(2,83.1){\small{e}}
\put(12,83.1){\small{a}}
\put(22,83.1){\small{d'}}
\put(32,83.1){\small{f$\sharp$'}}
\put(42,83.1){\small{b'}}
\put(52,83.1){\small{e''}}
\color{black}
\put(2,37.8){\small{f$\sharp$}}
\put(12,37.8){\small{b}}
\put(22,37.8){\small{e'}}
\put(32,37.8){\small{g$\sharp$'}}
\put(42,37.8){\small{c$\sharp$''}}
\put(52,37.8){\small{f$\sharp$''}}
\color{black}
\put(2,139.3){\small{d}}
\put(12,139.3){\small{g}}
\put(22,139.3){\small{c'}}
\put(32,139.3){\small{e'}}
\put(42,139.3){\small{a'}}
\put(52,139.3){\small{d''}}
\color{black}
\put(2,107.9){\small{e$\flat$}}
\put(12,107.9){\small{a$\flat$}}
\put(22,107.9){\small{d$\flat$'}}
\put(32,107.9){\small{f'}}
\put(42,107.9){\small{b$\flat$'}}
\put(52,107.9){\small{e$\flat$''}}
\end{picture}
\caption{Standard Quarter-comma Fretting (complete)}
\label{fig:quarter-diatonic-complete}
\end{figure}


\begin{figure}[ht]
\centering
\setlength{\unitlength}{0.5mm}
\begin{picture}(60,365)
% Draw fingerboard edges
\color{black}
\linethickness{0.075mm}
\put(0,0){\line(0,1){360}}
\put(60,0){\line(0,1){360}}

% Draw strings
% 6th course
\color{strings}
\linethickness{0.5mm}
\put(5,0){\line(0,1){360}}
% 5th course
\put(15,0){\line(0,1){360}}
% 4th course
\put(25,0){\line(0,1){360}}
% 3rd course
\put(35,0){\line(0,1){360}}
% 2nd course
\put(45,0){\line(0,1){360}}
% 1st course
\put(56,0){\line(0,1){360}}
% Insert string pitch names for lute in A (theorbo)
% 6th
\color{black}
\put(2,365){\small{A}}
% 5th
\put(14,365){\small{d}}
% 4th
\put(24,365){\small{g}}
% 3rd
\put(34,365){\small{b}}
% 2nd
\put(43,365){\small{e'}}
% 1st
\put(53,365){\small{a'}}
\color{black}
\linethickness{1mm}
\put(0,281){\line(1,0){60}}
\color{black}
\put(60,280){\small{\textemdash  2nd (chromatic)}}
\color{black}
\linethickness{1mm}
\put(0,144.5){\line(1,0){60}}
\color{black}
\put(60,143.5){\small{\textemdash  6th (diatonic)}}
\color{black}
\linethickness{1mm}
\put(0,21){\line(1,0){60}}
\color{black}
\put(60,20){\small{\textemdash  11th (diatonic)}}
\color{black}
\linethickness{1mm}
\put(0,240.3){\line(1,0){60}}
\color{black}
\put(60,239.3){\small{\textemdash  3rd (diatonic)}}
\color{black}
\linethickness{1mm}
\put(0,123.1){\line(1,0){60}}
\color{black}
\put(60,122.1){\small{\textemdash  7th (chromatic)}}
\color{black}
\linethickness{1mm}
\put(0,73.5){\line(1,0){60}}
\color{black}
\put(60,72.5){\small{\textemdash  9th (chromatic)}}
\color{black}
\linethickness{1mm}
\put(0,309.6){\line(1,0){60}}
\color{black}
\put(60,308.6){\small{\textemdash  1st (diatonic)}}
\color{black}
\linethickness{1mm}
\put(0,5){\line(1,0){60}}
\color{black}
\put(60,4){\small{\textemdash  12th (chromatic)}}
\color{black}
\linethickness{1mm}
\put(0,214.7){\line(1,0){60}}
\color{black}
\put(60,213.7){\small{\textemdash  4th (chromatic)}}
\color{black}
\linethickness{1mm}
\put(0,92.7){\line(1,0){60}}
\color{black}
\put(60,91.7){\small{\textemdash  8th (diatonic)}}
\color{black}
\linethickness{1mm}
\put(0,46.4){\line(1,0){60}}
\color{black}
\put(60,45.4){\small{\textemdash  10th (diatonic)}}
\color{black}
\linethickness{1mm}
\put(0,178.4){\line(1,0){60}}
\color{black}
\put(60,177.4){\small{\textemdash  5th (diatonic)}}
\color{black}
\linethickness{1mm}
\put(0,355){\line(1,0){60}}
\color{black}
\put(60,354){\small{\textemdash  Nut}}
\color{black}
\put(2,332.3){\small{B$\flat$}}
\put(12,332.3){\small{e$\flat$}}
\put(22,332.3){\small{a$\flat$}}
\put(32,332.3){\small{c'}}
\put(42,332.3){\small{f'}}
\put(52,332.3){\small{b$\flat$'}}
\color{black}
\put(2,196.5){\small{d}}
\put(12,196.5){\small{g}}
\put(22,196.5){\small{c'}}
\put(32,196.5){\small{e'}}
\put(42,196.5){\small{a'}}
\put(52,196.5){\small{d''}}
\color{black}
\put(2,59.9){\small{g}}
\put(12,59.9){\small{c'}}
\put(22,59.9){\small{f'}}
\put(32,59.9){\small{a'}}
\put(42,59.9){\small{d''}}
\put(52,59.9){\small{g''}}
\color{black}
\put(2,33.7){\small{a$\flat$}}
\put(12,33.7){\small{d$\flat$'}}
\put(22,33.7){\small{g$\flat$'}}
\put(32,33.7){\small{b$\flat$'}}
\put(42,33.7){\small{e$\flat$''}}
\put(52,33.7){\small{a$\flat$''}}
\color{black}
\put(2,260.6){\small{c}}
\put(12,260.6){\small{f}}
\put(22,260.6){\small{b$\flat$}}
\put(32,260.6){\small{d'}}
\put(42,260.6){\small{g'}}
\put(52,260.6){\small{c''}}
\color{black}
\put(2,227.5){\small{c$\sharp$}}
\put(12,227.5){\small{f$\sharp$}}
\put(22,227.5){\small{b}}
\put(32,227.5){\small{d$\sharp$'}}
\put(42,227.5){\small{g$\sharp$'}}
\put(52,227.5){\small{c$\sharp$''}}
\color{black}
\put(2,295.3){\small{B}}
\put(12,295.3){\small{e}}
\put(22,295.3){\small{a}}
\put(32,295.3){\small{c$\sharp$'}}
\put(42,295.3){\small{f$\sharp$'}}
\put(52,295.3){\small{b'}}
\color{black}
\put(2,161.4){\small{e$\flat$}}
\put(12,161.4){\small{a$\flat$}}
\put(22,161.4){\small{d$\flat$'}}
\put(32,161.4){\small{f'}}
\put(42,161.4){\small{b$\flat$'}}
\put(52,161.4){\small{e$\flat$''}}
\color{black}
\put(2,83.1){\small{f$\sharp$}}
\put(12,83.1){\small{b}}
\put(22,83.1){\small{e'}}
\put(32,83.1){\small{g$\sharp$'}}
\put(42,83.1){\small{c$\sharp$''}}
\put(52,83.1){\small{f$\sharp$''}}
\color{black}
\put(2,13){\small{a}}
\put(12,13){\small{d'}}
\put(22,13){\small{g'}}
\put(32,13){\small{b'}}
\put(42,13){\small{e''}}
\put(52,13){\small{a''}}
\color{black}
\put(2,133.8){\small{e}}
\put(12,133.8){\small{a}}
\put(22,133.8){\small{d'}}
\put(32,133.8){\small{f$\sharp$'}}
\put(42,133.8){\small{b'}}
\put(52,133.8){\small{e''}}
\color{black}
\put(2,107.9){\small{f}}
\put(12,107.9){\small{b$\flat$}}
\put(22,107.9){\small{e$\flat$'}}
\put(32,107.9){\small{g'}}
\put(42,107.9){\small{c''}}
\put(52,107.9){\small{f''}}
\end{picture}
\caption{Complete quarter-comma fretting for theorbo (extended courses not shown)}
\label{fig:quarter-diatonic-complete-a}
\end{figure}


\chapter{Fret Placement Guide}

Below is a set of tables for determining the placement of a fret or pitch for any of the 
temperaments discussed in this paper. Table~\ref{calc-12} covers fetting systems with
12 individual frets.  Table~\ref{calc-19} covers enharmonic fretting systems where
a total of 19 frets are possible, and assumes a standard Renaissance lute pitch of
G.  Because of this assumption, frets for B$\sharp$ and E$\sharp$ are provided so 
we can also place frets for lutes tuned in F, A, or other nominal pitches.

To determine the placement of a given fret, find the coefficient for the pitch
you wish to use and multiply it by your mensur length.  Use the resulting value,
and measure from the nut to find the exact position of the fret.  For example,
if a lute has a mensur of 65 centimeters, and we want to use Dowland's first fret,
multiply 65 by 0.0606 for a value of 3.939.  That is the distance, in centimeters,
from the nut to the first fret using Dowland's own placement.

\begin{table}
\tiny
\begin{center}
\begin{tabular}{| l || l | l | l | l | l | l | l | l | l | l | l | l |}
\hline
\textbf{Temperament / Fret} & \textbf{1} & \textbf{2} & \textbf{3} & \textbf{4} & \textbf{5} & \textbf{6} & \textbf{7} & \textbf{8} & \textbf{9} & \textbf{10} & \textbf{11} & \textbf{12} \\
\hline
\hline
Pietro Aron & 0.0429 & 0.1056 & 0.1641 & 0.2000 & 0.2523 & 0.2844 & 0.3313 & 0.3600 & 0.4018 & 0.4409 & 0.4661 & 0.5000 \\
\hline
Hans Gerle & 0.0606 & 0.1111 & 0.1616 & 0.2058 & 0.2500 & 0.2917 & 0.3333 & n/a & n/a & n/a & n/a & 0.5000 \\
\hline
John Dowland & 0.0606 & 0.1111 & 0.1515 & 0.2008 & 0.2500 & 0.2917 & 0.3333 & n/a & n/a & n/a & n/a & 0.5000 \\
\hline
Mersenne 1 & 0.0625 & 0.1111 & 0.1667 & 0.2000 & 0.2500 & 0.2889 & 0.3333 & 0.3750 & 0.4000 & 0.4371 & 0.4667 & 0.5000 \\
\hline
Mersenne 2 & 0.0556 & 0.1070 & 0.1576 & 0.2044 & 0.2476 & 0.2930 & 0.3298 & 0.3670 & 0.4022 & 0.4354 & 0.4667 & 0.5036 \\
\hline
Ganassi (initial)  & 0.0555 & 0.1111 & 0.1667 & 0.2084 & 0.2500 & 0.2917 & 0.3333 & 0.3750 & n/a & n/a & n/a & n/a \\
\hline
Ganassi (final)  & 0.0508 & 0.1111 & 0.1563 & 0.2099 & 0.2500 & 0.2881 & 0.3333 & 0.3672 & n/a & n/a & n/a & n/a \\
\hline
Bermudo I & 0.0636 & 0.1111 & 0.1563 & 0.2099 & 0.2500 & 0.2977 & 0.3333 & 0.3757 & 0.4074 & 0.4375 & n/a & n/a \\
\hline
Bermudo II & 0.0617 & 0.1111 & 0.1563 & 0.2099 & 0.2500 & 0.2963 & 0.3333 & 0.3745 & 0.4074 & 0.4375 & n/a & n/a \\
\hline
Bermudo III & 0.0564 & 0.1093 & 0.1563 & 0.2066 & 0.2500 & 0.2923 & 0.3319 & 0.3672 & 0.4049 & 0.4375 & n/a & n/a \\
\hline
17:18 Rule & 0.0556 & 0.1080 & 0.1576 & 0.2044 & 0.2486 & 0.2903 & 0.3298 & 0.3670 & 0.4022 & 0.4354 & 0.4667 & 0.4964 \\
\hline
Equal Temperament & 0.0561 & 0.1091 & 0.1591 & 0.2063 & 0.2508 & 0.2929 & 0.3326 & 0.3700 & 0.4054 & 0.4388 & 0.4703 & 0.5000 \\
\hline
\end{tabular}
\end{center}
\normalsize
\caption{Table of coefficients for 12-fret systems}
\label{calc-12}
\end{table}

\begin{sidewaystable}[htdp]
\tiny
\begin{center}
\begin{tabular}{| l || l | l | l | l | l | l | l | l | l | l | l | l | l | l | l | l | l | l |}
\hline
\textbf{Temperament} & \textbf{G$\sharp$} & \textbf{A$\flat$} & \textbf{A} & \textbf{A$\sharp$} & \textbf{B$\flat$} & \textbf{B} & \textbf{B$\sharp$} & \textbf{C} & \textbf{C$\sharp$} & \textbf{D$\flat$} & \textbf{D} & \textbf{D$\sharp$} & \textbf{E$\flat$} & \textbf{E} & \textbf{E$\sharp$} & \textbf{F} & \textbf{F$\sharp$} & \textbf{G$\flat$} \\
\hline
\hline
Third Comma & 0.0358 & 0.0704 & 0.1037 & 0.1358 & 0.1667 & 0.1966 & 0.2254 & 0.2531 & 0.2799 & 0.3057 & 0.3305 & 0.3545 & 0.3777 & 0.3999 & 0.4214 & 0.4422 & 0.4622 & 0.4814 \\
\hline
Quarter Comma & 0.0437 & 0.0649 & 0.1058 & 0.1449 & 0.1638 & 0.2004 & 0.2353 & 0.2522 & 0.2849 & 0.3008 & 0.3313 & 0.3606 & 0.3747 & 0.4021 & 0.4282 & 0.4409 & 0.4653 & 0.4771 \\
\hline
Fifth Comma & 0.0472 & 0.0624 & 0.1067 & 0.1489 & 0.1625 & 0.2020 & 0.2397 & 0.2519 & 0.2872 & 0.2986 & 0.3317 & 0.3632 & 0.3734 & 0.4030 & 0.4312 & 0.4403 & 0.4667 & 0.4752 \\
\hline
Sixth Comma & 0.0492 & 0.0611 & 0.1072 & 0.1511 & 0.1617 & 0.2030 & 0.2421 & 0.2516 & 0.2884 & 0.2973 & 0.3319 & 0.3647 & 0.3727 & 0.4035 & 0.4328 & 0.4399 & 0.4675 & 0.4741 \\
\hline
Seventh Comma & 0.0504 & 0.0602 & 0.1076 & 0.1526 & 0.1613 & 0.2036 & 0.2437 & 0.2515 & 0.2892 & 0.2965 & 0.3320 & 0.3657 & 0.3722 & 0.4039 & 0.4339 & 0.4397 & 0.4680 & 0.4735 \\
\hline
Eighth Comma & 0.0513 & 0.0596 & 0.1078 & 0.1536 & 0.1609 & 0.2040 & 0.2448 & 0.2514 & 0.2898 & 0.2960 & 0.3321 & 0.3663 & 0.3719 & 0.4041 & 0.4347 & 0.4396 & 0.4683 & 0.4730 \\
\hline
\end{tabular}
\end{center}
\normalsize
\caption{Table of coeffiecients for 19-fret systems}
\label{calc-19}
\end{sidewaystable}

\chapter{Calculations}

Here are the mathematical details used to calculate whole-number ratios for each fret.
Formulas are grouped according to source and the order of frets according to how they
appear in each source.  Each instruction method focuses on where to place the fret on
the neck of the instrument, but in order to determine the ratio we must find the
vibrating length. Therefore, each passage will have two calculations associated with
it.  One denoted \textit{F} which represents the location of the fret and another
denoted \textit{V} which represents the vibrating length.  In all cases, the mensur
length of the string will be represented by the constant \textit{m}.

The calculation of fret placement (\textit{F}) is determined according to the given
passage from the instructions, while the vibrating length (\textit{V}) is calculated as
the difference between the fret distance and mensur length.
\begin{eqnarray*}
    V_x = m - F_x
\end{eqnarray*}
As calculations proceed through a given source, the source often refers back to frets
that have already been placed.  For this, each fret calculation is referred to by name,
such as $F_{3}$ which would refer to the fret placement calculation for the third fret.

\linespread{1}
\section{Hans Gerle}
\textbf{Fret 12}
\begin{eqnarray*}
    F_{12}
        &=& \frac{m}{2} \\
    V_{12}
        &=& m - F_{12}
        = m - \frac{m}{2}
        = \frac{2m}{2} - \frac{1m}{2}
        = \frac{1m}{2}
        \to 2:1
\end{eqnarray*}
\textbf{Fret 7}
\begin{eqnarray*}
    F_{7}
        &=& 2 * ( \frac{F_{12}}{3} )
        = 2 * ( \frac{\frac{m}{2}}{3} )
        = \frac{\frac{2m}{2}}{3}
        = \frac{m}{3} \\
    V_{7} 
        &=& m - \frac{m}{3}
        = \frac{2m}{3}
        \to 3:2
\end{eqnarray*}
\textbf{Fret 1}
\begin{eqnarray*}
    F_{1}
        &=& 2 * ( \frac{F_{7}}{11} )
        = 2 * ( \frac{\frac{m}{3}}{11} )
        = \frac{\frac{2m}{3}}{11}
        = \frac{2m}{33} \\
    V_{1}
        &=& m - \frac{2m}{33}
        = \frac{31m}{33}
        \to 33:31
\end{eqnarray*}
\textbf{Fret 2}
\begin{eqnarray*}
    F_{2}
        &=& \frac{F_{7}}{3}
        = \frac{\frac{m}{3}}{3}
        = \frac{m}{9} \\
    V_{2}
        &=& m - F_{2}
        = m - \frac{m}{9}
        = \frac{8m}{9}
        \to 9:8
\end{eqnarray*}
\textbf{Fret 5}
\begin{eqnarray*}\label{Gr-5}
    F_{5}
        &=& \frac{F_{12}}{2}
        = \frac{\frac{m}{2}}{2}
        = \frac{m}{4} \\
    V_{5}
        &=& m - F_{5}
        = m - \frac{m}{4}
        = \frac{3m}{4}
        \to 4:3
\end{eqnarray*}
\textbf{Fret 6}
\begin{eqnarray*}
    F_{6}
        &=& \frac{F_{5} + F_{7}}{2}
        = \frac{\frac{m}{4} + \frac{m}{3}}{2}
        = \frac{\frac{7m}{12}}{2}
        = \frac{7m}{24} \\
    V_{6}
        &=& m - F_{6}
        = m - \frac{7m}{24}
        = \frac{17m}{24}
        \to 24:17
\end{eqnarray*}
\textbf{Fret 3}
\begin{eqnarray*}
    F_{3}
        &=& (3 + 5) * (\frac{F_{1}}{3})
        = 8 * (\frac{\frac{2m}{33}}{3}
        = 8 * \frac{2m}{99}
        = \frac{16m}{99} \\
    V_{3}
        &=& m - F_{3}
        = m - \frac{16m}{99} = \frac{83m}{99}
        \to 99:83
\end{eqnarray*}
\textbf{Fret 4}
\begin{eqnarray*}
    F_{4}
        &=& \frac{F_{3} + F_{5}}{2}
        = \frac{\frac{16m}{99} + \frac{m}{9}}{2}
        = \frac{\frac{64m}{396} + \frac{99m}{396}}{2}
        = \frac{\frac{163m}{396}}{2}
        = \frac{163m}{792} \\
    V_{4}
        &=& m - F_{4}
        = m - \frac{163m}{792}
        = \frac{629m}{792}
        \to 792:629
\end{eqnarray*}

\section{John Dowland}
All frets are identical to Gerle's ratios except:

\textbf{Fret 3}
\begin{eqnarray*}
    F_{3Dowland}
        &=& ( 3 + 4 + \frac{1}{2} ) * ( \frac{F_{1Dowland}}{3} ) \\
        &=& ( 7 + \frac{1}{2} ) * ( \frac{\frac{2m}{33}}{3} )
        = ( 7 + \frac{1}{2} ) * ( \frac{2m}{99} ) \\
        &=& \frac{14m}{99} + \frac{2m}{198}
        = \frac{28m}{198} + \frac{2m}{198}
        = \frac{30m}{198} \\
    V_{3Dowland}
        &=& m - F_{3Dowland}
        = m - \frac{30m}{198}
        = \frac{168m}{198}
        \to 198:168
\end{eqnarray*}
\textbf{Fret 4}
\begin{eqnarray*}
    F_{4Dowland}
        &=& \frac{F_{2Dowland} + F_{5Dowland}}{2} \\
        &=& \frac{\frac{30m}{198} + \frac{m}{4}}{2}
        = \frac{\frac{120m}{792} + \frac{198m}{792}}{2}
        = \frac{\frac{318m}{792}}{2}
        = \frac{318m}{1584} \\
    V_{4Dowland}
        &=& m - F_{4Dowland}
        = m - \frac{318m}{1584}
        = \frac{1266m}{1584}
        \to 1584:1266
\end{eqnarray*}
\textbf{Frets 8, 9 and 10}
\begin{eqnarray*}
    F_{8Dowland}
        &=& \frac{m-F_{1Dowland}}{3}
        = \frac{m - \frac{2m}{33}}{3}
        = \frac{\frac{31m}{33}}{3}
        = \frac{31m}{99} \\
    F_{9Dowland}
        &=& \frac{m-F_{2Dowland}}{3}
        = \frac{m - \frac{m}{9}}{3}
        = \frac{\frac{8m}{9}}{3}
        = \frac{8m}{27} \\
    F_{10Dowland}
        &=& \frac{m-F_{3Dowland}}{3}
        = \frac{m - \frac{30m}{198}}{3}
        =\frac{\frac{168m}{198}}{3}
        =\frac{168m}{594} \\
    V_{8Dowland}
        &=& m - F_{8Dowland}
        = m - \frac{31m}{99}
        = \frac{68m}{99}
        \to 99:68 \\
    V_{9Dowland}
        &=& m - F_{9Dowland}
        = m - \frac{8m}{27}
        = \frac{19m}{27}
        \to 27:19 \\
    V_{10Dowland}
        &=& m - F_{10Dowland}
        = m - \frac{168m}{594}
        = \frac{426m}{594}
        \to 594:426
\end{eqnarray*}

\section{Silvestro Ganassi}

\textbf{Fret 1}
\begin{eqnarray*}
    F_{1Ganassi}
        &=& \frac{F_2}{2}
        = \frac{\frac{m}{9}}{2}
        = \frac{m}{18} \\
    V_{1Ganassi}
        &=& m - F_1
        = m - \frac{m}{18}
        = \frac{17}{18}
        \to 18:17
\end{eqnarray*}
\textbf{Fret 3}
\begin{eqnarray*}
    F_{3Ganassi}
        &=& F_{1Ganassi} + F_{2Ganassi}
        = \frac{m}{18} + \frac{m}{9}
        = \frac{m}{18} + \frac{2m}{18}
        = \frac{3m}{18}
        = \frac{m}{6} \\
    V_{3Ganassi}
        &=& m - F_{3Ganassi}
        = m - \frac{m}{6}
        = \frac{5m}{6}
        \to 6:5
\end{eqnarray*}
\textbf{Fret 4}
\begin{eqnarray*}
    F_{4Ganassi}
        &=& \frac{F_{3Ganassi} + F_{5Ganassi}}{2}
        = \frac{\frac{m}{6} + \frac{m}{4}}{2}
        = \frac{\frac{4m}{24} + \frac{6m}{24}}{2}
        = \frac{\frac{10m}{24}}{2}
        = \frac{10m}{48} \\
    V_{4Ganassi}
        &=& m - F_{4Ganassi}
        = m - \frac{10m}{48}
        = \frac{38m}{48}
        \to 48:38
\end{eqnarray*}
\textbf{Fret 8}
\begin{eqnarray*}
    F_{8Ganassi}
        &=& F_{7Ganassi} + (F_{6Ganassi} - F_{5Ganassi}) \\
        &=& \frac{m}{3} + \frac{7m}{24} - \frac{m}{4}
        = \frac{m}{3} + \frac{7m}{24} - \frac{6m}{24}
        = \frac{m}{3} + \frac{m}{24} \\
        &=& \frac{8m}{24} + \frac{m}{24}
        = \frac{9m}{24}
        = \frac{3m}{8} \\
    V_{8Ganassi}
        &=& m - F_{8Ganassi}
        = m - \frac{3m}{8}
        = \frac{5m}{8}
        \to 8:5
\end{eqnarray*}