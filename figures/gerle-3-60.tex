\begin{figure}[ht]
\centering
\setlength{\unitlength}{1mm}
\begin{picture}(60,40.3)
% Draw fingerboard edges
\color{black}
\linethickness{0.075mm}
\put(0,0){\line(0,1){40.3}}
\put(60,0){\line(0,1){40.3}}

% Draw strings
% 6th course
\color{strings}
\linethickness{0.5mm}
\put(5,0){\line(0,1){40.3}}
\linethickness{0.25mm}
\put(7,0){\line(0,1){40.3}}
% 5th course
\put(15,0){\line(0,1){40.3}}
\put(17,0){\line(0,1){40.3}}
% 4th course
\put(25,0){\line(0,1){40.3}}
\put(27,0){\line(0,1){40.3}}
% 3rd course
\put(35,0){\line(0,1){40.3}}
\put(37,0){\line(0,1){40.3}}
% 2nd course
\put(45,0){\line(0,1){40.3}}
\put(47,0){\line(0,1){40.3}}
% 1st course
\put(56,0){\line(0,1){40.3}}
\color{markers}
\linethickness{0.5mm}
\put(0,6.5){\line(1,0){60}}
\color{black}
\put(60,5.9){\tiny{\textemdash equal semitone}}
\color{black}
\linethickness{1mm}
\put(0,5){\line(1,0){60}}
\color{black}
\linethickness{1mm}
\put(0,35.3){\line(1,0){60}}
\color{black}
\put(60,34.3){\small{\textemdash 2nd fret}}
\end{picture}
\caption{Scale drawing of Gerle's third fret}
\label{fig:gerle-3-60}
\end{figure}
