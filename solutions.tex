\chapter{Summary of Solutions}

Lute players today who want to perform music from the sixteenth and seventeenth
centuries must address the issue of using historically appropriate temperaments on
their instruments. Because of the nature of fretted instruments, using temperaments
with unequal semitones requires a different approach than is used with keyboards or
other instruments. In some ways, lutes have greater flexibility with temperaments than
keyboards do, but this flexibility results in a multitude of choices and options that
players must consider when fretting their instruments. Such choices can also reflect
personal opinions rather than historical accuracy.

To realize temperaments with unequal semitones on the lute, there are two basic approaches, both of
which are supported by historical evidence. The first is the ``fixed semitone'' method where each
fret is one type of semitone: chromatic, diatonic, equal or something unique. The second approach
treats the lute as an enharmonic instrument, similar to harpsichords with split keys. In this
method, two kinds of semitones, such as the diatonic and chromatic, are available at certain
locations on the fretboard using split or slanted frets, or with the aid of tastini. The choice is
entirely at the discretion of the player, and either technique can work regardless of the context in
which it is used.

In this chapter, I shall elaborate on the flexibility of these approaches and describe scenarios in
which either can be employed successfully.  I shall also discuss methods for adjusting frets in
order to perform with keyboards and other instruments in ensembles. Temperaments  during the Baroque
period were not universally applied.  While the modern musical world ordinarily relies on
established standards of pitch as well as temperament, such standards did not exist at this time in
musical history. Pitch could vary from city to city, and even from ensemble to ensemble. This same
variation applied to temperaments as well, not only between different ensembles, but between
different instruments in the same ensemble. This is perhaps a reason that the lute held such an
important place: it could conform to the different requirements of temperament more easily than
keyboards; however, this is not to say that specialized lute temperaments were limited to ensemble
music.

\section{Frets with Fixed Semitones}

As lute players, we have the option to ``fix it and forget it'', or in other
words, set our frets to a certain position and leave them.  Players such as
Dowland probably used this option, as well as vihuela composers such as Bermudo.
This requires a customized kind of fretting system in which we accept the
limitations of the temperament and either set our frets in a way that makes the
entire fretboard available to us, or select which semitones we want to
play on which fret. The historical sources presented in chapter two showed a
predominant use of these kinds of customized temperaments consisting of
different kinds of semitones used throughout the octave. Neither do they
conform to a particular kind of meantone, such as quarter-comma or sixth-comma, nor are
they consistently Pythagorean or equal in nature.

Evidence clearly indicates that fretting systems approximating equal temperament had
their place and purpose on the lute, while most other instruments preferred to use a
non-equal temperament.  So why is it that fretted instruments held this exception? The
reason seems to be that it was simply easier for a lutenist to divide the octave into
twelve approximately equal semitones than it was for a keyboardist. A keyboardist would
temper intervals aurally, counting beats between notes, while a lute player could
visualize the fret distances and create quasi-equal semitones by making a good
visual approximation between two existing frets and placing the fret somewhere in the
middle.

The visual method of approximating equal semitones could also take an iterative approach as well,
wherein the fret is adjusted several times while playing to find the position at which the semitones
sound the best. Whether by visual placement or trial and error, the semitones that resulted from
such methods were not truly today's modern standard of ``Equal'' Temperament, but an ``equal-ish''
temperament. These kinds of customized temperaments would be irregular in nature, with semitones of
varying size throughout the octave.  Depending on the semitone, however, they would be much closer
to modern-day equal temperament than any of the existing meantone temperaments of the time.

The use of temperaments that could approximate equal semitones was a feature
of the lute that would have appealed to amateurs. The treatises from which we
take our fretting instructions were often written for amateur musicians and intended
for their education. A simple fretting plan that yields a quasi-equal temperament
would simplify matters for someone who was still learning how to play the instrument.
As players became more experienced, they could start moving frets around to their
liking, creating a temperament that could selectively use different sized
semitones. Bermudo describes this when he refers to musicians who move their
frets according to their ears, but he instead wants to make the vihuela ``more
perfect'' with equal semitones for inexperienced players.\autocite[78]{DE:1}

Temperaments which are almost equal, and other customized schemes would also
be suitable for solo lute music and repertoire for small ensembles such as
the lute and voice. For example, the Dowland song \textit{Come heavy sleep} discussed in
the previous chapter would benefit from a temperament with fifths that were not too
narrow.  In such a temperament,  the D$\sharp$ and F$\sharp$ would not sound quite
as strident as they would if were in quarter-comma meantone. Some of the
fretting proposals discussed in the previous chapter could work in this regard,
with fifths wider than quarter-comma, although yet another solution is to use a
different type of meantone.

Such varieties of meantone were presented in chapter one, and in chapter two we saw that many of the
historical fretting sources preferred sixth-comma meantone for certain frets, and that it figured
quite prominently for the first, third, and sixth frets (see table~\ref{table:comparison}). A
similar temperament was also typical for other fretted instruments such as cittern, which was
commonly tuned somewhere between quarter-comma meantone and equal temperament. \autocite[12]{PF:1}
Sixth-comma meantone is essentially the midway point between quarter-comma meantone and equal
temperament, because some of its thirds are tuned slightly wider than pure, but not as wide as in
equal temperament. Other meantone temperaments whose thirds were wider than pure work well in early
sixteenth-century repertoire such as the music of vihuela composers Alonzo Mudarra and Luis
Milan.\autocite[56]{WH:1} Lastly, we can refer to Praetorius, in his \textit{Syntagma Musicum},
describing frets with 4$ \frac{1}{2} $ commas.\autocite[68]{MP:1} Although he is describing a
semitone divided equally, it is still sixth-comma in nature because the wholetone has 9 parts,
indicating that the frets were initially set in that kind of temperament.

Sixth-comma meantone is an excellent choice for fretted instruments because it does not restrict the
instrument as much as quarter-comma does, and generally does not require many modifications to make
it successful. At the same time, it makes a good compromise between equal semitones and the pure
thirds of quarter-comma meantone, allowing the lute to play all the semitones in the octave without
much difficulty. I can speak both personally and anecdotally that many lutenists prefer to use
either a regular sixth-comma meantone or its irregular varieties, such as Werkmeister and
Vallotti, as their default temperament for most music. These temperaments not only work better on
the lute, but also have the qualities of affect and expression that are lacking in equal
temperament.

As mentioned. custom fretting and temperaments such as sixth-comma meantone are ideal for the lute
in either solo music or small ensembles.  For example, in lute song repertoire and in ensembles with
other fretted instruments, instruments that do not have a fixed semitone size, such as a violin or
wind instrument, can conform to the temperament of the fretted instruments. Fixed-frets, however,
become problematic when using keyboards that are not tuned in sixth-comma, or if the members of the
ensemble prefer quarter-comma meantone.

It is possible to play a lute in quarter-comma meantone using a fixed system with one semitone per
fret; however, players must restrict the placement of certain semitones on the fretboard.  They
should choose the semitones needed for a particular piece, and place them in their designated
location on the fretboard.  For a theorbo in A, for example, we can set our frets using the standard
quarter-comma fretting pattern for lute described in chapter three 
(see figure~\ref{fig:quarter-diatonic-complete-a} in the appendix for the complete fretting).  
In this pattern, C$\sharp$ and F$\sharp$ are found on the fourth fret, and if we slant our sixth 
fret as described in figure~\ref{theorbo-slanted-sixth}, we would also have the G$\sharp$ as well. 
Depending on the piece of music, that might suffice for our needs.

If the needed semitones vary from piece to piece, or between sections of a larger work such as an
opera, players can change the position of the frets if they have enough time when not playing.
Alternative suggestions would be to slant certain frets, either before or during performance, or to
tie a double fret so that it could be split and one side moved higher or lower than the other. Using
varied pressure with the left-hand to alter the pitch is another possibility, but this has a limited
effect as it will also reduce the resonance of the stopped note, muffling it slightly. Beyond these
types of surface fixes, we must turn to extended methods if we want to have more than one kind of
semitone at a fret.

\section{Enharmonic Fretting}

If we truly need two different kinds of semitones available to us, and a fixed-fret
approach does not work, then we must turn to using either tastini or additional frets
tied to the fingerboard.  This offers the best possible solution and treats the lute as
an enharmonic instrument, allowing both chromatic and diatonic semitones to be
played. The drawback is that it increases the complexity of the instrument substantially,
and requires added technical skills on the part of the player. Players who work in
ensembles that use temperaments such as quarter-comma will opt for enharmonic fretting
if they wish to have a completely workable solution. Players can also choose
enharmonic fretting in solo literature.

Generally speaking, one tastino will make a chromatic semitone for a single course, while the
remaining courses are diatonic.  A common example of this are the tastini found between the nut and
first fret, as we saw in figure~\ref{fig:theorbo-tastini}. Whether stopping a course using a fret or
tastino, the best resonance is created on the instrument when the finger of the left-hand stops the
course just behind the fret or tastino, being as close to it as possible without going past it.
Usually, this process results in one kind of semitone per course so that, if a tastino were used at
one course, that course would contain a chromatic semitone, while the others would be diatonic.
However, some players cultivate the skill of playing both the diatonic and chromatic semitones on a
course where a tastino is used. In that case, the diatonic semitone is played with the finger placed
between the tastino and main fret, and the chromatic with the finger placed just behind the
tastino.  Skill requires being able to navigate the space between the fret and tastino, having
enough space between the two, and a small enough finger, to stop the course correctly in order
to achieve a good sound.

Tastini can be affixed to the fretboard in various ways. Ivory was the material of choice during
this time, but since it is illegal today, a toothpick and a piece of double-sided tape will work
just as well.  The main disadvantage of tastini, once a player is accustomed to using them, is
that they may slip or break.  More permanent solutions are possible, such as using glue and a more
durable wood, but this forces the instrument into a particular temperament which would needlessly
restrict its use when professional needs dictate tuning in a variety of temperaments. Since
permanent tastini are therefore not an ideal solution, the ``tape and toothpick'' approach might be
the best option.

For players who wish to extend a lute's capabilities even further, and avoid
some of the problems associated with tastini, we can make available chromatic and diatonic
semitones on all courses. Recalling Bermudo's example of an additional
\textit{mi} fret in figure~\ref{fig:bermudo-1-mifa}, if an extra fret spans the entire
width of the fretboard, we have the option of playing either semitone on every course.
Bermudo seems to suggest that players at the time were comfortable doing this with their left
hand.  The scale of the instrument would have a direct effect on the difficulty of
such a technique.  For example, on smaller lutes with a mensur of 60 centimeters or
less, the distance between a diatonic and chromatic semitone in quarter-comma meantone
at the first fret--where the distances are greatest--is just over a centimeter.
Depending on the size of one's fingers, it might be challenging to finger the diatonic
semitone in this space.  On instruments with a longer mensur length, such as the
theorbo where distances of 85 centimeters or larger are not uncommon, this distance
increases to almost two centimeters.

Instruments with longer mensur lengths have a greater distance between their semitones and therefore
could hold a small advantage in situations where different semitones are required. The larger
distance would make it easier for an experienced player to choose the diatonic semitone over the
chromatic one, and vice-versa. The thickness of the fret would also be a factor because the
chromatic fret would need to be slightly larger in diameter than the diatonic one. If this were not
the case, and the chromatic fret were smaller in diameter, then the thicker diatonic fret would
vibrate against the course if it were stopped using the chromatic fret.  While a larger diameter
chromatic fret would prevent this, it would also decrease the available fretboard space between
pitches, making larger instruments with a longer mesur an ideal choice for
enharmonic fretting solutions.  Ensemble situations are often the main cause for
problems reconciling temperaments with fretted instruments.  Theorbos and archlutes, which also have longer mensur lengths,
can enjoy preferred status not only because of their increased volume due to their body size and
string length, but also because they may navigate issues of semitone size better than some of their
smaller counterparts.

\section{Playing with Ensembles}

When performing with large ensembles, or groups with keyboards and other fretted instruments,
players must take special care to ensure that each musician's tuning and temperament matches
everyone else's. Typically, this means that the ensemble must agree about the temperament, 
and tuning and fret placement will be done according to that choice.  For example,
let us suppose that the temperament of choice for an ensemble is quarter-comma meantone.  Prior to
rehearsal, we would choose either a fixed-fret or enharmonic strategy, and use measurements or an
electronic tuner to place our frets and tastini in the correct positions.  At rehearsal time, we
would then tune our instrument to the appropriate pitch level, and hope for the best!  However, in
an actual rehearsal situation, things usually do not always work this way.

In ensembles with keyboard instruments, the reality is you are at the mercy of the keyboard tuner.
I have been fortunate to have worked with many excellent tuners that can tune an instrument in any
temperament.  However, keyboards sometimes are not tuned regularly, and heat or humidity changes in
the room in which the instrument is kept can have drastic effects on the pitch and stability of the
instrument.  No matter if a keyboard was tuned correctly to quarter-comma meantone, a day or two
later, it will have changed slightly with any change in the weather or climate conditions, and we
as lute players must adapt to those changes.

The tenuous nature of keyboard tuning, especially harpsichords, means that lutenists must be able to tune their instruments
to the keyboard, and more importantly, adapt to a temperament that may not necessarily be found
in a chart of fret measurements or on the dial of an electronic tuner.  If we arrive at a rehearsal
with our instrument correctly tuned to quarter-comma meantone, if the keyboard is not, then
we and the rest of the ensemble must adjust our frets to match its tuning, whatever that is.
In order to do this, it is important to remember the general concepts of temperaments on a lute,
rather than focus on technical specifics, such as determining the exact temperament of the 
keyboard.  Knowing the name of the keyboard's temperament is not as important as being able to
play in tune with it.

When faced with a keyboard in an unknown or unspecified temperament, the first thing to do is
ascertain the overall pitch level of the instrument.  Is its A tuned to 415 Hz, to 440 Hz, or
somewhere in the middle?  You will need to match the pitch level of your instrument to the keyboard.
Next, examine the more common semitones such as F$\sharp$, C$\sharp$, and B$\flat$, then compare
them to those found on your instrument or to an electronic tuner.  Do they match your pitches, or
the quarter-comma semitones as indicated on the electronic tuner?  If they do not, then you will
need to adjust your frets accordingly.  You will also need to determine the qualities of the
remaining semitones.  Does the keyboard have an E$\flat$ instead of a D$\sharp$, or a G$\sharp$
instead of an A$\flat$, and where is its wolf fifth?  Finally, and most importantly, are the
semitones consistent across all the octaves of the instrument?  This can help you and the other
members of the continuo section determine who is best equipped to play specific accidentals in the
music.

If the keyboard's tuning no longer matches yours, then you must discard
all of the fret placements you calculated earlier; however, this is not as drastic as it may seem.
Even if the keyboard has morphed into some unknown temperament, whether by climate change or a
tuning error, adjusting a few frets on your instrument will probably be enough to correct the
situation.  Assuming you are using a theorbo, tune your open A courses to the keyboard.  Since the
first fret is always a diatonic semitone, adjust the first fret of the theorbo to match the B$\flat$
of the keyboard.  Because this will affect the tuning of the seventh and eighth courses, you will
need to make additional corrections.  If your eighth course is on the long neck of the instrument,
you can simply tune the course to either F or F$\sharp$ to match the keyboard.  If your seventh and
eighth courses are on the fingerboard, then you can either slant the fret at an angle or adjust your
tastini to produce a G$\sharp$ and F$\sharp$ that match the keyboard.  If you are short
on time, or your tastini do not cooperate, you can elect to use an A$\flat$ for the seventh course
and adjust your sixth fret so that you have the G$\sharp$ available on the fifth course (see
figure~\ref{theorbo-slanted-sixth}).  You can make similar adjustments to the fourth fret to produce
an F$\sharp$ or you may tune your eighth course to F$\sharp$, even if it is on the fretboard.  You
will have to remember, however, that the F$\natural$ is no longer available.  This is perhaps why
some seventeenth-century theorbists used a F$\sharp$ for their lowest course, instead of the low
octave G, so that they could have both the F$\natural$ and the F$\sharp$ available as open courses.

After setting your first fret and determining how you will fret chromatic semitones of the lower
courses on the fingerboard, only a few other frets remain to check and adjust, if necessary.  The
second fret will be chromatic because it contains the F$\sharp$ and C$\sharp$; you will need to
ensure that these match the keyboard as well. Finally, check all the open strings of your instrument
and tune them to the keyboard. With your frets properly adjusted, if the keyboard is in a consistent
temperament, then your instrument should match the keyboard.

In an ideal world, the keyboard is kept in the proper conditions and tuned regularly so that its
temperament will be stable and you will not have to make any drastic adjustments to your frets other
than tuning your instrument's overall pitch level to that of the keyboard. Even if this is the case,
there will still be certain pitches, chords, and voicings that simply sound better on one instrument
and not the other.  In this scenario, one solution is for the continuo section to divide the
responsibilities of each player so that problematic figures go to the instrument with the best
tuning for that particular figure.  A lutenist can also change the voicing of a chord so that
offending notes are not placed in the highest tessitura of the chord, where they are easily heard,
but instead, placed in the lower registers where they are obscured from the listener.

In cases where your instrument's chromatic notes or accidentals do not agree with another
instrument's, despite your corrective steps, you can always omit them. Andreas
Werkmeister, whose irregular temperaments are an excellent choice in seventeenth-century repertoire,
published a treatise on accompaniment which is partially translated by F. T. Arnold:
\begin{blocks}
It is also not advisable that one should always just blindly play, together with the vocalists
and instrumentalists, the dissonances which are indicated in in the Thorough-Bass, and double
them
\autocite[210]{FTA:1}
\end{blocks}
Often, these dissonances are semitones that may be either diatonic or chromatic
in nature; therefore, if there is any disagreement between you and the other
instruments, it is best solved by omitting the offending note. This technique
can be applied to other notes which may not match among different
instruments, such as the thirds of chords.  In that case, the lute player or
keyboardist can elect to play an open fifth, and allow another instrument or the
soloist to complete the harmony with the third.  In this way, disagreements of
temperament are solved by avoiding the confrontation altogether.

Above all else, the most important aspect of utilizing temperaments when performing with ensembles
is that your instrument should sound its best. For example, even with all the instruments of an ensemble tuned
precisely to quarter-comma meantone, there still may be disagreements in pitch simply due to the
proclivities of each instrument. This is why all our practical examples and the historical evidence
support a great amount of flexibility in applying tuning and accompaniment techniques. There is no
doubt that musicians at this time employed unequal temperaments, but they were also very aware of
the difficulties that these temperaments could create.  We must, then, acknowledge that they too
grappled with them and were still able to create exquisite and beautiful music.

\section{Conclusion}

Regardless of the instrument's size, tuning, or repertoire, lute players have many different options
when it comes to choosing a temperament.  One can review all the historical evidence, and propose as
many different fretting solutions as one is able, but the principle will always remain the same: the
choice of fretting rests with the player. The sources are quite clear that there were a
multitude of different choices at one's disposal, some of which aroused heated arguments between
musicians. The same is true today and one can find very spirited discussions of tuning and
fretting in various online forums dedicated to topics on tuning and lute playing. Similar arguments
can be found in print too, and that will never change. \autocite{DD:4}

What is certain is that temperaments with unequal semitones were an aesthetic choice for musicians
during this time.  While the idea that a Pythagorean wholetone, which could not be divided equally,
provided a theoretical argument against the use of equal temperament, when mathematically equal
divisions of the semitone appeared later in the seventeenth century, with the advent of complex
algorithmic functions, musicians still favored their unequal semitones because they found the
resulting temperaments more aesthically pleasing, and congruent with the musical ideals of
affect and dissonance. Methods for approximating
equal temperament were widely known as early as the fifteenth century, yet the resistance to
adopting them was a clear indication that they were considered inferior to other temperaments that
created purer thirds, and more pungent dissonances, at the expense of limiting key choices.

Another fundamental aspect to unequal temperaments is the quality they give to different pitches and
keys.  In meantone temperaments, the difference between F$\sharp$ and B$\flat$ can be distressing,
an effect that composers utilized.  They chose certain keys based on the intervals and dissonances
within them, and used them to excite the  senses and provoke the passions. Unequal temperaments
gave composers the expressive power to do this. Today's modern ideals do not cultivate these
effects and the differences between semitones are homogenized, making all keys and accidentals
equal.  However, in the sixteenth and seventeenth century, differences between semitones created
differences between keys and spawned perceptions of the ``characters of the keys'' or idea that C
major has a difference quality that is fundamentally different than A major, beyond the simple issue
that one is higher or lower in pitch than the other.

Temperament was a concept that pervaded all music during this time, and we who have chosen to 
specialize in the field, must recognize its importance in imbuing music with special expressive
qualities.  This requires careful consideration and a deep understanding of the issues at work.  As lute
players, we have additional obligations because we have the option of curtailing some of these
aesthetic goals for practical solutions.  It seems that lutes and other fretted instruments were
granted a kind of immunity and were allowed to tune equally.  Botttrigari seemed to think so, and he
was probably not alone in his beliefs. \autocite[19]{HB:1}  However, this does not absolve us from
attempting to meet these aesthetic goals in the first place.

Modern conveniences, such as electronic tuners, can make it seem easy to choose one temperament over
another, and then move between them; however, our colleagues in the past had no such tools and
relied on the their ears.  This required a flexible approach from lute players who needed to match
with different kinds of ensembles.  The sources examined here bear this out, with multiple methods
for setting frets, and iterative techniques that instruct the player to try different positions and
decide on placements that sound the best.  Even these treatises disprove the notion of an exact
placement scheme. For example, every source instructs players to place the fifth fret at a pure
ratio, making it an un-tempered perfect fourth.  Because players use the fifth fret to tune the open
strings, such an interval would create havoc when tuning all the courses. The major third between
the fourth and third courses would be intolerably wide!  Tempered keyboard instruments could have
offending intervals, such as wolf fifths or other unusable thirds, hidden away in less obvious
places.  Lutes did not always have this ability.

Setting temperaments on the lute requires diligence and a great amount of experimentation to yield
effective results.  While we are allowed to use a kind of equal temperament, it is never the
preferred choice.  As lute playing evolves through the 21\textsuperscript{st} century, we will
continue to re-examine these arguments, and hopefully reach the same conclusions. The lute can and
did play in a variety of temperaments, both in solo and in ensemble contexts, but the effectiveness
of its performance shall always rest with the skill of the player. No one would ever question the
choice of temperament in performance unless it distracted the listener from the quality of the music
because it was badly tuned or was the wrong choice for the repertoire.  Therefore, the choice  of
temperament should be based upon what would best serve the music, by enhancing its expression and
giving it the warmth and brilliance it justly deserves.
