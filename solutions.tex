\chapter{Summary of Solutions}

Lute players today who want to perform music from the sixteenth and seventeenth
centuries must address the issue of using historically appropriate temperaments on
their instruments. Because of the nature of fretted instruments, using temperaments
with unequal semitones requires a different approach than is used with keyboards or
other instruments. In some ways, lutes have greater flexibility with temperaments than
keyboards do, but this flexibility results in a multitude of choices and options that
players must consider when fretting their instruments. The choices can also reflect
personal opinions rather than historical accuracy.

To realize temperaments with unequal semitones on the lute, there are two basic
approaches, both of which are supported by historical evidence. The first is the
``fixed semitone'' method where each fret is one type of semitone: chromatic, diatonic,
equal or something unique. The second approach treats the lute as an enharmonic
instrument, similar to harpsichords with split keys. In this method, two kinds of
semitones, such as the diatonic and chromatic, are available at
certain locations on the fretboard using split or slanted frets, or with the aid of
tastini. The choice is entirely at the discretion of the player, and either
technique can work regardless of the context in which it is used.

In this concluding chapter, I will elaborate on the flexibility of these approaches and
describe scenarios in which either can be employed successfully. The nature of lute
temperaments as this time indicates that they were not universally applied in the same
way. While the modern musical world relies on established standards of pitch as well
as temperament, such standards did not exist at this time in musical history. Pitch
could vary from city to city, and even from ensemble to ensemble. This same variation
applied to temperaments as well, not only between different ensembles, but between
different instruments in the same ensemble. This is perhaps why the lute held such an
important place because it could conform to the different requirements of temperament
more easily than keyboards; however, this is not to say that specialized lute
temperaments were only for ensemble music.

\section{Frets with fixed semitones}

As lute players, we have the option to ``fix it and forget it'' or in other words, set
our frets to a certain position and leave them. This requires a customized kind of
fretting system in which we accept the limitations of the temperament and either set
our frets in a way that makes the entire fretboard available to us or pick and choose
which semitones we want to play on which fret. The historical sources presented in
chapter two showed a predominant use of these kinds of customized temperaments
consisting of different kinds of semitones used throughout the octave. They are neither
completely one kind of meantone, such as quarter-comma or sixth-comma, nor are they
consistently Pythagorean or equal in nature.

Evidence clearly indicates that fretting systems approximating equal temperament had
their place and purpose on the lute, while most other instruments preferred to use a
non-equal temperament.  So why is it that fretted instruments held this exception? The
reason seems to be that it was simply easier for a lutenist to divide the octave into
twelve approximately equal semitones than it was for a keyboardist. A keyboardist would
temper intervals aurally, counting beats between notes, while a lute player could
visualize the fret distances and create quasi-equal semitones by making a good
visual approximation between two existing frets and placing the fret somewhere in the
middle.

The visual method of approximating equal semitones could also take an iterative approach as well,
wherein the fret is adjusted several times while playing to find the position at which the semitones
sound the best. Whether by visual placement or trial and error, the semitones that resulted from
such methods were not true Equal Temperament, as today's modern standard, but where an ``equal-ish''
temperament. These kinds of customized temperaments would be irregular in nature, with semitones of
varying size throughout the octave.  Depending on the semitone, however, they would be much closer
to modern-day equal temperament than some of the existing meantone temperaments of the time.

The use of temperaments that could approximate equal semitones was a distinguishing
feature of the lute and would have appealed to amateurs. The treatises from which we
take our fretting instructions were often written for amateur musicians and intended
for their education. A simple fretting solution that yields an equal-ish temperament
would simplify matters for someone who was still learning how to play the instrument.
Once experienced enough, however, the player could start moving frets around to his or
her liking creating a temperament that could selectively use different sized
semitones. Bermudo describes this when he refers to musicians who move their
frets according to their ears, but he instead wants to make the vihuela ``more
prefect'' with equal semitones for inexperienced players.\autocite[78]{DE:1}

Equal-ish temperaments and other customized schemes would also lend themselves to solo
lute music and repertoire for small ensembles such as the lute and voice. For example,
the Dowland song \textit{Come heavy sleep} in the previous chapter would benefit
from a temperament with fifths not quite as flat.  In such a temperament, 
the D$\sharp$ and F$\sharp$ would not sound quite as strident as
they would if were in quarter-comma meantone. Some of the custom fretting solutions
discussed in the previous chapter could work in this regard, with fifths wider that
quarter-comma, although yet another solution is to use a different type of
meantone.

Different varieties of meantone were presented in chapter one and in chapter two we saw
that many of the historical fretting sources preferred sixth-comma meantone for certain
frets.   Sixth-comma figured quite prominently in these for the first, third and sixth
frets (see table~\ref{table:comparison}). A similar temperament was also typical for
other fretted instruments such as cittern, which was commonly tuned using a
temperament somewhere between quarter-comma meantone and equal temperament.
\autocite[12]{PF:1}  Sixth-comma meantone is essentially the midway point between
quarter-comma meantone and equal temperament because its thirds are tuned slightly
wider than pure, but not as wide as in equal temperament. Other meantone
temperaments whose thirds were wider than pure, and therefore similar to a sixth-comma
and other meantone varieties, work well in early sixteenth-century repertoire such as
the music of vihuela composers Alonzo Mudarra and Luis Milan.\autocite[56]{WH:1}
Lastly, we can refer to Praetorius, in his \textit{Syntagma Musicum}, describing
frets with 4$ \frac{1}{2} $ commas.\autocite[68]{MP:1} Although he is describing a
semitone divided equally, it is still sixth-comma in nature because the wholetone has 9
parts, indicating that the frets were initially set in that kind of temperament.

Sixth-comma meantone is an excellent choice for fretted instruments because it does not restrict the
instrument as much as quarter-comma does, and generally does not require many modifications to make
it successful. At the same time, it makes a good compromise between equal semitones and the pure
thirds of quarter-comma meantone, allowing the lute to play all the semitones in the octave without
much difficulty. Irregular temperaments, such as Vallotti and Werkmeister, that share many of the
characteristics of sixth-comma meantone were used well into the seventeenth and eighteenth
centuries. These temperaments lend themselves quite well to the lute and I can speak both personally
and anecdotally that many lutenists prefer to use either a regular sixth-comma meantone or its
irregular varieties as their default temperament for most music.

Custom fretting solutions and temperaments such as sixth-comma meantone are ideal
for the lute in either solo music or small ensembles.  For example, in lute song
repertoire and in ensembles with additional fretted instruments, or instruments that do not
have a fixed semitone size, such as a violin or wind instrument, customized temperaments
do not present a problem because the instruments can conform to the temperament of
the lute or other fretted instruments. Fixed-fret solutions, however, become
problematic when using keyboards that are not tuned in sixth-comma, or if the members
of the ensemble prefer quarter-comma meantone over other temperaments.

It is possible to play a lute in quarter-comma meantone using a fixed system with one semitone per
fret; however, players must restrict the placement of certain semitones
on the fretboard.  They should choose the semitones needed for a particular
piece, and place them in their designated location on the fretboard.  For a theorbo
in A, for example, we can set our frets using the standard quarter-comma fretting
pattern for lute described in chapter three (see figure~\ref{fig:quarter-diatonic-complete-a}
in the appendix for the complete fretting).  In this pattern, C$\sharp$ and F$\sharp$
are found on the fourth fret, and if we slant our sixth fret as described in
figure~\ref{theorbo-slanted-sixth}, we would also have the G$\sharp$ as well. Depending
on the piece of music, that might be sufficient for our needs.

If the needed semitones vary from piece to piece, or between sections of a larger work
such as an opera, players could change the position of the frets if they have enough
time when not playing. Additional solutions could involve slanting
other frets, either before or during performance, or choosing to tie a double fret so
that it can be split and one side moved higher or lower than the other. Using varied
pressure with the left-hand to alter the pitch is another possibility, but this
has a limited effect as it will also reduce the resonance of the stopped note, muffling
it slightly. Beyond these types of surface fixes, we must turn to extended methods if
we want to have more than one kind of semitone at a fret.

\section{Enharmonic fretting}

If we truly need two different kinds of semitones available to us, and a fixed-fret
approach does not work, then we must turn to using either tastini or additional frets
tied to the fingerboard.  This offers the best possible solution and treats the lute as
an enharmonic instrument, allowing both chromatic and diatonic semitones to be
played. The drawback is that it increases the complexity of the instrument substantially,
and requires added technical skills on the part of the player. Players who work in
ensembles that use temperaments such as quarter-comma will opt for enharmonic fretting
if they wish to have a complete working solution. However, anyone is free to use an
enharmonic fretting in solo literature as well.

Generally speaking, one tastino will make a chromatic semitone for a single course, while the
remaining courses are diatonic.  A common example of this are the tastini found between the nut and
first fret, as we saw in figure~\ref{fig:theorbo-tastini}. Whether stopping a course using a fret or
tastino, the best resonance is created on the instrument when the finger of the left-hand stops the
course just behind the  fret or tastino, being as close to it as possible without going past it.
Usually, this process results in one kind of semitone per course so that if a tastino was used at
one course, that course would contain a chromatic semitone, while the others were diatonic.
However, some players  cultivate the skill of playing both the diatonic and chromatic semitones on a
course where a tastino is used. In that case, the diatonic semitone is played with the finger placed
between the tastino and main fret,  and the chromatic with the finger placed just behind the
tastino.  The skill involved is to correctly navigate the space between the fret and tastino, and to
have enough space between the two, and a small enough finger, to stop the course correctly and
achieve a good sound.

Tastini can be affixed to the fretboard in various ways. Historically, ivory was used,
but today, a toothpick and a piece of double-sided tape will work just just as well.  The main
disadvantage of tastini, once a player is accustomed to using them, is that they may
slip or break.  More permanent solutions are possible, such as using glue and a more
durable wood, but this forces the instrument into a particular temperament.  If
professional needs dictate tuning in a variety of temperaments, permanent
tastini may not be the ideal solution, in which case the ``tape and toothpick''
approach will be the only option.

For players who wish to extend a lute's capabilities even further, as well as avoiding
some of the problems associated with tastini, we can have chromatic and diatonic
semitones available on all courses. Recalling Bermudo's example of an additional
\textit{mi} fret in figure~\ref{fig:bermudo-1-mifa}, if an extra fret spans the entire
width of the fretboard, we have the option of playing either semitone on every course.
Bermudo seems to suggest that players at the time were comfortable doing this with their left
hand.  The scale of the instrument would have a direct effect on the the difficulty of
such a technique.  For example, on smaller lutes with a mensur of 60 centimeters or
less, the distance between a diatonic and chromatic semitone in quarter-comma meantone
at the first fret--where the distances are greatest--is just over a centimeter.
Depending on the size of one's fingers, it might be challenging to finger the diatonic
semitone in this space.  On instruments with a longer mensur length, such as the
theorbo where distances of 85 centimeters or larger are not uncommon, this distance
increases to almost two centimeters.

Instruments with longer mensur lengths have a greater distance between their semitones and therefore
could hold a small advantage in situations where different semitones are required. The larger
distance would make it easier for an experienced player to choose the diatonic semitone over the
chromatic one, and vice-versa. The thickness of the fret would also be a factor because the
chromatic fret would need to be slightly larger in diameter than the diatonic one. If this were not
the case, and the chromatic fret were smaller in diameter, then the thicker diatonic fret would
vibrate against the course if it was stopped using the chromatic fret.  While a larger diameter
chromatic fret would prevent this, it would also decrease the available fretboard space between
pitches, making larger instruments such as the theorbo an ideal choice for enharmonic fretting
solutions.  Ensemble situations are often the main cause for temperament problems with fretted
instruments.  Theorbos and archlutes, which also have longer mensur lengths, can enjoy preferred
status not only because of their increased volume due to their body size and string length, but also
because they may navigate issues of semitone size better than some of their smaller counterparts.

\section{Conclusion}

Regardless of the instrument's size, tuning, or repertoire, lute players have many
different options when it comes to choosing a temperament.  One can review all the
historical evidence, and propose as many different fretting solutions as one is able,
but the principle will always remain the same: the choice of fretting will always rest
with the player. The sources are quite clear that there were a multitude of different
options available and certain options aroused heated arguments between musicians. The
same is true today as well. One can find very spirited discussions of tuning and
fretting in various online forums dedicated to topics on tuning and lute playing.
Similar arguments can be found in print too, and that will never change. \autocite{DD:4}

As lute playing continues to evolve through the 21\textsuperscript{st} century, we will continue to
rehash these arguments and hopefully reach the same conclusions. The lute can and did play in a
variety of temperaments, both in solo and in ensemble contexts, but the efficacy of its performance
shall always rest with the skill of the player. No one would ever question the choice of temperament
in performance unless it distracted the listener from the quality of the music.  Therefore, it
should make no difference what temperament we use, as long as it best serves the music, giving it
the warmth and brilliance it justly deserves.
