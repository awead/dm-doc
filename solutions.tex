\chapter{Solutions for Modern Lutenists}

Given the evidence presented here from historical sources and contemporary sources such
as Mark Lindley's book on historical temperaments for fretted instruments, there are
two clear points that emerge.  First, while most other instruments exclusively used meantone and
temperaments with semitones of varied sizes, lutes also used temperaments with equal
semitones.  This does not imply that they used our modern-day equal temperament, nor
does it imply that they their kind of equal temperament was used exclusively in the same
way that other instruments used non-equal temperaments.  What the evidence shows us is
that lute players had more options when it came to temperaments and that they could take
advantage of these options when it suited their needs.  The second important point
regarding lute temperaments is an indication of how temperaments were viewed as whole.

The nature of lute temperaments as this time indicates that they were not universally
applied in the same way as today's temperaments.  The modern musical world relies on
established standards of pitch as well as temperament.  Such standards did not exist
before the eighteenth century.  Pitch varied from city to city, and even from ensemble
to ensemble.  It seems that the same variation applied to temperaments as well.  A
further distinction is that not only could temperament vary from ensemble to
ensemble, it also might vary between instruments in the same ensemble.

% Examples:
%  - L'Orfeo, Act 5: different tunings, lute in A and lute in G
%  - solo music could have used a different temperament
%  -
%
%
%
%
%
%