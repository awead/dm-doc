\chapter{Summary of Solutions}

Lute players today who want to perform music from the sixteenth and seventeenth
centuries must address the issue of using historically appropriate temperaments on
their instruments. Because of the nature of fretted instruments, using temperaments
with unequal semitones requires a different approach than is used with keyboards or
other instruments. In some ways, lutes have greater flexibility with temperaments than
keyboards do, but this flexibility results in a multitude of choices and options that
players must consider when fretting their instruments. The choices can also reflect
opinions of personal choice and go beyond the importance of historical accuracy.

To realize temperaments with unequal semitones on the lute, there are two basic
approaches, both of which are supported by the historical evidence. The first is the
``fixed semitone'' method where each fret is one type of semitone: chromatic, diatonic,
equal or something unique. The second approach treats the lute as an enharmonic
instrument, similar to harpsichords with split keys. In this method, two kinds of
semitones, such as the diatonic and chromatic semitone, are available at
certain locations on the fretboard using split or slanted frets, or with the aid of
tastini. The choice of method is entirely at the discretion of the player, and either
technique can work regardless of the context in which it is used.

In this concluding chapter, I will elaborate on the flexibility of these approaches and
describe scenarios in which either can be employed successfully. The nature of lute
temperaments as this time indicates that they were not universally applied in the same
way. While the modern musical world relies on established standards of pitch as well
as temperament, such standards did not exist at this time in musical history. Pitch
could vary from city to city, and even from ensemble to ensemble. This same variation
applied to temperaments as well, not only between different ensembles, but between
different instruments in the same ensemble. This is perhaps why the lute held such an
important place because it could conform to the different requirements of temperament
more easily than keyboards; however, this is not to say that specialized lute
temperaments were only for ensemble music.

\section{Frets with fixed semitones}

As lute players, we have the option to ``fix it and forget it'' or in other words, set
our frets to a certain position and leave them. This requires a customized kind of
fretting system in which we accept the limitations of the temperament and either set
our frets in a way that makes the entire fretboard available to us or pick and choose
which semitones we want to play on which fret. The historical sources presented in
chapter two showed a predominant use of these kinds of customized temperaments
consisting of different kinds of semitones used throughout the octave. They are neither
completely one kind of meantone, such as quarter-comma or sixth-comma, nor are they
consistently Pythagorean or equal in nature.

Evidence clearly indicates that fretting systems approximating equal temperament had
their place and purpose on the lute, while most other instruments preferred to use a
non-equal temperament.  So why is it that fretted instruments held this exception? The
reason seems to be that it was just easier for a player to divide the octave into
twelve approximately equal semitones than it was for a keyboard. A keyboardist would
temper intervals aurally, count beats between two notes, while a lute player can
visualize the fret distances and create quasi-equal semitones by simply make a good
visual approximation between two existing frets and placing the fret somewhere in the
middle.

The visual manner in which a player could approximate equal semitones could also take
an iterative approach as well, where the fret is adjusted several times while playing
to find the right spot in which the semitones sound the best. Whether by visual
placement or trial and error, the semitones that resulted from such methods were not
true Equal Temperament, as today's modern standard, but where an ``equal-ish''
temperament. These kinds of customized temperaments would be irregular in nature, with
semitones of potentially varying size throughout the octave thus still technically
remaining slightly unequal by nature. Depending on the semitone, however, they would be
much closer to modern-day equal temperament than some of the existing meantone
temperaments of the time.

The use of temperaments that could approximate equal semitones was a distinguishing
feature of the lute and would have appealed to amateurs. The treatises from which we
take our fretting instructions were often written for amateur musicians and intended
for their education. A simple fretting solution that yields an equal-ish temperament
would simplify matters for someone who was still learning how to play the instrument.
Once experienced enough, however, the player could start moving frets around to his or
her liking creating their own temperament that could selectively use different sized
semitones. This is what Bermudo describes when he refers to musicians who move their
frets according to their ears, but he instead wants to make the vihuela ``more
prefect'' with equal semitones for inexperienced players.\autocite[78]{DE:1}

Equal-ish temperaments and other customized schemes would also have had a strong appeal
in solo lute music and small ensembles such as the lute and voice.  The example of the
Dowland song \textit{Come, heavy sleep} in the previous chapter, would benefit from a
temperament that did not tune its fifths quite as flat.  For example, by using
less-tempered fifths, the D$\sharp$ and F$\sharp$ would not sound quite as strident as
they would if were in quarter-comma meantone.  This creates the notion that, in
addition to quasi-equal and custom temperaments, lutes also were able to use meantone
temperaments alternative to the common quarter-comma meantone of the time.

The use of various customized quasi-equal temperaments on the lute did not discount
meantone as another solution for frets in a fixed system. As we saw in the previous
chapter, quarter-comma meantone fretting presented certain challenges but it was not
the only kind of meantone. In sixth-comma meantone, a lute can still effectively play
all the semitones in the octave without much difficulty. Sixth-comma semitones figured
quite prominently in the historical sources for the first, third and sixth frets (see
table~\ref{table:comparison}). We also find that sixth-comma was the typical
temperament for other fretted instruments such as cittern, which was commonly tunned
using a temperament somewhere in between quarter-comma meantone and equal temperament.
\autocite[12]{PF:1}  Additionally, meantone temperaments whose thirds were slightly
less than pure and therefore similar to a sixth-comma and other meantone varieties,
work well in early fifteenth-century repertoires such as the vihuela composers Alonzo
Mudarra and Luis Milan.\autocite[56]{WH:1}

Sixth-comma meantone was an excellent choice for fretted instruments because it did not
restrict the instrument as much as quarter-comma did, and generally did not require
many modifications to make it successful. At the same time, sixth-comma also made a
good compromise between equal semitones and the pure thirds of quarter-comma meantone.
Different varieties of sixth-comma meantone were used well into the seventeenth and
eighteenth centuries, such as Vallotti and Werkmeister. These temperaments lend
themselves quite well to the lute and I can speak both personally and anecdotally that
many lutenists prefer to use either a regular sixth-comma meantone or its irregular
varieties as their default temperament for most music.

For solo repertoire, custom fretting solutions and other temperaments such as sixth
comma meantone present no problem.  In lute song ensembles, or other ensembles with
additional fretting instruments or instruments that do not have a fixed semitone size
such as a violin or wind instrument, customized temperaments do not present a problem
either so on long as any other fixed semitone instrument can conform to the same
temperament as the lute.  Keyboards that wish to use different temperaments could also
match the same temperament as the lute.

For ensembles that wish to use quarter-comma meantone, it is possible to use a lute in
this temperament using a fixed semitone fretting. The player can choose where to put
the semitones she might need for a particular piece, and if the player is skilled
enough, the semitones can change rapidly, either between movements or sections of a
work, such as an opera or other long composition where the player might have several
minutes in which they were not playing. Alternatively, if another player wished to be
able to use the same temperament, but have more semitones available to him or not have
to rely on quick fret changes, he could use tastini or other extended methods of
fretting so that he would have his choice of either chromatic or diatonic semitone when
he needed it.

\section{Enharmonic fretting}

The approach that offers the best possible solution treats the lute as an enharmonic
instrument and allows both the chromatic and diatonic semitones to be played on the
same fret. The only drawback to the scheme is that increases the complexity of the
instrument substantially, and requires added technical skills on the part of the
player. Most players will opt for enharmonic fretting for ensembles that use
temperaments such as quarter-comma and wish to have a complete working solution.
However, players are free to use enharmonic fretting in the solo literature.

Generally speaking, one tastino will make a chromatic semitone for a single course,
while the remaining courses are diatonic.  A common example of this are the tastini
found between the nut and first fret.  Some players cultivate the skill of playing both
the diatonic and chromatic notes.  For the diatonic semitone, the left-hand finger is
placed in the area of the fretboard between the tastino and main fret.  For the
chromatic fret, the finger is placed just behind the tastino just like any other fret.
Recalling Bermudo's example of an additional \textit{mi} fret in
figure~\ref{fig:bermudo-1-mifa}, this extra fret spans the entire width of the
fretboard suggesting that players then were comfortable choosing either semitone with
the left hand.  Depending on the scale of the instrument, this distance can be small
which would make it difficult; therefore, it seems that only very experienced players
would develop such a skill. Other modifications of frets such as slanting or choosing
to tie a double fret so that it can be split, are also perfectly acceptable methods of
creating multiple semitone types per fret. Although, using these methods sometimes
precludes the ability to have both types available at the same location.

\section{Conclusion}

One can review all the historical evidence, and propose as many different fretting
solutions as one is able, but the principle will always remain the same. The choice
of fretting will always rest with the player. The sources are quite clear that there
were a multitude of different options available and certain options aroused heated
arguments between musicians. The same is true today as well. One can find very
spirited discussions of tuning and fretting on various online forums dedicated to
topics on tuning and lute playing. Similar arguments can be found in print too, and
that will never change. \autocite{DD:4}

As lute playing continues to evolve through the 21st century, we will continue to
rehash these arguments and hopefully continue to reach the same conclusions. The lute
can and did play in a variety of temperaments, both in solo and in ensemble contexts,
but the efficacy of its performance shall always rest with the skill of the player. No
one would ever question the choice of temperament from exquisite performance.
Therefore, it should make no difference what temperament we use, so long as it
represents the music giving it the warmth and brilliance that it justly deserves.