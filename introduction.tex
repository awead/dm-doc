\chapter*{Introduction}
\addcontentsline{toc}{section}{Introduction}

In 1594, the composer Hercole Bottrigari published 
\textit{Il Desiderio, overo de' concerti di varii strumenti musicali}, which 
translated means, ``concerning the playing together of various musical instruments.''
The book is presented as a conversation between two fictional characters, a common 
literary device at the time.  The main character in the conversation is
Alemanno Benelli, who represents the views of Bottrigari, and the second
is Gratioso Desiderio, who has sought out Benelli for answers about music,
specifically the performance of different instruments together in an ensemble.

In the course of their conversation, Benelli asks Desiderio to tune a lute to
the pitches of a harpsichord.  He begins by instructing him to tune his open
strings to the harpsichord and then asks him to try other pitches:

\begin{blocks}
\begin{center}
\begin{tabular}{r l}
Benelli:   & Now I play F fa ut on the [harpsichord]. Now touch                      \\
           & your F fa ut on the first fret of the [E string] of                     \\
           & your Lute. Are you tuned in unison with me?                             \\
Desiderio: & We are not.                                                             \\
\ldots     &                                                                         \\
Benelli:   & Now I touch the raised G sol re ut on the [harpsichord].                \\ 
           & Touch the fourth fret \ldots Are you in unison and                      \\
           & tunned with the [harpsichord]?                                          \\
Desiderio: & No, we are not. And what causes that, M. Alemanno? \autocite[18]{HB:1}  \\
\end{tabular}
\end{center}
\end{blocks}

Bottrigari is illustrating a common problem at this time: fretted instruments
were tunned according to a different standard than keyboards. He eventually goes
on to explain to Desiderio how lutes can play with harpsichords, but Benelli's
(\textit{i.e.} Bottrigari's) solution to the problem is only of many that  were
in use at this time.
  
Today, Desiderio would not have this kind of problem because all instruments are
tunned according to the same standard of equal temperament.  This has been the
case for most of the twentieth century, but when performing music from
Bottrigari's time, the equal temperament standard does not apply.  Although it
was known to musicians at time, equal temperament was not favored during the
sixteenth and early seventeenth centuries.  Instead, meantone temperament was
the preferred choice for instruments; however, it was not standardized in the
way that equal temperament is today.  Many kinds of meantone temperaments
existed during this period, and variants depended on the date, geographical
location and even the instrument.  These continued into the seventeenth and
eighteenth centuries, while newer, non-meantone temperaments were invented.  The
majority of these applied to keyboard instruments because of the fixed nature of
their pitches, but the same notion applied to all instruments that equal
temperament was not used.

The problem Desiderio faced was that lutes could have their frets set for a
variety of different temperaments.  In ensembles, they generally had to follow
the same meantone standard; however, the execution of a meantone temperament on
a lute differed from that performed on other instruments, hence the disagreement
between his lute and Benelli's harpsichord.  Keyboard instruments, for example,
had their own procedures for producing different versions.  Lutes and similar
fretted instruments could produce these as well, but the process and the
limitations a particular temperament imposed could differ. The implications for
a lute player at this time were that the player would need to alter his or her
technique, and in some cases, the instrument itself, in order to accommodate a
particular temperament. This raises the question of how performers at that time
dealt with the issue. More importantly, it should give today's performers pause
when determining what temperament to use.  Just as we cannot apply an all-
encompassing modern temperament to older music, neither can we apply an all-
encompassing historically-informed one either.

The principal difference between today's equal temperament and the majority of
those found at the turn of the seventeenth century is that the latter had
semitones of non-uniform or varied size.  This could either be regular, where
the distance between the semitones within an octave followed a predictable
pattern, or it could be irregular with the distance varying greatly from one to
the next. When tuning a keyboard instrument where each string or pipe can be
pitched independently of the others, this presents no problem.  Bowed
instruments without frets may alter their pitches with the placement of the
left-hand fingers.  Wind instruments have fingerings and the performer's
embouchure to provide small adjustments in pitch to match any kind of
temperament.

Because fretted instruments place their frets so that each intersects all of the
strings on the instrument, the inherent problem with attempting to use semitones of
varying size is that when you set the size of a semitone for one string of the
instrument, you have set the same sized semitone for all the other notes on the
adjacent strings. This creates unavoidable problems if the temperament you are trying
to emulate requires a small semitone on one string but a larger one
on the string next to it. The result is that as you progress down the fingerboard and
set the size of each of your semitones, whether they are equidistant or not, you must
choose what sized semitone benefits the majority of the pitches on that fret. If the
majority of the pitches on a fret are best suited to a small semitone, then you must
have a semitone of the same small size for all of those pitches. Alternatively, if the
majority of the pitches require a large semitone, then all the pitches get the same
sized, larger semitone.  The inherent problem is that there is usually one pitch on the
fret that does not fit and needs to have a semitone of a different size than
all the others.

Lute players have dealt with these problems of varied semitone size for centuries, and
while there is no perfect solution, there are many combinations of
possible solutions that produce good results. What follows is a detailed examination
of these solutions.  Before we get to these, however, we need a a brief
explanation of the history of tuning and temperament. Chapter one provides both
context and a working technical knowledge of temperament. The second chapter
examines historical fretting systems published during the sixteenth and early
seventeenth centuries. Here we will see how lute players at the time dealt with
temperament and the shortcomings of their solutions. Chapter three moves beyond
historical fretting systems and attempts to a provide unified fretting system for use
in performance by combining historical techniques with modern practices.  We will see
how historical fretting systems must be modified in order to be successful and what
other techniques may be used to achieve a workable system. The last chapter provides a
conclusion to our predicament and summarizes the many different ways lute players
may navigate temperaments in modern day performance.
