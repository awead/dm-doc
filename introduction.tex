% Chapter 1

\chapter{Introduction}

\section{Why Lutes Can't Play in Tune}

The pervasive attitude in today's early music ensembles is to use quarter-comma
meantone for most music written during the early seventeenth century.
While this temperament is absolutely correct for most melody and keyboard
instruments, it is neither historical appropriate for plucked instruments nor is
it a temperament that they may accommodate successfully.  While it is possible
for a lute or similar fretted instrument to conform to a quarter-comma meantone
temperament, the limitations that the temperament places on the instrument make
it very difficult for the perform to play in a normal fashion.  Standard chord
shapes and melodic patterns that are fundamental to a performer's technique must
be altered or even abandoned in the face of a temperament that leaves them
sounding unusable.  This raises the question of how performers at that time
dealt with the issue, and how we today may be bestowing a standardization of
temperament where there historically may not have been one.

The problem with tuning lutes is that when you set the fret for a chromatic or
diatonic semitone one one string, you have also set the same semitone on all
adjacent strings.  This creates an inherent problem because the temperament that
you may be trying to emulate might require a chromatic semitone on one string
and fret but a diatonic semitone on the another string using the same fret.  The
result is that as you progress down the fingerboard, you must make a decision
that benefits all the tones on adjacent strings the best. \autocite[120]{RD:1}
If the majority of the semitones on that fret are chromatic, then you must place
a chromatic semitone; otherwise, if the majority of the tones are diatonic, then
you must place a diatonic semitone.  The problem is that there is usually always
a semitone somewhere that looses out.

To tackle these issues, we should first examine the evidence that survives of
historical fretting techniques and compare them with the temperament standards
that existed for other instruments.  Next, we need to look at the methods
players at the time to modify their frets and accommodate difficult
temperaments.  Thirdly, how do these historical temperaments compare to how
lutes players today tune their instruments and what changes have been made.
Lastly, the vast amount of music that we have for lutes, both in ensemble and
solo, should be used as a litmus test to see if any of these historical or
modern temperaments are really successful enough in a concert situation.
\autocite[130]{RD:1}
