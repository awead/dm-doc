\chapter*{Introduction}
\addcontentsline{toc}{section}{Introduction}

Today's standard of tuning is equal temperament and just about every instrument that
anyone purchases is usually built with that temperament in mind.  This has been the
case for most of the twentieth century, but when performing music from earlier periods
in Western music literature, the equal temperament standard does not apply.  Generally
speaking, the standard for most music from the sixteenth and early seventeenth
centuries is meantone temperament, not equal temperament. However, meantone temperament
was not the same kind of standard in the way that equal temperament is today.  Many
different kinds of meantone temperaments existed during this period, and variations
depended on the time period such as early or late sixteenth century, the geographical
location and even the instrument.  As these variations in temperament continued into
the seventeenth and eighteenth centuries, newer, non-meantone temperaments were
invented and the options became even more numerous and complex with majority of them
applied to keyboard instruments because of the fixed nature of their pitches.

Fretted instruments such as lutes and theorbos also had to follow the same meantone
standard, but the execution of these temperaments on a lute differed from other
instruments. Keyboard instruments, for example, had their own procedures for producing
different varieties of meantone temperament.  Lutes and other similar fretted
instruments could conform to same some of the varieties, but the process was different
and the limitations a particular temperament imposed on a keyboard instrument could be
different than those imposed on a fretted instrument.  The implications for a lute
player at this time were that, depending on the temperament, the player would need to
alter his or her technique, and in some cases, the instrument itself, in order to
accomodate a particular temperament. This raises the question of how performers at that
time dealt with the issue.  More importantly, it should give today's performers pause
when determining what temperament to use.  Just as we cannot apply an all-encompassing
modern temperament to older music, neither can we apply all-encompassing
historically-informed one either.

The principle difference between today's equal temperament and the temperaments used at
the turn of the seventeenth century is that every temperament at that time had
semitones of non-uniform or varied size.  The variation could either be regular, where
the variation between the sizes of semitones within an octave followed a predictable
pattern, or the variation could be irregular with semitone size varying greatly from
one note to the next. When tuning a keyboard instrument where each string or pipe can
be pitched independently of the others, this presents no problem.  Bowed instruments
without frets may vary their pitches with the placement of the left-hand fingers.  Wind
instruments have fingerings and the performer's emboucher to provide small adjustments
in pitch to match any kind of temperament.

Because fretted instruments place their frets so that one fret intersects all of the
strings on the instrument, the inherent problem with attempting to use semitones of
varying size is that when you set the size of a semitone for one string of the
instrument, you have set the same sized semitone for all the other notes on the
adjacent strings. This creates unavoidable problems if the temperament you are be
trying to emulate requires a small semitone on one string but a larger, differently
sized one on the string next to it. The result is that as you progress down the
fingerboard and set the size of each of your semitones, whether they are the same size
or not, you must choose what sized semitone benefits the majority of the pitches on
that fret. \autocite[120]{RD:1} If the majority of the pitches on a fret are best
suited to a small semitone, then you must have a semitone of the same small size for
all of those pitches. Alternatively, if the majority of the pitches require a large
semitone, then all the pitches get the same sized, larger semitone.  The problem is
that there is usually always one pitch on the fret somewhere that does not fit and
needs to have a semitone of a different size than all the others.

Lute players have dealt with these problems of varied semitone size for centuries, and
there is no perfect solution to the problem, but there are many combinations of
different kinds of solutions that produce good results. To examine them in detail
requires a brief explanation of the history of tuning and temperament, an investigation
of historical fretting systems from the time and how they worked, and what it means to
realize a historically appropriate temperament on a lute today.  The first chapter
contains a brief summary of tuning and temperament, while the second chapter examines
instructions from the sixteenth and early seventeenth centuries on fretting lutes. In
this chapter we will see the shortcomings of many historical temperaments and why which
ones could work better than others. Chapter Three moves beyond historical fretting
systems and attempts to provide modern solutions by combining historical solutions into
a unified fretting system for use in performance.  We will see how historical fretting
systems must be modified in order to be successful and what other techniques may used
to achieve a working system. The last chapter provides a conclusion to our predicament
and offers some suggestions for how lutes players today may navigate temperaments in
modern day performance contexts. \autocite[130]{RD:1}
