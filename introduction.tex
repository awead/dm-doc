% Introduction -- shouldn't be a chapter

\section*{Introduction}
\addcontentsline{toc}{section}{Introduction}

\section{Why Lutes Can't Play in Tune}

The pervasive attitude in today's early music ensembles is to use quarter-comma meantone
for most music written during the early seventeenth century. While this temperament is
absolutely correct for most melody and keyboard instruments, it is neither historically
appropriate for plucked instruments nor is it a temperament that they can accommodate
easily.  While it is possible for a lute or a similar fretted instrument to conform to
quarter-comma meantone temperament, the process of realizing the temperament imposes
limitations on the instrument.  This can result in notes that are unusable in such
temperament that are otherwise quite commonly found in the instrument's repertoire where
another different kind of temperament is used.  Standard chord shapes and melodic patterns
that are fundamental to a performer's technique must be altered or even abandoned in the
face of a temperament that leaves them sounding unusable.  This raises the question of how
performers at that time dealt with the issue, and how we today may be bestowing a
standardization of temperament when historically, such a standard may have been in use.

Unlike today's modern temperaments, every temperament used during this time had semitones
of non-uniform or varied size.  The variation could either be regular, where the variation
between the sizes of semitones within an octave followed a predictable pattern, or the
variation could be irregular with semitone size varying greatly from one note to the next.
When tuning a keyboard instrument where each string or pipe can be pitched independently
of the others, this presents no problem.  Bowed instruments without frets may vary their
pitches with the placement of the left-hand fingers.  Wind instruments have fingerings and
the performer's emboucher to provide small adjustments in pitch to match any kind of
temperament.

Because fretted instruments place their frets so that one fret intersects all of the
strings on the instrument, the inherent problem with attempting to use semitones of
varying size is that when you set the size of a semitone for one string of the instrument,
you have set the same sized semitone for all the other notes on the adjacent strings. This
creates unavoidable problems if the temperament you are be trying to emulate requires a
small semitone on one string but a larger, differently sized one on the string next to it.
 The result is that as you progress down the fingerboard and set the size of each of your
semitones, whether they are the same size or not, you must which kind of semitone size
benefits the majority of the pitches on that fret. \autocite[120]{RD:1} If the majority of
the pitches on a fret are best suited to a small semitone, then you must have a semitone
of the same small size for all of those pitches. Alternatively, if the majority of the
pitches require a large semitone, then all the pitches get the same sized, larger
semitone.  The problem is that there is usually always one pitch on the fret somewhere
that does not fit and needs to have a semitone of a different size than all the others.

Lute players have dealt with these problems of varied semitone size for centuries, and
there is no perfect solution to the problem, but there are many combinations of different
kinds of solutions.  Some of these require rethinking our current approach towards
historical temperaments, particularly with ensembles.  To examine these solutions requires
a brief explanation of the history of tuning and temperament, what it meant to use
semitones of different sizes and which of these kinds of temperaments were most commonly
in use during the sixteenth and seventeenth centuries. After the first chapter is used to
explain historical temperaments, the second chapter will examine historical fretting
systems during this time and compare them with the temperament standards that existed for
other instruments at the time.  In this chapter we will see the shortcommings of many
historical temperaments and why which ones could work better than others. Chapter three
moves beyond historical fretting systems and attempts to provide modern solutions by
combining historical solutions into a unified fretting system for use in performance.  We
will see how these fretting systems must be modified in order to be successful and why
certain problems are unavoidable and have no real good solution. The last chapter provides
a conclusion to our predicament and offers some suggestions for how lutes players today
may navigate temperaments in modern day performance contexts. \autocite[130]{RD:1}
