% Modern solutions for fretting
% Outline:
%  1. Opening section
%     Brief outline of the chapter?
%
%     It is possible to tune a lute or theorbo in quarter-comma meatone, but historical fretting
%     systems are of no use, chord positions are limited and we often have to utilize techniques that historical sources
%     speak out against.
%
%  2. The Lute in Ensembles
%     If ensembles are playing in 1/4, then the lute has to as well. How?
%      - choosing where to place chromatic or diatonic semitones
%
%  3. The Theorbo
%
%  4. Modern alternatives
%      a. tuning with tech
%      b. alternating frets
%      c. tastini
%      d. the theorbo
%
%
%

\chapter{Modern Lute Fretting}

If we return to the initial inquiry at the outset of this paper, most ensembles in the
sixteenth and early seventeenth century relied on quarter-comma meantone as their standard
temperament. As we have seen in the previous chapter, none of the major sources on lute
fretting employ intervals that are quarter-comma in nature.  This presents any would-be
lute player with a large gap that he or she would need to cross before being able to
participate in any ensemble from this period.

Because of the arrangement of semitones on a lute, meantone temperaments can only be
executed in a limited way.  Players in the sixteenth century recognized this limitation
and proposed various ways of dealing with the problem.  Some of them ultimately decided
that the lute was simply an equal semitone instrument, or others observed that it was
simply different and that music sounding bad on one instrument seemed to sound much better
on a lute.\textit{Citation needed} However, we are still faced with the reality that lutes
performed with other instruments in large and small ensembles, all of which very likey
used quarter-comma meantone temperament.  So it seems to reason that players had found a
way to deal with the problem at least in some way.

In this chapter we will examine how and why the different historical fretting solutions
set forth in the previous chapter fail to address the problems posed by quarter-comma
meantone and look at alternative solutions, some of which are modern and discount the
approaches of the treatieses from the period. The historical fretting solutions we
examined earlier that attempt unequal semitones come close to sixth comma meantone, but
not quarter comma and would therefore not be suitable in a ensemble context.  This does
not mean, however, that a quarter-comma meantone temperament is not possible on the lute.
Instead we must use different methods and approaches that historical sources neglect, such
as the use of a diatonic semitone for the first fret instead of a chormatic one which so
many of the sources seem to favor.

Other solutions to the deliemma of meantone temperament on the lute involve the use of
\textit{tastini} or ``little frets'' made of wood that only span one string and can be
glued on the fret board.  These small frets make it possible to have a different kind of
semitone on one string, such as a chromatic semitone, while the rest of the strings all
have the diatonic semitone.  A second solution involves the use of frets placed angles so
they are not parallel to the nut or bridge.  This results in a fret that could be a
diatonic semitone on one end and a chromatic one on the other end.  The arguments against
such solutions found in historical sources support the notion that the practice was
commonly used; however, it was problematic enough that some authors sought to rally
against it. Nevertheless, we must consider these solutions if we are able to find a
musically acceptable solution to the quarter-comma dilemma.

Every solution to the problem of meantone temperaments on the lute is a kind of
compromise.  Even less devisive solutions included alternative frettings for certain notes
and varying the finger pressure of the left hand to alter the pitch enough to make it
agreeable in a tempered setting alter the instruments ability to function in given
context.  It is possible to play in quarter comma meantone on the lute, but as we shall
see, it does limit some of the instrument's capabilities.

\section{The Lute in Ensembles}

Any kind of fretted instrument performing late 16th or early 17th-century music will need
to utilize quarter-comma meantone temperament.  This was the standard temperament for
keyboard instruments and most, if not all, of bowed string and wind instruments as well.
For lutes, the problem is that none of the available historical fretting instructions
mention quarter-comma meantone at all.  The fretting schemes that were presented in the
previous chapter make use of tempered frets, some of which fall near sixth-comma meantone,
but none use full quarter-comma meantone fifths or thirds.  Either the history evidence is
lacking or there were cases when a lute was tuned to a different temperament than the rest
of the ensemble.

Most musicians today are uncomfortable with the idea of deliberately playing in a
different temperament than the rest of the instruments in an ensemble, so lute players
have found ways in which to make quarter-comma meantone successful on their instruments.
Such as temperament is possible on fretted instruments, but in the case of the lute, it
creates other problems such as limiting the instrument's compass and its ability to play
idiomatically.  As meantone frets limit the types of semitones one may play on a fret,
this also limits the types of chords one may use as well, specifically certain left-hand
chords shapes that are commonly used.  When used as a continuo instrument, the chord of
chord shape is completely at the player's discretion, so long as it agrees with the
harmony.  Since the player has more control over the number of voices in the chord and its
location on the fretboard, it is possible to overcome the limitations of a meantone
fretting system when playing in ensembles.  In solo situations, however, this is not
possible since the exact location of each pitch is dictated by the tablature and as we
will see, tablature often indicates frets that would not be in tune if a meantone fretting
system was used.

Before we examine meantone fretting in solo music, we can look at how quarter-comma
meantone can be realized on the lute.  If we recall, the main feature of meantone
temperaments are thirds that are much closer to pure than Pythagorean tuning or equal
temperament.  The two side-effects of this are narrow fifths, or fifths that are flatter
than pure, and unequal semitones, namely the diatonic and chromatic semitone.  For the
keyboard, this meant having to choose between an F$\sharp$ and a G$\flat$, or a D$\sharp$
and an E$\flat$.  In any meantone temperament, these were two different notes.  When
tuning a keyboard, once the choice is made a chromatic or a diatonic semitone, this choice
can be replicated consistently in all the octaves of the instruments range and independent
from any other note on the instrument. On a keyboard, every F$\sharp$ will be the same in
every octave no mater if we happen to have chosen an A$\flat$ next to it or a C$\sharp$ a
fourth below.  On the lute, this is not possible because fret placement dictates the
semitone size for all notes along that fret.  In that case, the choice of a chromatic
semitone at the first course will mean the note at the second and third courses will have
to be chromatic as well.

For any temperament with unequal semitones, a lutenist can only choose between a chromatic
or a diatonic semitone when placing frets on his or her instrument.  For example, if were
fretting our instrument in quarter-comman meantone, the first fret would either be a
chromatic semitone, with three commas or a diatonic one with four commas.  The choice of
semitone is up to the player, but the main consideration the player should make is whether
or not any pitches on that fret must either be chromatic or diatonic.  For a lute in
standard G tuning, the second course is tunned to D; therefore, the first fret of the D
course could either be E$\flat$ or D$\sharp$, the second fret E and the third absolutely
has to be F.  The distance between E and F is a diatonic semitone which makes the distance
between the second and third fret a diatonic semitone.  Because of this one requirement in
tuning between two notes on a single course, the distance of a diatonic semitone between
these two frets will apply to all other courses as well.

The other consideration players can make when choosing semitone size is that some
chromatic notes are more common than others.  For examnple, it is more common to find an
F$\sharp$ in 17th-century music than it is a G$\flat$.  Returning to our previous example
of the D course, the fourth fret makes better sense as a chromatic semitone, giving us an
F$\sharp$, instead of a diatonic one which would have produced a G$\flat$.  If we use this
same logic and move back to the first fret of that course, we could choose an E$\flat$
over a D$\sharp$ by using the reasoning that we are probabaly more likely to encounter an
E$\flat$ than a D$\sharp$, although this is not always the case.

For the sake of argument, let us say that we have settled on the semitone choices we have
discussed above: diatonic first and third frets, followed by a chromatic forth fret.  The
result of these semitone sizes are summarized in figure~\ref{fig:quarter-diatonic}, and as
we can see, this has an interesting impact on the rest of the pitches on the other courses
of the same frets. \begin{figure}[ht]
\centering
\setlength{\unitlength}{0.5mm}
\begin{picture}(60,191.6)
% Draw fingerboard edges
\color{black}
\linethickness{0.075mm}
\put(0,0){\line(0,1){186.6}}
\put(60,0){\line(0,1){186.6}}

% Draw strings
% 6th course
\color{strings}
\linethickness{0.5mm}
\put(5,0){\line(0,1){186.6}}
\linethickness{0.25mm}
\put(7,0){\line(0,1){186.6}}
% 5th course
\put(15,0){\line(0,1){186.6}}
\put(17,0){\line(0,1){186.6}}
% 4th course
\put(25,0){\line(0,1){186.6}}
\put(27,0){\line(0,1){186.6}}
% 3rd course
\put(35,0){\line(0,1){186.6}}
\put(37,0){\line(0,1){186.6}}
% 2nd course
\put(45,0){\line(0,1){186.6}}
\put(47,0){\line(0,1){186.6}}
% 1st course
\put(56,0){\line(0,1){186.6}}
% Insert string pitch names for lute in G
% 6th
\color{black}
\put(2,191.6){\small{G}}
% 5th
\put(14,191.6){\small{c}}
% 4th
\put(24,191.6){\small{f}}
% 3rd
\put(34,191.6){\small{a}}
% 2nd
\put(43,191.6){\small{d'}}
% 1st
\put(53,191.6){\small{g'}}
\color{black}
\linethickness{1mm}
\put(0,136.2){\line(1,0){60}}
\color{black}
\put(60,135.2){\small{\textemdash  1st (diatonic)}}
\color{black}
\linethickness{1mm}
\put(0,107.6){\line(1,0){60}}
\color{black}
\put(60,106.6){\small{\textemdash  2nd (chromatic)}}
\color{black}
\linethickness{1mm}
\put(0,41.3){\line(1,0){60}}
\color{black}
\put(60,40.3){\small{\textemdash  4th (chromatic)}}
\color{black}
\linethickness{1mm}
\put(0,5){\line(1,0){60}}
\color{black}
\put(60,4){\small{\textemdash  5th (diatonic)}}
\color{black}
\linethickness{1mm}
\put(0,66.9){\line(1,0){60}}
\color{black}
\put(60,65.9){\small{\textemdash  3rd (diatonic)}}
\color{black}
\linethickness{1mm}
\put(0,181.6){\line(1,0){60}}
\color{black}
\put(60,180.6){\small{\textemdash  Nut}}
\color{black}
\put(2,158.9){\small{A$\flat$}}
\put(12,158.9){\small{D$\flat$}}
\put(22,158.9){\small{G$\flat$}}
\put(32,158.9){\small{B$\flat$}}
\put(42,158.9){\small{E$\flat$}}
\put(52,158.9){\small{A$\flat$}}
\color{black}
\put(2,23.1){\small{C}}
\put(12,23.1){\small{F}}
\put(22,23.1){\small{B$\flat$}}
\put(32,23.1){\small{D}}
\put(42,23.1){\small{G}}
\put(52,23.1){\small{C}}
\color{black}
\put(2,87.2){\small{B$\flat$}}
\put(12,87.2){\small{E$\flat$}}
\put(22,87.2){\small{A$\flat$}}
\put(32,87.2){\small{C}}
\put(42,87.2){\small{F}}
\put(52,87.2){\small{B$\flat$}}
\color{black}
\put(2,54.1){\small{B}}
\put(12,54.1){\small{E}}
\put(22,54.1){\small{A}}
\put(32,54.1){\small{C$\sharp$}}
\put(42,54.1){\small{F$\sharp$}}
\put(52,54.1){\small{B}}
\color{black}
\put(2,121.9){\small{A}}
\put(12,121.9){\small{D}}
\put(22,121.9){\small{G}}
\put(32,121.9){\small{B}}
\put(42,121.9){\small{E}}
\put(52,121.9){\small{A}}
\end{picture}
\caption{Standard Quarter-comma Fretting}
\label{fig:quarter-diatonic}
\end{figure}
 Since we have chosen E$\flat$ over
D$\sharp$ for the first fret, this results in an A$\flat$ for the third course, which is a
good choice; however, the fourth and fifth courses are G$\flat$ and D$\flat$, instead of
the more likey F$\sharp$ and C$\sharp$. This also creates a problem because the G$\flat$
found on the first fret of the fourth course will not match the F$\sharp$ on the fourth
fret of the second course. Similarly, the D$\flat$ on the fifth course will not match the
C$\sharp$ on the third.

The alternative solution is to have a chromatic semitone for the first instead of a
diatonic one, seen in figure ~\ref{fig:quarter-chromatic}, but this does not improve
things greatly. \begin{figure}[ht]
\centering
\setlength{\unitlength}{0.5mm}
\begin{picture}(60,191.6)
% Draw fingerboard edges
\color{black}
\linethickness{0.075mm}
\put(0,0){\line(0,1){186.6}}
\put(60,0){\line(0,1){186.6}}

% Draw strings
% 6th course
\color{strings}
\linethickness{0.5mm}
\put(5,0){\line(0,1){186.6}}
\linethickness{0.25mm}
\put(7,0){\line(0,1){186.6}}
% 5th course
\put(15,0){\line(0,1){186.6}}
\put(17,0){\line(0,1){186.6}}
% 4th course
\put(25,0){\line(0,1){186.6}}
\put(27,0){\line(0,1){186.6}}
% 3rd course
\put(35,0){\line(0,1){186.6}}
\put(37,0){\line(0,1){186.6}}
% 2nd course
\put(45,0){\line(0,1){186.6}}
\put(47,0){\line(0,1){186.6}}
% 1st course
\put(56,0){\line(0,1){186.6}}
% Insert string pitch names for lute in G
% 6th
\color{black}
\put(2,191.6){\small{G}}
% 5th
\put(14,191.6){\small{c}}
% 4th
\put(24,191.6){\small{f}}
% 3rd
\put(34,191.6){\small{a}}
% 2nd
\put(43,191.6){\small{d'}}
% 1st
\put(53,191.6){\small{g'}}
\color{black}
\linethickness{1mm}
\put(0,107.6){\line(1,0){60}}
\color{black}
\put(60,106.6){\small{\textemdash  2nd (chromatic)}}
\color{black}
\linethickness{1mm}
\put(0,41.3){\line(1,0){60}}
\color{black}
\put(60,40.3){\small{\textemdash  4th (chromatic)}}
\color{black}
\linethickness{1mm}
\put(0,5){\line(1,0){60}}
\color{black}
\put(60,4){\small{\textemdash  5th (diatonic)}}
\color{black}
\linethickness{1mm}
\put(0,151){\line(1,0){60}}
\color{black}
\put(60,150){\small{\textemdash  1st (chromatic)}}
\color{black}
\linethickness{1mm}
\put(0,66.9){\line(1,0){60}}
\color{black}
\put(60,65.9){\small{\textemdash  3rd (diatonic)}}
\color{black}
\linethickness{1mm}
\put(0,181.6){\line(1,0){60}}
\color{black}
\put(60,180.6){\small{\textemdash  Nut}}
\color{black}
\put(2,23.1){\small{c}}
\put(12,23.1){\small{f}}
\put(22,23.1){\small{b$\flat$}}
\put(32,23.1){\small{d'}}
\put(42,23.1){\small{g'}}
\put(52,23.1){\small{c''}}
\color{black}
\put(2,129.3){\small{A}}
\put(12,129.3){\small{d}}
\put(22,129.3){\small{g}}
\put(32,129.3){\small{b}}
\put(42,129.3){\small{e'}}
\put(52,129.3){\small{a'}}
\color{black}
\put(2,87.2){\small{B$\flat$}}
\put(12,87.2){\small{e$\flat$}}
\put(22,87.2){\small{a$\flat$}}
\put(32,87.2){\small{c'}}
\put(42,87.2){\small{f'}}
\put(52,87.2){\small{b$\flat$'}}
\color{black}
\put(2,166.3){\small{G$\sharp$}}
\put(12,166.3){\small{c$\sharp$}}
\put(22,166.3){\small{f$\sharp$}}
\put(32,166.3){\small{a$\sharp$}}
\put(42,166.3){\small{d$\sharp$'}}
\put(52,166.3){\small{g$\sharp$'}}
\color{black}
\put(2,54.1){\small{B}}
\put(12,54.1){\small{e}}
\put(22,54.1){\small{a}}
\put(32,54.1){\small{c$\sharp$'}}
\put(42,54.1){\small{f$\sharp$'}}
\put(52,54.1){\small{b'}}
\end{picture}
\caption{Alternate quarter-comma fretting for the lute}
\label{fig:quarter-chromatic}
\end{figure}
 It allows for an F$\sharp$, C$\sharp$
and G$\sharp$ on the lower courses, which matches the C$\sharp$ and F$\sharp$ of the
fourth fret, but the D$\sharp$ and A$\sharp$ on the upper courses proves to be a problem.
A$\sharp$ is an especially unlikely chromatic note and it does not match the B$\flat$
found on the third fret of the sixth course.

There are two main reasons why the alternative solution of a chromatic first fret is
undesireable. First, it seems that most of the historical sources agree that the first
fret was some kind of diatonic semitone.  Refering back to table~\ref{table:comparison},
Dowland, Gerle, Gnasi and Bermudo all agreed that the first fret was diatonic in nature.
Although theirs was not a quarter-comma semitone but was closer to sixth-comma, it
indicates a preference for the diatonic semitone or flattened notes for the first fret as
opposed to sharpened notes.

The second reason than a chromatic first fret is unlikely is because it would create false
unisons between the B$\flat$ found on the sixth and the A$\sharp$ on the third courses, as
well as the E$\flat$ on the fifth and the D$\sharp$ on the second. These unisons are quite
common in lute tablature, and comprise some of the most common chord shapes used in
continuo playing.  A chromatic first fret would render those chords unplayable, making it
unlikely that a chromatic semitone would be used in either a solo or ensemble context.

\section{The Theorbo}

As lutes were used more and more in ensembles towards the end of the sixteenth century, a
new type of lute was invented specifically intended to play in ensembles.  The theorbo, as
it was called, was a much larger instrument and had additional bass strings that extended
beyond the length of the neck.  Although all kinds of lutes, including the theorbo, were
generally fretted the same way, the tuning of the theorbo presented other alternative
fretting solutions that were not available on the the lute.  The theorbo's nominal pitch
was almost always A, instead of the G as with other lutes of the time.  Technically, any
lute or theorbo can be tunned to any key, and it was not uncommon to find lutes pitched to
F, G, A and even D.  This applied to lutes in a consort where each existed in a variety of
sizes.  The pitch was often a result of however big the instrument was and what strings on
it would fit without breaking.

When taken in an ensemble context, the pitch of a lute or theorbo had to be standarized
somewhat so that it could play with other instruments.  In English consort music as well
as most all lute song publications in England, the lute was pitched to G.  The theorbo in
Italy on the other hand was usually pitched to A.  A lute in no matter what pitch, was
always tuned such that the preceding course was always lower than the one following it, so
the arrangement of pitches for a lute in G was:
\begin{figure}[h]
\centering
\includegraphics{examples/lute-tuning.pdf}
\caption{Standard lute tuning in G}
\end{figure}
The most common tuning of the theorbo at this time, however, differed from standard lute
tuning and used first and second courses that were an octave lower:
\begin{figure}[h]
\centering
\includegraphics{examples/theorbo-tuning.pdf}
\caption{Theorbo tuned in A with re-entrant first and second courses}
\end{figure}

Theorbos were designed for accompaniment and needed to provide more volume than other
lutes of the day; therefore, the body size was much larger and the strings were longer.
Because of the increased mensur length, it was not possible to preserve the low to high
arrangement of courses as they were on the lute. Players found that as they tried to tune
the upper strings to their normal lute pitches, the strings would break and it was not
possible to fashion a gut string thin enough to hold the pitch at that length.  To solve
the problem, they tuned the strings to the same pitch but at an octave lower and thus
preserving the same interval relationships between strings as they were on the lute  This
made chord shapes identical between instruments and only altered the voicings of the
chords. Since the theorbo was primarily a continuo instrument, the change in voicing did
not present a problem. In fact, it became more of an advantage.  The re-entrant tunning
kept the overall tessitura of the instrument lower and away from the accompanied singer or
instrumentalist.

All theorbos had eight additional bass strings that descended diatonically in pitch from
the A on the sixth course.  Therefore, the seventh, eighth and ninth courses would be G, F
and E, continuing on to an octave G on the fourteenth course.  The disposition of these
lower courses varied somewhat from instrument to instrument.  Praetorius disucssed two
kinds of theorbos, one which he called a Roman style theorbo that had six courses on the
fretboard and all eight of the lower courses on the extended neck of the
instrument.\textit{Citation needed!} The other type, which he called the Bolognese-style
theorbo, had eight courses on the fretboard and the rest of the bass courses were on the
extended neck.

Because of the variation Praetorius describes, players today will often have the seventh
as well as the eighth course on their fretboards before the additional strings on the
extended neck. See figure ~\ref{fig:theorbo-extended} below.
\begin{figure}[ht]
\centering
\setlength{\unitlength}{0.5mm}
\begin{picture}(80,191.6)
% Draw fingerboard edges
\color{black}
\linethickness{0.075mm}
\put(0,0){\line(0,1){186.6}}
\put(80,0){\line(0,1){186.6}}

% Draw strings
\color{strings}
\linethickness{0.5mm}
\put(5,0){\line(0,1){186.6}}
\put(15,0){\line(0,1){186.6}}
\put(25,0){\line(0,1){186.6}}
\put(35,0){\line(0,1){186.6}}
\put(45,0){\line(0,1){186.6}}
\put(55,0){\line(0,1){186.6}}
\put(65,0){\line(0,1){186.6}}
\put(75,0){\line(0,1){186.6}}

% Insert string pitch names for lute in A (theorbo)

\color{black}
\put(2,191.6){\small{F}}

\put(14,191.6){\small{G}}

\put(24,191.6){\small{A}}

\put(34,191.6){\small{d}}

\put(43,191.6){\small{g}}

\put(53,191.6){\small{b}}
\put(63,191.6){\small{e'}}
\put(73,191.6){\small{a'}}


\color{black}
\linethickness{1mm}
\put(0,136.2){\line(1,0){80}}
\color{black}
\put(80,135.2){\small{\textemdash  1st (diatonic)}}
\color{black}
\linethickness{1mm}
\put(0,107.6){\line(1,0){80}}
\color{black}
\put(80,106.6){\small{\textemdash  2nd (chromatic)}}
\color{black}
\linethickness{1mm}
\put(0,41.3){\line(1,0){80}}
\color{black}
\put(80,40.3){\small{\textemdash  4th (chromatic)}}
\color{black}
\linethickness{1mm}
\put(0,5){\line(1,0){80}}
\color{black}
\put(80,4){\small{\textemdash  5th (diatonic)}}
\color{black}
\linethickness{1mm}
\put(0,66.9){\line(1,0){80}}
\color{black}
\put(80,65.9){\small{\textemdash  3rd (diatonic)}}
\color{black}
\linethickness{1mm}
\put(0,181.6){\line(1,0){80}}
\color{black}
\put(80,180.6){\small{\textemdash  Nut}}
\color{black}
\put(2,158.9){\small{G$\flat$}}
\put(12,158.9){\small{A$\flat$}}
\put(22,158.9){\small{B$\flat$}}
\put(32,158.9){\small{e$\flat$}}
\put(42,158.9){\small{a$\flat$}}
\put(52,158.9){\small{c'}}
\put(62,158.9){\small{f'}}
\put(72,158.9){\small{b$\flat$'}}
\color{black}
\put(22,23.1){\small{d}}
\put(32,23.1){\small{g}}
\put(42,23.1){\small{c'}}
\put(52,23.1){\small{e'}}
\put(62,23.1){\small{a'}}
\put(72,23.1){\small{d''}}
\color{black}
\put(22,87.2){\small{c}}
\put(32,87.2){\small{f}}
\put(42,87.2){\small{b$\flat$}}
\put(52,87.2){\small{d'}}
\put(62,87.2){\small{g'}}
\put(72,87.2){\small{c''}}
\color{black}
\put(22,54.1){\small{c$\sharp$}}
\put(32,54.1){\small{f$\sharp$}}
\put(42,54.1){\small{b}}
\put(52,54.1){\small{d$\sharp$'}}
\put(62,54.1){\small{g$\sharp$'}}
\put(72,54.1){\small{c$\sharp$''}}
\color{black}
\put(22,121.9){\small{B}}
\put(32,121.9){\small{e}}
\put(42,121.9){\small{a}}
\put(52,121.9){\small{c$\sharp$'}}
\put(62,121.9){\small{f$\sharp$'}}
\put(72,121.9){\small{b}'}
\end{picture}
\caption{Theorbo with extend courses}
\label{fig:theorbo-extended}
\end{figure}

The advantage to having these additional courses on the neck is that a
player is able to fret additional chromatic notes with the left hand.  On the longer
strings that are attached to the extention, this is not possible and any chromatic changes
in the pitches of those stings must be done using the tuning pegs prior to playing.

While it might have been possible to tune a lute's first course to a chromatic semitone,
this was impossible on the theorbo. For example, the first fret had to be diatonic because
of the open E$\natural$ and B$\natural$ on the second and third courses so that the
pitches on those courses of the first fret would be F$\natural$ and C$\natural$.  If a
chromatic semitone was used, an E$\sharp$ and B$\sharp$ would result, making this type of
semitone unusable. Similarly, the presence of a B$\flat$ at the third fret of the fourth
course dictates that the next fret must be chromatic to create a B$\natural$ at the fourth
fret of the same course. Even the sixth fret is determined to be diatonic because of the
E$\natural$ to F$\natural$ that occurs on the third course.

Essentially, the frets of a theorbo tunned to meantone temperament were ``fixed'' in their
positions because the location of the diatonic semitones between B and C, and E and F
determined which fret was either chromatic or diatonic.  If we also take into
consideration the same issue of octaves that affect the position of frets on the lute in
G, it makes it impossible to have the frets on a theorbo positioned in any other way that
would not make it unplayable in a meantone temperament.

\subsection{Solutions utilizing re-entrant tuning}

Because of the theorbo's unique re-entrant tuning, it has the potential advantage of using
both the chromatic and the diatonic semitones.  Notes that were normally an octave
apart on the lute were unisons on the theorbo.  This offered a possible solution to some
of the problems of semitone size because a particular pitch could be a diatonic semitone
in one location and a chromatic one in another and still be in the same octave. Referring
to figure~\ref{fig:theorbo-extended}, the A$\flat$ on the first fret of the fourth course
and the G$\sharp$ found on the fourth fret of the second course are in the same octave,
whereas on a lute in standard G tuning they are an octave apart. Similarly, the E$\flat$
on the fifth course of the first fret also has its chromatic counterpart D$\sharp$ on the
third course of the fourth fret.  All the player has to do is choose the appropriate
fingering for the left hand.

For chords that require chromatic semitones, such as a G$\sharp$ or a D$\sharp$, the
player can use pitches found on the fourth fret.  Some of the more common left-hand chord
patterns that use this fret include the E major triad and different kinds of chords with a
sixth above the bass.
\begin{figure}[h]
\centering
\includegraphics{examples/g-sharp.pdf}
\caption{Chords using chromatic semitones on the fourth fret}
\label{fourth-fret-chords}
\end{figure}
For other chords that require diatonic semitones, such as the A$\flat$ or E$\flat$, the
player may use pitches on the first fret.  This includes A$\flat$ major, F minor and C
minor.
\begin{figure}[h]
\centering
\includegraphics{examples/a-flat.pdf}
\caption{Chords using diatonic semitones on the first fret}
\label{first-fret-chords}
\end{figure}
Although instances of an A$\flat$ major triad are rare, all the needed pitches are on the
first fret and the F minor and C minor triads are both possible using a limited number of voices.

While re-entrant tuning makes it possible to play chords with different kinds of
semitones, the left-hand chord fingerings that result are not idiomatic to the instrument.
More ideal chords for the theorbo are easier to execute, use as many strings as possible,
and favor open strings whenever possible.  The fingerings for F minor and C minor listed
in figure~\ref{first-fret-chords} are not common existing theorbo tablatures of the time, nor are
they used very often among modern players. The more common fingerings for these chords
make use of the fourth fret.  Additionally, the E major chord in figure~\ref{fourth-fret-chords}
makes use of the appropriate semitone but completely ignores several available open
strings.  More common chord fingerings below in figure~\ref{common-chords} indicate that a player
would more likely use a fully-voiced F minor or C minor chord with a barre at the third
fret than those listed previously.  An E major chord that makes use of the open B and
E strings would sound much more resonant and would be easier to play for the left-hand.
\begin{figure}[h]
\centering
\includegraphics{examples/common-chords.pdf}
\caption{Common theorbo chord shapes}
\label{common-chords}
\end{figure}
The obvious problem with these more idiomatic chord shapes is that if they are used on
a theorbo in meantone temperament, their semitones are opposite of what they should be.
The E major chord shown above would have a diatonic A$\flat$ instead of the chromatic
G$\sharp$ and the thirds of the F minor and C minor chords would be chromatic in nature
instead of diatonic.

\subsection{Tastini}

If we consider some of the tuning problems specific to theorbos or other lutes with
extended bass courses,
the pitches of the chromatic notes on the seventh and eighth courses are determined by the
quality of the first fret.  Assuming that the first fret on a theorbo is a diatonic
semitone, this would make the pitches on this fret for these two lower courses A$\flat$
and G$\flat$.  It is far more likely that a G$\sharp$ and an F$\sharp$ are needed, but
shifting the entire first fret to a chromatic semitone would alter the rest of the notes
and result in a B$\sharp$ on the third course instead of a C.

To correct this problem and apply a chromatic semitone localized only to one or two
courses, lute players during this time employed the use of \textit{tastini}.  The
diminutive form of \textit{tasto}, the Italian word for fret, these ``little frets'' were
small pieces of wood that were glued on to the fretboard to create a chromatic semtione on
one or two courses while the remainder of the courses on the fret were diatonic.  Courses
beyond the sixth on a theorbo were used for bass support and any that were on the
fretboard, such as the seventh and eighth course, were only stopped at the first fret.
This made tastini an ideal choice since it only affected the first fret. Players now had
the ability to use an F$\sharp$ and G$\sharp$ while keeping the rest of the pitches on the
first fret at their original diatonic position. See figure ~\ref{fig:theorbo-tastini}.

While there are no surviving instruments with their tastini intact, we do know that they
were used because of different historical accounts that describe them.  The earliest
account comes from Vicenzo Galilei's \textit{Fronimo}, where he did not have very good
things to say about them.  Galilei's description of tastini suggests that players were
using them in different places on the instrument and not just the first fret.  Galilei was
writing at a time shortly before the theorbo came into use, so he described their use on
the lute, such as a tastino on the just beneath the second fret of the fifth course, in
order to make thirds less sharp.[Fronino 165]  Although in today's uses, most players do
not use a tastino in this location.

Galilei's main disagreement over the use of tastini was that it made adjustments to one
fret only in a certain pitch context, for example when you might want an F$\sharp$ instead
of a G$\flat$, but that one adjustment does not work in other pitch contexts or match the
same pitch of the instrument in a different location on the fingerboard.  He also
maintained that the lute was tuned in equal semitones and a well-placed fretting system
was sufficient to play all the pitches necessary. To him, tasini ruined that sort of
system, which is why he disagreed with them so vehemently.

If we return to our example usage of tastini on the theorbo, the solution that a tastino
presents is localized to only one note.  In other words, we would not plan on ever needing
a G$\flat$ on the seventh course so it seems entirely appropriate to fix that one fret to
be F$\sharp$ instead.  Galilei was arguing against tasini because depending on the
context, that fret might need to be either sharp or flat.  So a tastino offers no better
solution but only shifts the problem from one semitone to another.  In his mind, if the
lute was tuned in equal semitones, there would be no need for a tastino because the fret
would function correctly as either a chromatic or diatonic semitone.  In the case of the
theorbo, we only need it to be chromatic and are happy to completely avoid its diatonic
usage at the particular course and fret.

No matter if with to follow Galilei's advice or not, his attitude towards tastini is the
most important indiciator that tastini were in use.  Obviously, some players must have
been using them at this time otherwise, Galilei would not have mentioned it.  However, we
must note that the manner in which Galilei describes their usage has no direct method of
application in the present situation of quarter-comma meantone.  We would use tasitini on
our theorbo in a way Galilei had not envisioned.  This is probably because he was writing
at time shortly before the theorbo had come into use.

A later sources on tastini that appeared after the theorbo came into use come from Jean
Denis who was a harpsichord builder during the first half of the seventeenth century. He
refers to ``staggered'' frets on the lute made of ivory which could be the same as the
tastini to which Galilei was also refering. [ref to lindey, p. 47]  Another more imporant
reference comes from Chritopher Simpson's \textit{Compendium} who comes the closest to
refering to tastini as they are commonly used today on the theorbo.

% Other sources to look at:
%  Simpson "Compendium"
% see http://www.opensubscriber.com/message/lute@cs.dartmouth.edu/9412405.html
% MT40.S6 1970  at Case Music Library

\begin{figure}[ht]
\centering
\setlength{\unitlength}{0.5mm}
\begin{picture}(80,191.6)
% Draw fingerboard edges
\color{black}
\linethickness{0.075mm}
\put(0,0){\line(0,1){186.6}}
\put(80,0){\line(0,1){186.6}}

% Draw strings
\color{strings}
\linethickness{0.5mm}
\put(5,0){\line(0,1){186.6}}
\put(15,0){\line(0,1){186.6}}
\put(25,0){\line(0,1){186.6}}
\put(35,0){\line(0,1){186.6}}
\put(45,0){\line(0,1){186.6}}
\put(55,0){\line(0,1){186.6}}
\put(65,0){\line(0,1){186.6}}
\put(75,0){\line(0,1){186.6}}

% Insert string pitch names for lute in A (theorbo)

\color{black}
\put(2,191.6){\small{F}}

\put(14,191.6){\small{G}}

\put(24,191.6){\small{A}}

\put(34,191.6){\small{d}}

\put(43,191.6){\small{g}}

\put(53,191.6){\small{b}}
\put(63,191.6){\small{e}}
\put(73,191.6){\small{a}}


\color{black}
\linethickness{1mm}
\put(0,136.2){\line(1,0){80}}
\put(2,151){\line(1,0){17}}
\color{black}
\put(80,135.2){\small{\textemdash  1st (diatonic)}}
\color{black}
\linethickness{1mm}
\put(0,107.6){\line(1,0){80}}
\color{black}
\put(80,106.6){\small{\textemdash  2nd (chromatic)}}
\color{black}
\linethickness{1mm}
\put(0,41.3){\line(1,0){80}}
\color{black}
\put(80,40.3){\small{\textemdash  4th (chromatic)}}
\color{black}
\linethickness{1mm}
\put(0,5){\line(1,0){80}}
\color{black}
\put(80,4){\small{\textemdash  5th (diatonic)}}
\color{black}
\linethickness{1mm}
\put(0,66.9){\line(1,0){80}}
\color{black}
\put(80,65.9){\small{\textemdash  3rd (diatonic)}}
\color{black}
\linethickness{1mm}
\put(0,181.6){\line(1,0){80}}
\color{black}
\put(80,180.6){\small{\textemdash  Nut}}
\color{black}
\put(2,158.9){\small{F$\sharp$}}
\put(12,158.9){\small{G$\sharp$}}
\put(22,158.9){\small{B$\flat$}}
\put(32,158.9){\small{E$\flat$}}
\put(42,158.9){\small{A$\flat$}}
\put(52,158.9){\small{C}}
\put(62,158.9){\small{F}}
\put(72,158.9){\small{B$\flat$}}
\color{black}
% Tastini pitches
\put(2,140){\small{G$\flat$}}
\put(12,140){\small{A$\flat$}}
\put(22,23.1){\small{D}}
\put(32,23.1){\small{G}}
\put(42,23.1){\small{C}}
\put(52,23.1){\small{E}}
\put(62,23.1){\small{A}}
\put(72,23.1){\small{D}}
\color{black}
\put(22,87.2){\small{C}}
\put(32,87.2){\small{F}}
\put(42,87.2){\small{B$\flat$}}
\put(52,87.2){\small{D}}
\put(62,87.2){\small{G}}
\put(72,87.2){\small{C}}
\color{black}
\put(22,54.1){\small{C$\sharp$}}
\put(32,54.1){\small{F$\sharp$}}
\put(42,54.1){\small{B}}
\put(52,54.1){\small{D$\sharp$}}
\put(62,54.1){\small{G$\sharp$}}
\put(72,54.1){\small{C$\sharp$}}
\color{black}
\put(22,121.9){\small{B}}
\put(32,121.9){\small{E}}
\put(42,121.9){\small{A}}
\put(52,121.9){\small{C$\sharp$}}
\put(62,121.9){\small{F$\sharp$}}
\put(72,121.9){\small{B}}
\end{picture}
\caption{Theorbo with added tastini}
\label{fig:theorbo-tastini}
\end{figure}


Other solutions to the problem of chromatic and diatonic frets involved attempting to
place frets at an angle so that, presumeably, a fret could be diatonic on one side of the
fingerboard and chromatic on the other.  For the theorbo, this has a possible application
at the sixth fret, which should be a diatonic semitone due to the E at the third course on
the preceeding fret. Because of the diatonic semintone between E and F, all the pitches at
the sixth fret are diatonic creating E$\flat$ and A$\flat$ on the the sixth and fifth
courses.  Unless a tastino is used on the seventh string, the only other location for the
G$\sharp$ is at the second course which has limitations.  For example, a first inversion E
major chord, or in continuo terms, a G$\sharp$ with the figure ``6'' beneath it, is not
possible using the G$\sharp$ found at the second course.

The most common chord shape for chords of the sixth on the theorbo is to play the bass on
the fifth course or lower.  F$\sharp$ is available on this course but not a G$\sharp$.  In
order to have a true G$\sharp$, we could employ a tastino at the seventh fret and create a
G$\sharp$ an octave below, or the other alternative is to move one side of the sixth fret
back towards the nut, creating a chromatic semitone near the fifth and sixth frets, but
still leaving the fret in its diatonic position at the opposite end.

The problem with this is that pitches in the middle of the fret are somewhere between
chromatic and diatonic.  Juan Bermudo discusses the practice of angled frets on the
vihuela and comes to the same conclusions:
\begin{blocks}
... some player hope to fix the abovementioned faults by putting the frets where the said
faults occur at an angle, taking them out of line. This is not a solution but a cover-up
[...] Take a fret where there is a fault (where it is \textit{mi} for strings but needs to
be \textit{fa} for others) and you will find that, by slanting the fret, it does not hit
any string in the right place. (pages 112-113)
\end{blocks}


% Solo context and lute-song/tablature accompaniment suggests equal semitones



