\chapter{Modern Lute Fretting}

If we recall the excerpt from Hercole Bottrigari's \textit{Il Desiderio} at the beginning of this paper, he demonstrated
that not all the pitches of a lute match those of a harpsichord. This was because the standard tuning in the sixteenth
and early seventeenth centuries was meantone temperament, a pr`erred temperament among instruments due to its
harmonious, pure thirds. In the previous chapter, we found that none of the major sources on lute fretting specified a
uniformly meantone fretting system, quarter-comma or otherwise. Today, lute players who wish to participate in ensembles
that are employing meantone temperaments are going to find themselves in the same situation in which Desiderio found
himself, trying to match the pitches on his lute with those of the harpsichord.

It seems obvious that if ensembles were using a meantone temperament, then lutes must have as well, but it is less
obvious how they accomplished this. Because of the arrangement of semitones on a lute, meantone temperaments must be
executed differently than on other instruments. Bottrigari and others recognized these problems and proposed practical
solutions for dealing with them by ultimately deciding that the lute was an equal semitone instrument. Others observed
that it was unique in its temperament and music sounding bad on one instrument seemed to sound better on a
lute.\autocite[45]{ML:1} However, the fact remains that, in large and small ensembles, lutes were combined with other
instruments which very likely used quarter-comma and other varieties of meantone temperament. So it stands to reason
that players had found ways to reconcile the problem.

Since different historical schemes set forth in the previous chapter fall short of a comprehensive solution to meantone
fretting, we are left to find our own solutions. In this chapter, I will address the issue of meantone temperaments in
ensembles, specifically quarter-comma meantone because it is the most common temperament associated with music of this
period and is also one of the more problematic temperaments to realize on the lute. I will examine it in three different
contexts: 1) a continuo ensemble with theorbo; 2) a continuo ensemble with lute; and 3) ensembles with lute or theorbo
in tablature notation.

The solutions proposed here will build on ideas we have taken from historical sources, but will also incorporate
alternative ones, some of which are modern and contradict the approach of historical treatises. They include the use of
\textit{tastini} or ``little frets'' made of wood or ivory that only span one string and can be glued to the fret board.
These small frets make it possible to have a different kind of semitone on one string, such as a chromatic semitone,
while the rest of the strings all have the diatonic semitone. Another alternative involves the use of frets placed at
an angle so that they are not parallel to the nut or bridge. This results in a fret that could be a diatonic semitone
at one end and a chromatic one at the other. The arguments found in historical sources against such solutions support
the notion that the practice was commonly used, even if these sources sought to rally against it.

Any attempt to use meantone temperament on the lute is a kind of compromise. The least drastic solutions include
alternative fingerings for certain notes and varying the finger pressure of the left hand to alter the pitch.
Adjustments such as these could make pitches more agreeable in a tempered setting and enable the instruments to function
in a given meantone context. A lutenist could have also omitted offending notes from a realization, or remained
completely tacet if the temperament did not agree with the rest of the ensemble. There are many different combinations
of techniques that can enable a lute to play in a quarter-comma meantone temperament, but as we shall see, each
combination has a different effect on the instrument's capabilities.

\section{The Lute in Ensembles}

If we overlook the problems with historical fretting instructions, it is possible to realize quarter-comma meantone
temperament on a lute or theorbo, but it has some consequences. When a lute is put into a meantone temperament, its
compass may be limited and some of its idiomatic qualities may suffer. For example, common left-hand chord shapes that
provide the most resonance may not be possible. These issues can be mitigated if the instrument is used for basso
continuo, where the choice of chord shape is at the player's discretion. Since the player has more flexibility and
control over the number of voices in the chord and its location on the fretboard, it is easier to overcome some of
meantone's limitations in an ensemble setting. In solo situations, where the exact location of each pitch is dictated
by the tablature, a meantone temperament could produce undesirable results and necessitate the use of a different
temperament. It is important to note that there is no apparent evidence for quarter-comma fretting in any solo
literature from the seventeenth century onwards. Therefore, it is possible that for later solo repertoire lutenists
employed less problematic meantone temperaments such as sixth-comma, or utilized the quasi-equal and custom temperaments
that were featured in the previous chapter. This leaves issues with quarter-comma meantone mostly relegated to ensemble
literature throughout the period of this study, and to a lesser extent, solo literature in the sixteenth century.

Meantone fretting on a lute imposes a uniform semitone type for each fret. The main feature of meantone temperaments are
thirds that are much closer to pure than Pythagorean tuning or equal temperament. The two consequences of this are
narrow fifths, or fifths that are flatter than pure, and unequal semitones, namely the diatonic and chromatic semitone.
For the keyboard, this means choosing between an F$\sharp$ and a G$\flat$, or a D$\sharp$ and an E$\flat$. In any
meantone temperament, these are two different notes, rather than enharmonic equivalents. When tuning a keyboard, the
choice between a chromatic or a diatonic semitone can be independent of any other note on the instrument. It is possible
to have an F$\sharp$, A$\flat$ and C$\sharp$ all on the same octave. On the lute, however, this is not possible because
fret placement dictates the size of the semitone for all notes along that fret. For example, the choice of a chromatic
semitone on one course to yield a C$\sharp$ might force another course to have G$\sharp$ instead of A$\flat$.

In temperaments with unequal semitones, lutenists must choose between a chromatic or a diatonic semitone when placing
frets on their instrument. For example, if we were fretting our instrument in quarter-comma meantone, the first fret
would either be a chromatic semitone or a diatonic one. The choice of semitone is up to the player, but the main
consideration the player should take into account is whether any pitches on that fret must be chromatic or diatonic.
For a lute in standard G tuning, the second course is tuned to D; therefore, the first fret of the D course could either
be E$\flat$ or D$\sharp$, the second fret of the course E and the third is F. The distance between E and F is a
diatonic semitone, which makes the distance between the second and third fret diatonic. Because of this one requirement
in tuning between two notes on a single course, the distance of a diatonic semitone between these two frets will apply
to all other courses as well. Given these requirements, Dowland's chromatic sixth-comma fret in
table~\ref{table:comparison} seems impossible, because this would create something flatter than F. Gerle's diatonic
fret, on the other hand, makes more sense in this context.

When choosing semitone size, players must also consider that some chromatic notes are more common than others. It is
more common to find an F$\sharp$ in the seventeenth century than it is a G$\flat$. Returning to our previous example of
the D course, the fourth fret makes better sense as a chromatic semitone, giving us an F$\sharp$, instead of a G$\flat$.
If we use this same logic and move back to the first fret of that course, we could choose an E$\flat$ over a D$\sharp$
by reasoning that we are probably more likely to encounter an E$\flat$ than a D$\sharp$, although this is not always the
case.

For the sake of argument, let us say that we have settled on the semitone choices we have
discussed above: diatonic first and third frets, followed by a chromatic forth fret. The
result of these semitone sizes are summarized in figure~\ref{fig:quarter-diatonic}, and as
we can see, this has an interesting impact on the pitches on the other courses
of the same frets.
\begin{figure}[ht]
\centering
\setlength{\unitlength}{0.5mm}
\begin{picture}(60,191.6)
% Draw fingerboard edges
\color{black}
\linethickness{0.075mm}
\put(0,0){\line(0,1){186.6}}
\put(60,0){\line(0,1){186.6}}

% Draw strings
% 6th course
\color{strings}
\linethickness{0.5mm}
\put(5,0){\line(0,1){186.6}}
\linethickness{0.25mm}
\put(7,0){\line(0,1){186.6}}
% 5th course
\put(15,0){\line(0,1){186.6}}
\put(17,0){\line(0,1){186.6}}
% 4th course
\put(25,0){\line(0,1){186.6}}
\put(27,0){\line(0,1){186.6}}
% 3rd course
\put(35,0){\line(0,1){186.6}}
\put(37,0){\line(0,1){186.6}}
% 2nd course
\put(45,0){\line(0,1){186.6}}
\put(47,0){\line(0,1){186.6}}
% 1st course
\put(56,0){\line(0,1){186.6}}
% Insert string pitch names for lute in G
% 6th
\color{black}
\put(2,191.6){\small{G}}
% 5th
\put(14,191.6){\small{c}}
% 4th
\put(24,191.6){\small{f}}
% 3rd
\put(34,191.6){\small{a}}
% 2nd
\put(43,191.6){\small{d'}}
% 1st
\put(53,191.6){\small{g'}}
\color{black}
\linethickness{1mm}
\put(0,136.2){\line(1,0){60}}
\color{black}
\put(60,135.2){\small{\textemdash  1st (diatonic)}}
\color{black}
\linethickness{1mm}
\put(0,107.6){\line(1,0){60}}
\color{black}
\put(60,106.6){\small{\textemdash  2nd (chromatic)}}
\color{black}
\linethickness{1mm}
\put(0,41.3){\line(1,0){60}}
\color{black}
\put(60,40.3){\small{\textemdash  4th (chromatic)}}
\color{black}
\linethickness{1mm}
\put(0,5){\line(1,0){60}}
\color{black}
\put(60,4){\small{\textemdash  5th (diatonic)}}
\color{black}
\linethickness{1mm}
\put(0,66.9){\line(1,0){60}}
\color{black}
\put(60,65.9){\small{\textemdash  3rd (diatonic)}}
\color{black}
\linethickness{1mm}
\put(0,181.6){\line(1,0){60}}
\color{black}
\put(60,180.6){\small{\textemdash  Nut}}
\color{black}
\put(2,158.9){\small{A$\flat$}}
\put(12,158.9){\small{D$\flat$}}
\put(22,158.9){\small{G$\flat$}}
\put(32,158.9){\small{B$\flat$}}
\put(42,158.9){\small{E$\flat$}}
\put(52,158.9){\small{A$\flat$}}
\color{black}
\put(2,23.1){\small{C}}
\put(12,23.1){\small{F}}
\put(22,23.1){\small{B$\flat$}}
\put(32,23.1){\small{D}}
\put(42,23.1){\small{G}}
\put(52,23.1){\small{C}}
\color{black}
\put(2,87.2){\small{B$\flat$}}
\put(12,87.2){\small{E$\flat$}}
\put(22,87.2){\small{A$\flat$}}
\put(32,87.2){\small{C}}
\put(42,87.2){\small{F}}
\put(52,87.2){\small{B$\flat$}}
\color{black}
\put(2,54.1){\small{B}}
\put(12,54.1){\small{E}}
\put(22,54.1){\small{A}}
\put(32,54.1){\small{C$\sharp$}}
\put(42,54.1){\small{F$\sharp$}}
\put(52,54.1){\small{B}}
\color{black}
\put(2,121.9){\small{A}}
\put(12,121.9){\small{D}}
\put(22,121.9){\small{G}}
\put(32,121.9){\small{B}}
\put(42,121.9){\small{E}}
\put(52,121.9){\small{A}}
\end{picture}
\caption{Standard Quarter-comma Fretting}
\label{fig:quarter-diatonic}
\end{figure}

Since we have chosen E$\flat$ over D$\sharp$ for the first fret, this results in a
B$\flat$ for the third course, which is a good choice; however, the fourth and fifth
courses are G$\flat$ and D$\flat$, instead of the more likely F$\sharp$ and C$\sharp$.
This also creates a problem because the G$\flat$ found on the first fret of the fourth
course will not match the F$\sharp$ on the fourth fret of the second course. Similarly,
the D$\flat$ on the fifth course will not match the C$\sharp$ on the third.

The alternative solution is to have a chromatic semitone for the first fret instead of a
diatonic one, as seen in figure ~\ref{fig:quarter-chromatic}, but this only makes things
worse.
\begin{figure}[ht]
\centering
\setlength{\unitlength}{0.5mm}
\begin{picture}(60,191.6)
% Draw fingerboard edges
\color{black}
\linethickness{0.075mm}
\put(0,0){\line(0,1){186.6}}
\put(60,0){\line(0,1){186.6}}

% Draw strings
% 6th course
\color{strings}
\linethickness{0.5mm}
\put(5,0){\line(0,1){186.6}}
\linethickness{0.25mm}
\put(7,0){\line(0,1){186.6}}
% 5th course
\put(15,0){\line(0,1){186.6}}
\put(17,0){\line(0,1){186.6}}
% 4th course
\put(25,0){\line(0,1){186.6}}
\put(27,0){\line(0,1){186.6}}
% 3rd course
\put(35,0){\line(0,1){186.6}}
\put(37,0){\line(0,1){186.6}}
% 2nd course
\put(45,0){\line(0,1){186.6}}
\put(47,0){\line(0,1){186.6}}
% 1st course
\put(56,0){\line(0,1){186.6}}
% Insert string pitch names for lute in G
% 6th
\color{black}
\put(2,191.6){\small{G}}
% 5th
\put(14,191.6){\small{c}}
% 4th
\put(24,191.6){\small{f}}
% 3rd
\put(34,191.6){\small{a}}
% 2nd
\put(43,191.6){\small{d'}}
% 1st
\put(53,191.6){\small{g'}}
\color{black}
\linethickness{1mm}
\put(0,107.6){\line(1,0){60}}
\color{black}
\put(60,106.6){\small{\textemdash  2nd (chromatic)}}
\color{black}
\linethickness{1mm}
\put(0,41.3){\line(1,0){60}}
\color{black}
\put(60,40.3){\small{\textemdash  4th (chromatic)}}
\color{black}
\linethickness{1mm}
\put(0,5){\line(1,0){60}}
\color{black}
\put(60,4){\small{\textemdash  5th (diatonic)}}
\color{black}
\linethickness{1mm}
\put(0,151){\line(1,0){60}}
\color{black}
\put(60,150){\small{\textemdash  1st (chromatic)}}
\color{black}
\linethickness{1mm}
\put(0,66.9){\line(1,0){60}}
\color{black}
\put(60,65.9){\small{\textemdash  3rd (diatonic)}}
\color{black}
\linethickness{1mm}
\put(0,181.6){\line(1,0){60}}
\color{black}
\put(60,180.6){\small{\textemdash  Nut}}
\color{black}
\put(2,23.1){\small{c}}
\put(12,23.1){\small{f}}
\put(22,23.1){\small{b$\flat$}}
\put(32,23.1){\small{d'}}
\put(42,23.1){\small{g'}}
\put(52,23.1){\small{c''}}
\color{black}
\put(2,129.3){\small{A}}
\put(12,129.3){\small{d}}
\put(22,129.3){\small{g}}
\put(32,129.3){\small{b}}
\put(42,129.3){\small{e'}}
\put(52,129.3){\small{a'}}
\color{black}
\put(2,87.2){\small{B$\flat$}}
\put(12,87.2){\small{e$\flat$}}
\put(22,87.2){\small{a$\flat$}}
\put(32,87.2){\small{c'}}
\put(42,87.2){\small{f'}}
\put(52,87.2){\small{b$\flat$'}}
\color{black}
\put(2,166.3){\small{G$\sharp$}}
\put(12,166.3){\small{c$\sharp$}}
\put(22,166.3){\small{f$\sharp$}}
\put(32,166.3){\small{a$\sharp$}}
\put(42,166.3){\small{d$\sharp$'}}
\put(52,166.3){\small{g$\sharp$'}}
\color{black}
\put(2,54.1){\small{B}}
\put(12,54.1){\small{e}}
\put(22,54.1){\small{a}}
\put(32,54.1){\small{c$\sharp$'}}
\put(42,54.1){\small{f$\sharp$'}}
\put(52,54.1){\small{b'}}
\end{picture}
\caption{Alternate quarter-comma fretting for the lute}
\label{fig:quarter-chromatic}
\end{figure}

It allows for an F$\sharp$, C$\sharp$ and G$\sharp$ on the lower courses, matching the
C$\sharp$ and F$\sharp$ of the fourth fret, but the D$\sharp$ and A$\sharp$ on the
upper courses prove to be a problem. A$\sharp$ is an especially unlikely chromatic
note and does not match the B$\flat$ found on the third fret of the sixth course.

There are two critically important reasons why the alternative solution of a chromatic first fret is undesirable. First,
it seems that most of the historical sources agree that the first fret was some kind of diatonic semitone. Referring
back to table~\ref{table:comparison}, Dowland, Gerle, Ganassi, and Bermudo all agreed that the first fret was diatonic
in nature. Although their system was closer to sixth-comma than quarter, it indicates a preference for the diatonic
semitone, or notes that have the $\flat$ accidental as opposed to a $\sharp$.

The second reason is that a chromatic first fret would create false unisons between the B$\flat$ and the A$\sharp$, as
well as the E$\flat$ on the fifth and the D$\sharp$ on the second. These unisons are quite common in lute tablature, and
comprise some of the most common chord shapes used in continuo playing. (see example~\ref{first-fret-lute-chords}) A
chromatic first fret would render these chords unusable without additional adjustments, making it unlikely that a
chromatic semitone would be used in either a solo or ensemble context.

\begin{example}[h]
\centering
\includegraphics{examples/first-fret-lute-chords.pdf}
\caption{Common chords on the lute using the first fret}
\label{first-fret-lute-chords}
\end{example}

While these figures demonstrate that a lute can be placed in a meantone fretting,
it does not imply that it is a requirement. Composers such as Bottrigari seemed 
to think that the lute was simply an instrument that played in
equal semitones, even in ensembles with instruments that played un-equally.
In the discussion between his two fictional protagonists in \textit{Il Desiderio}, 
Benelli tells Desiderio:

\begin{blocks}
Therefore I do not wish either to affirm or deny, or even to dispute, whether or not 
the semitone said to be minor is minor, or if indeed it has a position between the 
greatest and the least; it will suffice to demonstrate by its effects -- i.e. the
Clavicembalo, the Organ, and their like, sound two unequal semitones, one larger 
than the other. The Lute and the Viols sound two equal semitones, that is, a tone
divided into two equal semitones according to the idea of Aristoxenus. \autocite[17]{HB:1}
\end{blocks}

Bottrigari was comfortable with this arrangement because he classified the instruments into different categories.
According to him, instruments fell into three different types: 1) stable; 2) stable but alterable; and, 3) completely
alterable. The determining factor for an instrument was its ability to alter pitch. Instruments in the first category
were stable because their pitches could not be altered. These included all keyboard instruments. The third category
included instruments whose pitches were entirely changeable, such as the trombone and violin. Their pitches existed
continuously, either along the slide of the trombone, or at any point on the fingerboard of a violin. In the middle,
where instruments are ``stable but alterable'', is where we find lutes, viols, recorders, and transverse flutes. In
Bottrigari's second category of instruments, pitches existed at fixed points but were variable to certain degree. Flute
players could vary pitch by changing the placement of their fingers, or controlling their breath. Lute players,
according to Bottrigari, could ``touch their frets a little higher or a little lower'' in order to vary their pitches.
\autocite[15]{HB:1}

The suggestion here is that lutes played in ensembles using a different temperament than the rest of the instruments.
Keyboard instruments would have used a meantone, or other kind of unequal temperament, and the lutes would have played
with these unequally-tuned instruments using their native quasi-equal temperament. In order for such an ensemble to be
successful, Bottrigari suggests that only the instruments of the first two categories, stable and stable but alterable,
or those from the first and third, stable and entirely alterable, be used together, and never all three together. The
problem with using all three, according to him, is that the middle group will not be able to match pitches with the
first and third groups at the same time. To this, he also adds: ``I think it best to add that no concert of instruments
should ever be given without the addition of a human voice.'' \autocite[23]{HB:1}

For Bottrigari, the issue was not with temperament, but with arrangement and orchestration. Certain instruments should
only play with one kind of instrument and not another. While this provides players some latitude when choosing
temperaments, it still begs the question as to how players could have adjusted their instruments to match other
instruments tuned in a meantone temperament. For that, we should examine a kind of lute that was specifically designed
for ensemble use.

\section{The Theorbo}

As lutes were used more and more in ensembles towards the end of the sixteenth century, a new type of lute was invented
specifically intended to be played in ensembles. The theorbo, as it was called, was a much larger instrument and had
additional bass strings that extended beyond the length of the neck. Although all kinds of lutes, including the theorbo,
were generally fretted in the same way, the tuning of the theorbo presented other alternative fretting solutions that
were not available on lutes using standard tuning. The theorbo's nominal pitch was almost always A, instead of the G as
with other lutes of the time. Technically, any lute or theorbo can be tuned to any key, and it was not uncommon to find
lutes pitched to F, G, A, and D. This applied to lutes in a consort where each instrument existed in a variety of sizes.
The pitch could also result from the instrument's mensur length. The top string would be tuned as high as possible
without breaking and that point determined the overall pitch of the instrument.

In an ensemble, the pitch of a lute or theorbo had to be standardized so that it could play with other instruments. In
English consort music as well as most lute song publications in England, the standard lute pitch was G. In Italy,
however, the theorbo was usually pitched to A. Regardless of the pitch of a lute, it was tuned so that the preceding
course was lower than the one following it. (see example~\ref{g-lute}) This was not the case for the theorbo, which had
first and second courses that were an octave lower. (see example~\ref{a-theorbo})

\begin{example}[h]
\centering
\includegraphics{examples/lute-tuning.pdf}
\caption{Standard lute tuning in G}
\label{g-lute}
\end{example}
\begin{example}[h]
\centering
\includegraphics{examples/theorbo-tuning.pdf}
\caption{Theorbo tuned in A with re-entrant first and second courses}
\label{a-theorbo}
\end{example}

Since theorbos were designed for accompaniment and needed to provide more volume than other lutes of the day, the body
size was much larger and the strings longer. Because of the increased mensur length, it was not possible to preserve the
low to high arrangement of courses as they were on the lute. Players found that as they tried to tune the upper strings
to their normal lute pitches, they would break and it was not possible to fashion a gut string thin enough to hold the
pitch at that length. To solve the problem, they tuned the strings to the same pitch but at an octave lower, thus
preserving the same intervalic relationships between strings as they were on the lute. This made chord shapes identical
between instruments and only altered the voicing of the chords. Since the theorbo was primarily a continuo instrument,
the change in voicing did not present a problem; in fact, it became more of an advantage. The re-entrant tuning kept the
overall tessitura of the instrument lower and away from that of the accompanied singer or instrumentalist.

All theorbos had eight additional bass strings that descended diatonically in
pitch from the A on the sixth course. Therefore, the seventh, eighth, and ninth
courses would be G, F, and E, continuing on to an octave G on the fourteenth
course. The disposition of these lower courses varied somewhat from instrument
to instrument. Praetorius discussed two kinds of theorbos: the first he called
a Roman style theorbo, which had six courses on the fretboard; and a second,
which he called the Paduan-style theorbo that had eight courses on the
fretboard. \autocite[59]{MP:1} Because of this variation, players today may
opt to have their seventh and eighth courses on their fretboards before the
additional strings on the extended neck. See figure ~\ref{fig:theorbo-extended}
below.
\begin{figure}[ht]
\centering
\setlength{\unitlength}{0.5mm}
\begin{picture}(80,191.6)
% Draw fingerboard edges
\color{black}
\linethickness{0.075mm}
\put(0,0){\line(0,1){186.6}}
\put(80,0){\line(0,1){186.6}}

% Draw strings
\color{strings}
\linethickness{0.5mm}
\put(5,0){\line(0,1){186.6}}
\put(15,0){\line(0,1){186.6}}
\put(25,0){\line(0,1){186.6}}
\put(35,0){\line(0,1){186.6}}
\put(45,0){\line(0,1){186.6}}
\put(55,0){\line(0,1){186.6}}
\put(65,0){\line(0,1){186.6}}
\put(75,0){\line(0,1){186.6}}

% Insert string pitch names for lute in A (theorbo)

\color{black}
\put(2,191.6){\small{F}}

\put(14,191.6){\small{G}}

\put(24,191.6){\small{A}}

\put(34,191.6){\small{d}}

\put(43,191.6){\small{g}}

\put(53,191.6){\small{b}}
\put(63,191.6){\small{e'}}
\put(73,191.6){\small{a'}}


\color{black}
\linethickness{1mm}
\put(0,136.2){\line(1,0){80}}
\color{black}
\put(80,135.2){\small{\textemdash  1st (diatonic)}}
\color{black}
\linethickness{1mm}
\put(0,107.6){\line(1,0){80}}
\color{black}
\put(80,106.6){\small{\textemdash  2nd (chromatic)}}
\color{black}
\linethickness{1mm}
\put(0,41.3){\line(1,0){80}}
\color{black}
\put(80,40.3){\small{\textemdash  4th (chromatic)}}
\color{black}
\linethickness{1mm}
\put(0,5){\line(1,0){80}}
\color{black}
\put(80,4){\small{\textemdash  5th (diatonic)}}
\color{black}
\linethickness{1mm}
\put(0,66.9){\line(1,0){80}}
\color{black}
\put(80,65.9){\small{\textemdash  3rd (diatonic)}}
\color{black}
\linethickness{1mm}
\put(0,181.6){\line(1,0){80}}
\color{black}
\put(80,180.6){\small{\textemdash  Nut}}
\color{black}
\put(2,158.9){\small{G$\flat$}}
\put(12,158.9){\small{A$\flat$}}
\put(22,158.9){\small{B$\flat$}}
\put(32,158.9){\small{e$\flat$}}
\put(42,158.9){\small{a$\flat$}}
\put(52,158.9){\small{c'}}
\put(62,158.9){\small{f'}}
\put(72,158.9){\small{b$\flat$'}}
\color{black}
\put(22,23.1){\small{d}}
\put(32,23.1){\small{g}}
\put(42,23.1){\small{c'}}
\put(52,23.1){\small{e'}}
\put(62,23.1){\small{a'}}
\put(72,23.1){\small{d''}}
\color{black}
\put(22,87.2){\small{c}}
\put(32,87.2){\small{f}}
\put(42,87.2){\small{b$\flat$}}
\put(52,87.2){\small{d'}}
\put(62,87.2){\small{g'}}
\put(72,87.2){\small{c''}}
\color{black}
\put(22,54.1){\small{c$\sharp$}}
\put(32,54.1){\small{f$\sharp$}}
\put(42,54.1){\small{b}}
\put(52,54.1){\small{d$\sharp$'}}
\put(62,54.1){\small{g$\sharp$'}}
\put(72,54.1){\small{c$\sharp$''}}
\color{black}
\put(22,121.9){\small{B}}
\put(32,121.9){\small{e}}
\put(42,121.9){\small{a}}
\put(52,121.9){\small{c$\sharp$'}}
\put(62,121.9){\small{f$\sharp$'}}
\put(72,121.9){\small{b}'}
\end{picture}
\caption{Theorbo with extend courses}
\label{fig:theorbo-extended}
\end{figure}

The advantage to having these additional courses on the neck is that a player is
able to fret additional chromatic notes with the left hand. On the longer
strings that are attached to the extension this is not possible and any
chromatic changes in the pitches of those strings must be done using the tuning
pegs prior to playing.

While it might have been possible to tune a lute's first course to a chromatic semitone, this was impossible on the
theorbo. For example, the first fret had to be diatonic because of the open E$\natural$ and B$\natural$ on the second
and third courses so that the pitches on those courses on the first fret would be F$\natural$ and C$\natural$. If a
chromatic semitone was used, an E$\sharp$ and B$\sharp$ would result, making this type of semitone unusable. Similarly,
the presence of a B$\flat$ at the third fret of the fourth course dictates that the next fret must be chromatic to
create a B$\natural$ at the fourth fret of the same course. Even the sixth fret is determined to be diatonic because of
the E$\natural$ to F$\natural$ that occurs on the third course. These restrictions could explain why Praetorius's
theorbo had all of its extended courses on the long neck, and off of the fretted short neck. This would enable the
player to set a fixed F$\sharp$ and G$\sharp$ on the seventh and eight courses and leave the pitches on the first fret
in their preferred diatonic semitones.

Essentially, the frets of a theorbo tuned to meantone temperament were ``fixed'' in their positions because the location
of the diatonic semitones between B and C, and E and F, determined whether the fret was either chromatic or diatonic. If
we also take into consideration the similar issue of octaves affecting the position of frets on the lute in G, it is
apparent that a theorbo should have its initial five frets arranged the same way as a lute, with a diatonic semitone for
the first fret and alternating diatonic and chromatic semitones for successive frets. If we was to blend successfully
with quarter-comma meantone, we will need to limit some of the semitones that are available on our instruments and find
alternative methods of playing them.

\subsection{Solutions Utilizing Re-entrant Tuning}

The theorbo has the advantage of using both chromatic and diatonic semitones within the same octave. Because of the re-
entrant nature of the theorbo's tuning, certain notes that are an octave apart on the lute are unisons on a theorbo.
This offers a possible solution to some of the problems of semitone size because a particular pitch could be a diatonic
semitone at one fret while being chromatic at a different fret, and still be in the same octave. Referring to
figure~\ref{fig:theorbo-extended}, the A$\flat$ on the first fret of the fourth course and the G$\sharp$ found on the
fourth fret of the second course are in the same octave, whereas on a lute in standard G tuning they are an octave
apart. Other notes are still an octave apart, such as the E$\flat$ on the fifth course of the first fret and the
D$\sharp$ on the third course of the fourth fret. However, in continuo playing octave displacement does not matter and
players are able to substitute different octaves as needed. Therefore, all the player has to do is choose the
appropriate fingering for the left hand to obtain either the chromatic or diatonic semitone.

For chords that require chromatic semitones, such as a G$\sharp$ or a D$\sharp$, the
player can use pitches found on the fourth fret. Some of the more common left-hand chord
patterns that use this fret include the E major triad and chords with a
sixth above the bass.
\begin{example}[h]
\centering
\includegraphics{examples/g-sharp.pdf}
\caption{Chords using chromatic semitones on the fourth fret}
\label{fourth-fret-chords}
\end{example}
For others requiring diatonic semitones, such as the A$\flat$ or E$\flat$, the
player may use pitches on the first fret. These include A$\flat$ major, F minor, and C
minor.
\begin{example}[h]
\centering
\includegraphics{examples/a-flat.pdf}
\caption{Chords using diatonic semitones on the first fret}
\label{first-fret-chords}
\end{example}
Although instances of an A$\flat$ major triad are rare, all the needed pitches are on
the first fret, and F minor and C minor triads are both possible using a limited
number of voices.

While re-entrant tuning makes it possible to play chords with different kinds of
semitones, sometimes the left-hand chord fingerings that result are not the easiest to
execute, nor are they as idiomatic to the instrument as other more commonly used
fingerings. More ideal chords for the theorbo are easier to execute, have more potential
voices, and favor open strings whenever possible. The fingerings for F
minor and C minor listed in example~\ref{first-fret-chords} are not commonly found in
existing theorbo tablatures of the time, nor are they used very often among modern
players. The more common fingerings for these chords use the fourth fret.
Additionally, the E major chord in example~\ref{fourth-fret-chords} uses the chromatic
semitone on the fourth fret, but ignores the open E and B on the third and second
courses. More common chord fingerings below in example~\ref{common-chords} indicate that
a player would more likely use a fully-voiced F minor or C minor chord with a barre at
the third fret than those listed previously. An E major chord that makes use of the
open B and E strings sounds much more resonant and is easier for the left hand.
\begin{example}[h]
\centering
\includegraphics{examples/common-chords.pdf}
\caption{Common theorbo chord shapes}
\label{common-chords}
\end{example}
The obvious problem with these more idiomatic chord shapes is that if they are used on
a theorbo in meantone temperament, their semitones are the opposite of what they should be.
The E major chord shown above would have a diatonic A$\flat$ instead of the chromatic
G$\sharp$ and the thirds of the F minor and C minor chords would be chromatic in nature
instead of diatonic.

The question raised here is: how did players manage in a meantone temperament
without the use of their common chord patterns? In a small ensemble, without
keyboards or other instruments with meantone requirements, we might have the
option of using a non-meantone temperament; however, if this is not desirable,
we should look at other available modifications to make meantone more successful
on our instruments. Selective left-hand fingerings alone are not enough to overcome
the problems imposed by meantone temperaments.

\subsection{Tastini}

The advantages that re-entrant tuning offers can render more idiomatic chord shapes unusable in a meantone temperament.
If we are to use these shapes, yet still be able to play in meantone, we need the ability to apply different semitones
locally within a fret instead of being forced to have all the pitches at one fret of a certain size. In other words, we
need to mix both chromatic and diatonic semitone sizes within the same fret. For example, consider the pitches of the
extended bass courses: the first chromatic note on the seventh and eighth courses is determined by the quality of the
first fret. Since the first fret on a theorbo is a diatonic semitone, this would make the pitches on this fret for these
two lower courses A$\flat$ and G$\flat$, respectively. It is far more likely that a G$\sharp$ and an F$\sharp$ are
needed, but shifting the entire first fret to a chromatic semitone would alter the rest of the notes and result in a
B$\sharp$ on the third course instead of a C.

To correct this problem and apply a chromatic semitone localized only to one or two courses, lute players during this
time employed the use of \textit{tastini}. The diminutive form of \textit{tasto}, the Italian word for fret, these
``little frets'' were small pieces of wood that were glued to the fretboard to create a chromatic semitone on one or two
courses while the remainder of the courses on the fret were diatonic. Courses beyond the sixth on a theorbo were used
for bass support, and any that were on the fretboard, such as the seventh and eighth course, were only stopped at the
first fret. This made the use of tastini an ideal choice since it only affected the first fret. Players now had the
ability to use an F$\sharp$ and G$\sharp$ while keeping the rest of the pitches on the first fret at their original
diatonic position. See figure~\ref{fig:theorbo-tastini}.

\begin{figure}[ht]
\centering
\setlength{\unitlength}{0.5mm}
\begin{picture}(80,191.6)
% Draw fingerboard edges
\color{black}
\linethickness{0.075mm}
\put(0,0){\line(0,1){186.6}}
\put(80,0){\line(0,1){186.6}}

% Draw strings
\color{strings}
\linethickness{0.5mm}
\put(5,0){\line(0,1){186.6}}
\put(15,0){\line(0,1){186.6}}
\put(25,0){\line(0,1){186.6}}
\put(35,0){\line(0,1){186.6}}
\put(45,0){\line(0,1){186.6}}
\put(55,0){\line(0,1){186.6}}
\put(65,0){\line(0,1){186.6}}
\put(75,0){\line(0,1){186.6}}

% Insert string pitch names for lute in A (theorbo)

\color{black}
\put(2,191.6){\small{F}}

\put(14,191.6){\small{G}}

\put(24,191.6){\small{A}}

\put(34,191.6){\small{d}}

\put(43,191.6){\small{g}}

\put(53,191.6){\small{b}}
\put(63,191.6){\small{e}}
\put(73,191.6){\small{a}}


\color{black}
\linethickness{1mm}
\put(0,136.2){\line(1,0){80}}
\put(2,151){\line(1,0){17}}
\color{black}
\put(80,135.2){\small{\textemdash  1st (diatonic)}}
\color{black}
\linethickness{1mm}
\put(0,107.6){\line(1,0){80}}
\color{black}
\put(80,106.6){\small{\textemdash  2nd (chromatic)}}
\color{black}
\linethickness{1mm}
\put(0,41.3){\line(1,0){80}}
\color{black}
\put(80,40.3){\small{\textemdash  4th (chromatic)}}
\color{black}
\linethickness{1mm}
\put(0,5){\line(1,0){80}}
\color{black}
\put(80,4){\small{\textemdash  5th (diatonic)}}
\color{black}
\linethickness{1mm}
\put(0,66.9){\line(1,0){80}}
\color{black}
\put(80,65.9){\small{\textemdash  3rd (diatonic)}}
\color{black}
\linethickness{1mm}
\put(0,181.6){\line(1,0){80}}
\color{black}
\put(80,180.6){\small{\textemdash  Nut}}
\color{black}
\put(2,158.9){\small{F$\sharp$}}
\put(12,158.9){\small{G$\sharp$}}
\put(22,158.9){\small{B$\flat$}}
\put(32,158.9){\small{E$\flat$}}
\put(42,158.9){\small{A$\flat$}}
\put(52,158.9){\small{C}}
\put(62,158.9){\small{F}}
\put(72,158.9){\small{B$\flat$}}
\color{black}
% Tastini pitches
\put(2,140){\small{G$\flat$}}
\put(12,140){\small{A$\flat$}}
\put(22,23.1){\small{D}}
\put(32,23.1){\small{G}}
\put(42,23.1){\small{C}}
\put(52,23.1){\small{E}}
\put(62,23.1){\small{A}}
\put(72,23.1){\small{D}}
\color{black}
\put(22,87.2){\small{C}}
\put(32,87.2){\small{F}}
\put(42,87.2){\small{B$\flat$}}
\put(52,87.2){\small{D}}
\put(62,87.2){\small{G}}
\put(72,87.2){\small{C}}
\color{black}
\put(22,54.1){\small{C$\sharp$}}
\put(32,54.1){\small{F$\sharp$}}
\put(42,54.1){\small{B}}
\put(52,54.1){\small{D$\sharp$}}
\put(62,54.1){\small{G$\sharp$}}
\put(72,54.1){\small{C$\sharp$}}
\color{black}
\put(22,121.9){\small{B}}
\put(32,121.9){\small{E}}
\put(42,121.9){\small{A}}
\put(52,121.9){\small{C$\sharp$}}
\put(62,121.9){\small{F$\sharp$}}
\put(72,121.9){\small{B}}
\end{picture}
\caption{Theorbo with added tastini}
\label{fig:theorbo-tastini}
\end{figure}


Players today have employed tastini on other frets as well, which can help solve the previous problems of idiomatic
chord shapes in meantone frettings. Referring again to figure~\ref{fig:theorbo-tastini}, an additional tastino on the
fourth course can provide us with a G$\sharp$, which enables us to play the more common E major chord shape described in
example~\ref{common-chords} using the open strings on courses two and three. The problem of C minor and F minor chords,
however, still remains. We could switch the entire fourth fret to a diatonic semitone, thereby giving us the needed
pitches, but we would lose the ability to play some of the sixth chords described in example~\ref{fourth-fret-chords}.
One could argue that additional tastini at the fourth fret could correct this problem, but such a solution might become
unwieldy. Also, it means sacrificing several chromatic semitones, in this case C$\sharp$, F$\sharp$ and G$\sharp$, for
the sake of two diatonic ones: E$\flat$ and A$\flat$.

While modern players have embraced the use of tastini, there are no surviving instruments with their tastini intact.
Yet, it is obvious they were in use because of the different historical accounts that describe them. The earliest of
these comes from Vicenzo Galilei's \textit{Fronimo}, where he did not have very good things to say about them. Galilei's
description of tastini suggests that players were using them in different places on the instrument and not just the
first fret.\autocite[165]{VG:1} Today, tastini are found usually only on theorbos, almost always at the first fret, and
rarely elsewhere. There are exceptions to this, and players today are free to put tastini anywhere, but the most common
location seems to be at the first fret.

Galilei's main disagreement over the use of tastini was that it made adjustments to one fret only in a certain pitch
context, for example when F$\sharp$ is wanted instead of a G$\flat$, but that one adjustment does not work in other
pitch contexts or match the same pitch at a different location on the fingerboard. He also maintained that the lute was
tuned in equal semitones and that a well-placed fretting system was sufficient to play all the pitches necessary. In his
mind, tastini ruined that sort of system because if the lute were tuned in equal semitones, there would be no need for a
tastino: the fret would function correctly as either a chromatic or diatonic semitone.

Whether or not we follow Galilei's advice, his attitude towards tastini is the most important indicator that they were
in use. Some players must have used them at this time, otherwise Galilei would not have mentioned it. However, we must
note that the manner in which Galilei describes their usage appears to have no direct application to correcting the
problems found in quarter-comma meantone. After all, since Galilei tuned equally, meantone was not an issue for him. All
of this points to the fact that if we are to use tastini on our theorbo or lute, it must be in a way Galilei had not
envisioned, and so we are left to create our own solutions.

Bermudo has a brief account of tastini in the context of the vihuela. His description
of their use fits very closely with our current usage of them:
\begin{blocks}
For faults that may arise, take the advice given before of looking for the notes on other
frets, or with the pressure of the finger when stopping the note, of by placing another
fret in front of the principle fret, which, when placed for this purpose, should be
thicker than the first fret so that it does not rub against the string. This [extra]
fret can be placed by dividing the distance from the third fret to the bridge into eight
parts, and wherever the compass reaches [downward from the third fret] will be the first
fret, which will form \textit{fa}. \autocite[115-116]{DE:1}
\end{blocks}
Recalling table~\ref{table:comparison}, Bermudo's first fret is a diatonic semitone or what
he calls \textit{fa}. In order to obtain the chromatic fret, or \text{mi}, Bermudo
proposes an additional fret that is front of the first fret, or placed between the nut
and the first fret. He says the fret should be slightly thicker, which we can infer
is so that the \textit{fa} fret does not prevent it from working properly.
\begin{figure}[ht]
\centering
\setlength{\unitlength}{1mm}
\begin{picture}(60,87.8)
% Draw fingerboard edges
\color{black}
\linethickness{0.075mm}
\put(0,0){\line(0,1){87.8}}
\put(60,0){\line(0,1){87.8}}

% Draw strings
% 6th course
\color{strings}
\linethickness{0.5mm}
\put(5,0){\line(0,1){87.8}}
\linethickness{0.25mm}
\put(7,0){\line(0,1){87.8}}
% 5th course
\put(15,0){\line(0,1){87.8}}
\put(17,0){\line(0,1){87.8}}
% 4th course
\put(25,0){\line(0,1){87.8}}
\put(27,0){\line(0,1){87.8}}
% 3rd course
\put(35,0){\line(0,1){87.8}}
\put(37,0){\line(0,1){87.8}}
% 2nd course
\put(45,0){\line(0,1){87.8}}
\put(47,0){\line(0,1){87.8}}
% 1st course
\put(56,0){\line(0,1){87.8}}
\color{black}
\linethickness{1mm}
\put(0,38.3){\line(1,0){60}}
\color{black}
\put(60,37.3){\small{\textemdash  fa}}
\color{black}
\linethickness{1mm}
\put(0,48.4){\line(1,0){60}}
\color{black}
\put(60,47.4){\small{\textemdash mi}}
\color{black}
\linethickness{1mm}
\put(0,5){\line(1,0){60}}
\color{black}
\put(60,4){\small{\textemdash 2nd fret}}
\color{black}
\linethickness{1mm}
\put(0,82.8){\line(1,0){60}}
\color{black}
\put(60,81.8){\small{\textemdash  Nut}}
\end{picture}
\caption{Bermudo's \textit{mi} and \textit{fa} first frets}
\label{fig:bermudo-1-mifa}
\end{figure}

Although these are not true tastini, insofar as his fret spans the entire
fingerboard instead of just the affected course, it is the strongest evidence we have
in support of using any kind of additional fret to create both chromatic and diatonic semitones.

A later reference to tastini from the seventeenth-century comes from Jean Denis who
was a harpsichord builder during the first half of the the century. He refers to
``staggered'' frets on the lute which could be made of ivory. The reference appears in
Lindley's book, and the context in which Denis was discussing tastini was that
someone had tuned a harpsichord in equal temperament. Denis criticizes this approach
stating that someone should perfect the lute and viola da gamba so that they may
accommodate unequal semitones instead of ``ruining a good and perfect tuning in order to
accommodate imperfect instruments.''\autocite[47]{ML:1}.

Another reference comes from Christopher Simpson's \textit{A Compendium of Practical
Music} and appears to describe tastini as they are commonly used today on the theorbo:
\begin{blocks}
I do not deny but that the slitting [\textit{sic}] of the keys in harpsichords and organs, as also the
placing of a middle fret near the top or nut of the viol or theorbo where the space is
wide, may be useful in some cases for the sweetening of such dissonances as may happen
in those places; but I do not conceive that the enharmonic scale is therein concerned,
seeing those dissonances are sometimes more, sometimes less, and seldom that any of them
do hit precisely the quarter of a note. \autocite[51]{CS:1}
\end{blocks}
The first part of Simpson's description matches Bermudo's additional fret exactly, as he describes
an additional fret between the first fret and the nut. He does not state whether or not this middle
fret spans the entire fretboard. The second part of his statement refers to the difference between
the diatonic and chromatic semitone. Simpson calls these ``quarter notes'' as we today might refer
to quarter tones, or half of a semitone. He seems to think that, from a practical standpoint, these
semitones are seldom precisely what they are supposed to be, either diatonic or chromatic. Simpson's
opinions aside, he does provide us with evidence that tastini or additional frets were used
historically in the same places on the fretboard of a lute as players might use them now. Beyond
tastini, there are more possibilities to overcome some of the difficult issues pertaining to
quarter-comma meantone temperament and frets.

\subsection{Other Solutions}

Aside from tastini, there were diverse methods by which lutenists were able to
coax meantone temperaments from their fretting. One of these involved placing frets at
an angle so that a fret could be diatonic on one side of the fingerboard and chromatic on
the other. For the theorbo, this could be used as a substitute for tastini. It
is possible to use an angled first fret, for example, to achieve the chromatic
semitones necessary on the seventh and eighth courses. Instead of placing a tastino at
the left side of the fingerboard, the first fret could be slanted so that it
angled towards the chromatic side of the semitone as it moved towards the
lower courses.

The problem with this is that pitches in the middle of the fret are somewhere between
chromatic and diatonic. Juan Bermudo discusses the practice of angled frets on the
vihuela and comes to the same conclusions:
\begin{blocks}
[...] some players hope to fix the abovementioned faults by putting the frets where the
said faults occur at an angle, taking them out of line. This is not a solution but a
cover-up [...] Take a fret where there is a fault (where it is \textit{mi} for strings
but needs to be \textit{fa} for others) and you will find that, by slanting the fret,
it does not hit any string in the right place. \autocite[112-113]{DE:1}
\end{blocks}
While we might be able to achieve a chromatic semitone at the eighth course, each
successive course would be slightly sharper until reaching the top course. Only the
first and last courses would be truly either chromatic or diatonic, the courses in the
middle would be something in between and not in a specific temperament.

Despite what Bermudo and other writers of the time have said about angled frets, we can
find use for them. Angled frets work best when they are strategically located and used
for frets that might have only one or two useful pitches on them. In a ``standard''
quarter-comma meantone fretting system, as shown in
figure~\ref{fig:quarter-diatonic-complete-a} of the appendix, the sixth fret is diatonic
and duplicates the E$\flat$ and A$\flat$ found at the first fret. A common left-hand
fingering for the first inversion triad, or a chord with a \textit{6} above the bass,
uses the bass on the fifth and sixth courses. These include the first-inversion D major
triad with the F$\sharp$ as well as the first-inversion A major triad with the
C$\sharp$, both found on the fourth fret. However, we lack the G$\sharp$ or D$\sharp$
for either E major or B major tonalities. If we move our sixth fret, which is commonly
diatonic, so that it is placed at an angle, it is possible to get a very close
approximation of a chromatic semitone for these pitches (see
figure~\ref{theorbo-slanted-sixth}). The slanted sixth fret allows us to play the
needed triads with G$\sharp$ and D$\sharp$ in the bass, and disadvantages are minimized
because the sixth fret is not commonly used in other left-hand chord shapes.
\begin{figure}[ht]
\centering
\setlength{\unitlength}{0.5mm}
\begin{picture}(80,70)
% Draw fingerboard edges
\color{black}
\linethickness{0.075mm}
\put(0,0){\line(0,1){70}}
\put(80,0){\line(0,1){70}}

% Draw strings
\color{strings}
\linethickness{0.5mm}
\put(5,0){\line(0,1){70}}
\put(15,0){\line(0,1){70}}
\put(25,0){\line(0,1){70}}
\put(35,0){\line(0,1){70}}
\put(45,0){\line(0,1){70}}
\put(55,0){\line(0,1){70}}
\put(65,0){\line(0,1){70}}
\put(75,0){\line(0,1){70}}

% Fifth course pitch names
\color{black}
\put(24,70){\small{d}}
\put(34,70){\small{g}}
\put(43,70){\small{c'}}
\put(53,70){\small{e'}}
\put(63,70){\small{a'}}
\put(73,70){\small{d''}}

\color{black}
\thicklines
\put(80,26.4){\line(-6,1){80}}
\color{black}
\put(80,25.4){\small{\textemdash  6th (diatonic)}}
\color{black}
\linethickness{1mm}
\put(0,5){\line(1,0){80}}
\color{black}
\put(80,4){\small{\textemdash  7th (chromatic)}}
\color{black}
\linethickness{1mm}
\put(0,60.3){\line(1,0){80}}
\color{black}
\put(80,59.3){\small{\textemdash  5th (diatonic)}}
\color{black}
\put(22,43.3){\small{f$\sharp$}}
\put(32,43.3){\small{g$\sharp$}?}
\put(42,43.3){\small{d$\flat$'}}
\put(52,43.3){\small{f'}}
\put(62,43.3){\small{b$\flat$'}}
\put(72,43.3){\small{e$\flat$''}}
\color{black}
\put(22,15.7){\small{e}}
\put(32,15.7){\small{a}}
\put(42,15.7){\small{d'}}
\put(52,15.7){\small{f$\sharp$'}}
\put(62,15.7){\small{b'}}
\put(72,15.7){\small{e''}}
\end{picture}
\caption{Theorbo with angled sixth fret}
\label{theorbo-slanted-sixth}
\end{figure}

Alternatively, we could employ a tastino underneath these two courses and avoid the
slanted fret altogether; however, such a solution would be at the player's discretion.

Other solutions for achieving a successful quarter-comma meantone temperament do not involve
adjusting frets at all, and advocate positioning the left hand so that individual courses are pulled in
one direction or another to raise their pitch slightly and compensate for frets that are using an
incorrect semitone. Praetorius describes such a method in his \textit{Syntagma Musicum}:
\begin{blocks}
Thus the semitones cannot be either major nor minor, but are, perforce, ``intermediate''
if anything. For I reckon that each fret [...] contains four-and-a-half commas, whereas
the major semitone contains five and the minor semitone only four. Since the error is
only half a comma either way, the ear hardly notices it with these instruments [...]
Major and minor semitones are both produced by the same fret, both sound in tune, [...]
especially since by particular applications of the finger to the string, over the fret,
it is possible to have some control over the pitch of the note produced.
\autocite[68]{MP:1}
\end{blocks}
It is clear that Praetorius is describing meantone temperament, because he
refers to chromatic (minor) and diatonic (major) semitones as having different numbers of
commas. However, Praetorius is actually referring to sixth-comma temperament
which divides its wholetones into nine commas, split four to five, versus a quarter-comma
which contains five commas per wholetone and is divided two to three.

\textit{Syntagma} was published in 1619, so we might assume that sixth-comma meantone had begun to replace quarter-comma
in some musical circles, but it is still a meantone temperament. More importantly, Praetorius describes fretted
instruments as having equal semitones, divided exactly in the middle between chromatic and diatonic, but essentially
played unequally. According to him, the player has the ability to change the quality of the semitones so that they might
be close to chromatic or diatonic as the music requires.

Additional evidence of using finger pressure to correct pitch problems is found in Bermudo's treatise. He advises
players either to locate the note somewhere else on the fingerboard or to use finger pressure to alter it, and refers to
this method several times in his treatise, indicating that it might have been preferable to the addition of extra frets.
\autocite[106]{DE:1} Bottrigari also describes this same technique, when he says players ``touch their frets a little
higher or a little lower'', as I mentioned earlier. \autocite[15]{HB:1}

Bending pitches using variable pressure in the left-hand, such as Praetorius, Bermudo, and Bottrigari describe, is
possible, but not easily done. Another common way for a lutenist to bend pitches is to pull the course with the finger
to one side or the other. This technique is very common in twentieth-century classical guitar literature where extreme
fluctuations of pitch are exploited for various compositional reasons. Although there is no evidence of its use in
historical lute tablatures or other musical sources for lute or theorbo, it is possible to conjecture that players
during that time might have found a way to utilize it in one form or another in order to adjust to meantone in ensemble
playing.

\section{Meantone Fretting in Tablature Sources}

Thus far, we have discussed the methods by which lute players can adapt to the problems of quarter-comma meantone
fretting systems in ensemble situations. For the most part, these methods require the player to place certain notes on
certain frets. In ensemble music, where basso continuo is used, the player has the ability to do this because the music
is in staff notation and no left-hand pitches are dictated anywhere. In fact, it is customary for players to move bass
notes into different octaves when necessary and even use reduction methods that omit repetitive notes. In essence, the
player may re-compose sections of his or her part to fit the instrument's compass and make it sound as idiomatic as
possible.

In lute tablature sources, the placement of notes in the left-hand is precisely dictated. The player usually does not
move notes to different locations on the fretboard in order to compensate for any potentially incorrect semitones in a
meantone temperament. Because of this fact, tablature sources alone indicate that quarter-comma meantone temperament is
not a viable temperament for solo lute literature in the late sixteenth and early seventeenth centuries. Additionally,
ensemble music for lute and voice, where the lute is an accompanying instrument using tablature rather than basso
continuo notation, further indicates that quarter-comma meantone is not usable, and a different kind of temperament
would be preferable, such as sixth-comma meantone or another temperament specific to the lute.

The lute song repertory is a unique genre featuring a lute part written in tablature, one or more
voices, and sometimes a bowed bass. Just as with solo music, players usually do not take the same
liberties with pitch placement as they do when playing basso continuo. In 1597, John Dowland published the
first book of music written in this new genre. Entitled \textit{The First Booke of Songs or Ayres},
his accompaniments were very complete, and equal in caliber to his solo works for lute. The songs
were composed in keys idiomatic for the lute in Renaissance tuning, such as G minor or major. Near the
end of Dowland's first book of Ayres is a well-known song entitled ``Come heavy sleep'' that opens
in G major but has a very striking key change to B major about mid-way through the song at measure 9.
Such a change of tonality befits the subject matter of the song, but if we look closely at
Dowland's placement of the pitches on the lute, they are at odds with a quarter-comma fretting
system.
\begin{example}[h]
\centering
\includegraphics{examples/come.pdf}
\caption{Dowland, ``Come heavy sleep'' from \textit{The First Booke of Songs or Ayres} (1597), m. 9}
\label{dowland-come}
\end{example}
In example~\ref{dowland-come}, there are repeated instances of F$\sharp$ and D$\sharp$ in the
first measure of our example, which are represented in the tablature part by the character
\textit{b}. English lute music was written in the French system of tablature, so the
frets are indicated with letters. The tablature character \textit{a} is the open
string and the character \textit{b} is the first fret. If we were trying to follow the
meantone fretting system I outlined earlier, these notes would be G$\flat$ and E$\flat$,
and would sound quite strident against the B. Recalling Dowland's own choice for the
first fret, as shown in table~\ref{table:comparison}, the quality of semitone is diatonic,
but is in sixth-comma and not quarter-comma. A sixth-comma fret would certainly be more
palatable in this case and perhaps explains why Dowland himself was advocating a sixth-comma
diatonic semitone instead of a quarter-comma one.

Other examples from Dowland's works highlight the central problem with employing meantone fretting
systems on lutes. Because of the way in which fretted instruments are tuned, there are cases when
both the diatonic and chromatic semitone are required at the same fret. For example, in his song
\textit{I saw my lady weepe}, there are both G minor chords and D major chords, which require a
B$\flat$ and an F$\sharp$. However, looking at an excerpt from the song below, we can see that
these notes are both placed on the same fret.
\begin{example}[h]
\centering
\includegraphics{examples/saw.pdf}
\label{dowland-saw}
\caption{Dowland, ``I saw my lady weepe'' from \textit{The Second Booke of Songs or Ayres} (1600), m. 9}
\end{example}
The notes and their corresponding tablature characters are highlighted in red. Both
notes occur on the first fret, where there is the tablature character \textit{b}. If
we were to use a meantone fretting system, the B$\flat$ would be true but the F$\sharp$
would be a G$\flat$ instead, unless we were using a tastino to correct the problem.
This kind of issue results when frets determine the quality of the semitone regardless
of what the particular pitch might be. Keyboard and other instruments are exempt from
this kind of issue because their semitones can be tuned independent of one another. A
keyboard or organ can very easily have F$\sharp$ and B$\flat$ existing at the same time.

Referring back to Dowland's own fretting instructions from the previous chapter, he described one kind of fretting
system that remained fixed once it was set. While he seems to use a sixth-comma diatonic semitone at the first fret,
which would ease the problem of having the F$\sharp$ and B$\flat$ on the same fret, he does not mention the use of
tastini or any other corrective frets. Although it would not sound quite as pronounced as in quarter-comma, the
difference between G$\flat$ and F$\sharp$ would still remain, even in sixth-comma meantone. Since other composers and
lute players were subject to the same tuning constraints, it seems Dowland was trying to create his own temperament that
satisfied both semitone requirements in his works. How he managed to do this is still somewhat of a mystery, since his
own fretting instructions are problematic. His contemporaries might have opted for an equal semitone approach by
splitting the difference between frets or opting for positions that simply satisfied their ears. Yet, the ensemble
dilemma would have remained, and applying an equal semitone solution, a sixth-comma meantone solution, or a completely
original system of fretting, would mean that the lute would be attempting to play in tune with an ensemble that was
using a different temperament.

The same issues that affect fretting systems for Renaissance lute also affect the theorbo as well. Although they are
less common, there are tablature accompaniments for the theorbo, and just as we studied lute tablatures for clues
regarding choices of temperament, we can examine theorbo accompaniment tablatures for the same information. An example
of an early seventeenth-century accompaniment comes from Girolamo Kapsberger, a theorbo player and composer who was
active in Rome. In addition to publishing several books of music for solo theorbo, he published four books of villanelle
written for one, two, and three voices, with written accompaniment for guitar and theorbo. Additional instruments could
have been used, doubling the vocal parts, but Kaspberger is non-specific as to which kinds. The vocal and bass parts are
written in staff notation, while the guitar and theorbo parts are written in their own specialized notation. In the case
of the guitar, alphabetic notation is used, where a series of different letters indicate which chord to play. The
theorbo part is in Italian tablature, where numbers are used in place of letters to indicate fret placement and the
order of strings is actually inverted from French tablature, placing the top string of the theorbo on the bottom line of
the tablature staff. For the sake of consistency, I have transcribed Kapsberger's tablature part into French tablature
so we can compare the examples with lute and theorbo.

The song ``All' ombra'', from his first book of villanelle, contains a cadence in A major
towards the end of the piece. The G$\sharp$ and C that are used are highlighted in red.
\begin{example}[h]
\centering
\includegraphics{examples/kaps_ombria.pdf}
\label{kaps-ombria}
\caption{Kapsberger, ``All' ombra'' from \textit{Di Villanelle, bk. 1} (16??), mm. 21--22 }
\end{example}
Similar to our previous example from Dowland, these two notes are found on the
same fret, indicated with the tablature character \text{b} highlighted in red.
If we were employing quarter-comma meantone on our theorbo, the G$\sharp$ would
instead be an A$\flat$. If this indeed was the case, this difference in semitone
quality could simply have been accepted. On the other hand, it seems more
likely that an alternative temperament was chosen that would have made the
G$\sharp$ more usable.

If Kapsberger's theorbo was definitely in quarter-comma meantone, a tastino would have been the only solution to avoid
the A$\flat$ issue on his first fret. An alternative solution that I proposed earlier would be to change the left-hand
fingering of the E major chord so that the G$\sharp$ on the fourth fret is used resulting in a chromatic semitone and
not a diatonic one. However, Kapsberger's tablature clearly indicates that the G$\sharp$ on the first fret is to be
used.

While these examples represent only a fraction of the types of problems that lute players faced when tuning to meantone
temperaments, the issue of chromatic versus diatonic frets was so pervasive in the literature that it was impossible to
ignore. According to historical evidence, some of the techniques lute players employed in their attempts to address the
conundrum were controversial. When we are playing in meantone temperaments today, we must be willing to do the same and
apply our own solutions, controversial or otherwise. In the concluding chapter of this study, I will summarize all of
the findings presented here and how we might apply them in today's performances.
