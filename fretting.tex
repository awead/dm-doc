% Lute fretting systems

\chapter{Lute Fretting Systems}

In the previous chapter, I set forth several different methods in which we can measure ratios of intervals to determine
the quality of their temperament. In this chapter, I will apply these methods to different fretting schemes and
determine, from a mathematical standpoint, what kind of tempered interval each fret represents. The overall picture for
lute temperaments from the sixteenth through early seventeenth centuries can be outlined as follows: Pythagorean tunings
were used exclusively at the beginning of the period, and were retained for certain intervals later in the period. By
the mid-sixteenth century, lute players began to add different types of tempered intervals into the existing Pythagorean
scheme. Some of these intervals were meantone, some approximating equal temperament, and others were original in their
quality. From the end of century onward, fretting systems remained mixed, or else used methods of dividing the octave
into twelve equal semitones; however, their semitones did not always match those of modern equal temperament. The
important distinction between these systems and keyboard temperaments at the time was that lute temperaments did not
fall completely into one kind of tuning or temperament and were specific to the instrument.

The majority of instructions published during this period provided the player with a practical means of setting frets
and did not dwell on theory. Almost all of the sources have very precise rules for determining fret placement that use
measuring tools such as a straightedge and compass. These were commonly used for making geometrical divisions, and
when used to calculate fret placement could divide a line into any number of equal parts. The remainder of the sources
are less precise and instruct the player to place a fret somewhere near another until it sounds agreeable, or to move a
fret slightly in one direction or the other, without providing exact measurements.

What follows is a comparison of several fretting systems that appeared in lute and vihuela treatises. For those sources
that have instructions on placing frets using a compass, I have determined the exact ratio of the interval using the
arithmetic  outlined in the previous chapter. The details of the calculations can be found in the appendix.

\section{Pythagorean Tunings for Lute}

Starting in the 9th century, most fretted instruments were tuned according to Pythagorean ratios. The placement of
frets on these instruments yielded pure fifths, fourths, and octaves, as well as wholetones with a ratio of 9:8.
\autocite[9]{ML:1} The result of this, as we saw in the previous chapter, was that major thirds were much wider than
pure. What we know of the lute prior to 1500 suggests that it has a four or five strings or courses tuned in fourths.
Such a tuning would accommodate Pythagorean tuning quite well, but after 1500, the lute's repertoire changed and a sixth
string was added. The new tuning of the 6-course lute used a major third between the instrument's third and fourth
courses which exposes the major limitation of a Pythagorean system of tuning.

Because theory treatises in the early sixteenth century still advocated Pythagorean systems of tuning due to its
mathematical principles, lute players were left to forge their own solutions to the tuning dilemma. One of the
mainstays of lute repertoire throughout the sixteenth century was intabulated polyphonic music. The first books of music
printed by Petrucci in the early 1500s contain pieces of this kind, and composers of lute music included many of them in
their publications. Because this music had many harmonies that relied on thirds, a Pythagorean system of tuning would
have presented some obvious shortcomings. It is at this point that we see lute players begin a gradual shift away from
Pythagorean tunings and towards meantone temperaments.

During the first half of the sixteenth century, at least three different tuning methods for lute were published. Two of
these retained the existing Pythagorean tradition of the day, while the third departed from it and employed a kind of
meantone tuning similar to that which Pietro Aron and Vicentino were developing during this period. \textit{Epithoma
musice instrumentalis}, published in 1530 by Oronce Fin\'{e}, was one of the sources that retained Pythagorean
principles. Fin\'{e} was a professor of mathematics at the University of Paris who published mostly theoretical works,
but it was his personal interest in music that compelled him to write \textit{Epithoma}. Written in Latin, the book
includes instructions for tuning a lute, reading tablature, and setting frets. Pierre Attaingnant was a friend of
Fin\'{e}'s and it was probably this friendship that resulted in the treatise's publication. An additional set of lute
instructions appear in Attaingnant's publication \textit{Tres breve et familiere introduction pour entendre et apprendre
[...]} in 1529, this time written in French. Some stipulate that these were instructions from Fin\'{e} and just
translated, but this has never been fully proven.\autocite[1]{PV:1}

Another example of Pythagorean tuning appeared in a book published about the middle of sixteenth-century. The book was
not about music but instead contained various topics concerning France. Appearing in 1556 and carrying the long title of
\textit{Discours non plus melancoliques que diverses, de choses mesmement, qui appartiennent a notre FRANCE: \& a la fin
La maniere de bien \& iustement entoucher les Lucs \& Guiternes}, it has been attributed to Bonaventure des Periers, but
some sources list it as an anonymous author. The subject of tuning lutes and guitars appears in the last chapter and has
nothing to do with any of the other chapters in the book. The final chapter of \textit{Discours [...]} contains
instructions for fretting and tuning these instruments as well as a diagram of a mesolab, an ancient geometric tool that
was first used to solve the problem of dividing a string into equal semitones.

In his assessment of their fretting instructions, Mark Lindley has concluded that both Fin\'{e} and the anonymous
\textit{Discours [...]} preserved the same characteristics of other Pythagorean tuning methods. These included earlier
theorists from the fifteenth century such as Henri Arnaut, Johannes Legreze, Nicola Burzio, and Franchino Gafurio, as
well as more contemporary theorists like Heinrich Schreiber of Erfurt, and Pietro Aron \autocite[11]{ML:1}. The features
of these fretting schemes resulted in fourths and fifths at ratios of 4:3 and 3:2 respectively, and wholetones with a
ratio of 9:8. Semitones depended on which fret the measurements began. In \textit{Discours [...]}, frets 2, 4, and 6
were a series of 9:8 ratios beginning with the nut, while frets 1, 3, 5, and 7 were series of 9:8 ratios beginning with
the seventh fret and moving backwards. Fin\'{e}'s method was slightly different, placing frets 6, 8, and 10 at 9:8
ratios starting from the twelfth fret.

The inherent problem with these systems is that the fret placement is based purely on mathematics and maintaining
Pythagorean ratios without any deference to practical musicianship. Fin\'{e} was not a musician, and Lindley suggests
that the author of \textit{Discours [...]} was not either \autocite[11]{ML:1}. Furthermore, after attempting to play
various pieces from the period, he concludes that ``it seems doubtful to me that sensitive players would really have
left the pythagorean scheme unaltered.'' \autocite[13]{ML:1} This becomes apparent in later sources from Juan Bermudo
and Sylvestro Ganassi who each began with a Pythagorean tuning scheme and then adjusted it to make it more palatable.

Successive instructions on fretting from the mid-sixteenth century onward de-emphasized Pythagorean systems in favor of
tempered intervals and methods of equal fretting. While they certainly acknowledged Pythagorean intervals, and writers
such as Bermudo and Ganassi offered complete fretting systems using Pythagorean ratios, they were never intended to be
the final solution. Most musicians recognized the importance of Pythagorean intervals for the fourth, fifth, octave, and
wholetone; however, they also used non-Pythagorean intervals to create a better solution. The primary distinction
between sources that advocated Pythagorean tuning over a tempered kind was that Pythagorean sources tended to focus more
on mathematical issues than musical ones.


%
% Hans Gerle
%
\section{Gerle's Fretting Instructions}

The first major break with Pythagorean systems of tuning for lute came from Hans Gerle, a lutenist and composer who
published his treatise in 1533. In addition to its specific explanation of fret placement, it is also one of the best
sources for lute instruction and practice in the early sixteenth century. It contains detailed explanations of tuning,
right-hand and left-hand technique, and tablature. The most important feature of Gerle's treatment of lute fretting,
and the others that I will be analyzing in this chapter, is that Gerle was a lute player and not a theorist. From this
viewpoint, matters of theory are discarded in favor of practicality. Gerle probably knew very little of the
implications of breaking with the tradition of Pythagorean tuning. His emphasis was on finding a practical method of
fretting the lute so that it would sound agreeable. The importance of his instructions become more apparent when we see
other lutenists such as John Dowland borrowing many of Gerle's methods over 75 years later.

Gerle's instructions are directed towards a player who may or may not have theoretical knowledge of music, but would
have had some rudimentary knowledge of arithmetic and geometry. His process involved using a compass and a
straightedge, such as a piece of wood or other material that the player cuts to be the same length as the vibrating
string length of the instrument. Marks were made on the straightedge, using the compass to divide the distances between
each mark into different parts. Once all the appropriate marks were made, they could be transferred to the fretboard
and the frets placed accordingly.

He uses a very straight-forward, step-by-step approach for setting each fret, beginning
with the twelfth.
\begin{blocks}
Take a straight-edge that is thin or else a flat piece of wood like a ruler, and make it
of such a length that at the top it touches the piece of wood that the strings lie on and
also touches the bridge that the strings lie on, and when you have made the ruler so that
it touches at both ends (don't make it too short; it must touch as I have said), mark the
bottom part at the bridge with an a, and the top part with a b, so that you will know
which end belongs to the bridge. Then lay the ruler on a table, and take a compass and
find the middle of the ruler. Mark it with a point or little dot and put an m there.
\end{blocks}
The letter \textit{a} marks the bridge and the letter \textit{b} marks the nut.
By placing the twelfth fret in the middle of the string, he divides the string in half
yielding $ \frac{1}{2} $, or in terms of a ratio, this would be an octave of 2:1.
Compasses were widely used at this time to perform all kinds of geometrical divisions and
we can assume most learned individuals at this time would know how to use one in order to
execute Gerle's instructions. Euclid, the Greek mathematician, described procedures for
dividing lines into equal parts in his famous work \textit{Elements}. This book was
first translated into Latin in 1482 and appeared in English translation by 1570. Although
Gerle was German, it is reasonable to assume that he could have had access to either a
copy in Latin or one in German. As far as we know, Gerle was not educated at a university
and probably had little or no knowledge of Latin; however, it is still reasonable to
assume that with Euclid's work in circulation, he could have come by the knowledge needed
to perform the required geometrical calculations.

% Gerle's seventh fret
Gerle continues using pure intervals for the fifth, or seventh fret:
\begin{blocks}
Then divide from the m to the b [in] three parts; and the first part from the m gives you
the seventh and lowest fret. Mark it with a dot and put the number 7 there.
\end{blocks}
The letter \textit{b}, marked in the previous step, denotes the nut at one end of the ruler.
Gerle now switches to numbers for frets and indicates the seventh fret with the number 7.
The fret is marked at the first of the three parts starting from the twelfth fret or
middle of the string. The resulting fret placement is one-third the length of the string,
making the vibrating length of the remaining string two-thirds. We can express this
semantically as a string divided into three parts with a vibrating length of two parts, or
3:2, which corresponds to the pure fifth. Gerle could have divided the entire string into
three parts and placed the seventh fret at the first part from the nut, but it seemed more
important to base successive frets on the location of existing ones, in this case, the
twelfth.

% Gerle's first fret
Gerle's instructions continue with the first fret, and similar to his instructions for the
seventh, he builds on the calculations of the previous fret to find its location.
\begin{blocks}
Then divide elevenfold from the number to the b, and two of the same parts down from the b
give you the first fret. Mark this also with a dot and put the number 1 there.
\end{blocks}
The ``number'' he is referring to is seven, or the seventh fret that we just marked in the
previous step. Here he has us divide the distance from the nut to this fret in eleven
parts and mark the second of these parts starting from the nut. This creates a ratio of
33:31, which is not Pythagorean but instead somewhere between the diatonic semitone
16:15 and the ratio 17:16. Mark Lindley has determined that this distance is actually a
sixth-comma semitone, but what kind? In chapter two, we examined systems of equal
division that divided the octave into multiple parts greater than twelve. Sixth-comma
meantone corresponded to an octave divided into 55 parts, and the first fret on the lute
would be the first semitone in a 55-division system. If the semitone is chromatic, 
it has four parts, and if it is diatonic, it has five.

To determine which semitone Gerle was using we can compare our calculations of his fret
distance with the known values of different semitones. If we express Gerle's 33:31 ratio
as a decimal by dividing 33 by 31, we get $ \frac{33}{31} = 1.0645 $ rounded off
to the nearest fourth decimal place.
\begin{table}[h!]
  \begin{center}
  \begin{tabular}{ r l }
    Equal temperament semitone:   & $ 2^\frac{1}{12} = 1.0600 $ \\
    Sixth-comma chromatic semitone: & $ 2^\frac{4}{55} = 1.0517 $ \\
    Sixth-comma diatonic semitone: & $ 2^\frac{5}{55} = 1.0650 $ \\
  \end{tabular}
  \end{center}
  \caption{Equal and sixth-comma semitones}
  \label{table:6semitones}
\end{table}
Comparing that value to the values of other semitones in table ~\ref{table:6semitones}
we can see that Gerle's first fret matches the diatonic semitone in sixth-comma
meantone temperament.

Gerle has departed from Pythagorean tuning and opted for a tempered semitone at the first
fret. This is not to say that he is advocating a sixth-comma meantone tuning overall.
Gerle still uses a pure fifth at the seventh fret, which is not the same as a
fifth tuned in sixth-comma:
\begin{table}[h!]
  \begin{center}
  \begin{tabular}{ r l l }
    Pure fifth:    & 3:2         & $ = 1.5000 $ \\
    Sixth-comma fifth: & $ 2^\frac{32}{55} $ & $ = 1.4967 $ \\
  \end{tabular}
  \end{center}
  \caption{Comparison of pure and sixth-comma fifths}
  \label{table:6fifths}
\end{table}
The fifth is comprised of three wholetones and one diatonic semitone. In a sixth-comma
system this makes up 32 parts or \textit{dieses}, and the resulting interval is slightly smaller
than the pure Pythagorean fifth, which is what we should expect in any regular meantone
temperament (see table ~\ref{table:6fifths}). This will have implications when we examine
the internal tuning issues of lutes later.

% Gerle's second fret
After having placed the first fret, Gerle continues with the placement of the second, or the wholetone.
Here, he returns to the standard Pythagorean 9:8 ratio:
\begin{blocks}
Then divide again from the number 7 to the b threefold, and the first part down
from the b gives you the second fret. Mark it also with a dot and put the
number 2 there.
\end{blocks}
Here we have a three part division from the seventh fret to the letter \textit{b}, 
which was marked in the first step at the nut.
By returning to our earlier calculation of the seventh fret, we can formulate the
second fret distance and vibrating length accordingly which results in the Pythagorean
wholetone.

% Gerle's fifth fret
So far, Gerle has mixed Pythagorean pure intervals with tempered ones, such as
the fifth, the major second, and the tempered minor second. As he moves to the
perfect fourth at the fifth fret, he writes:
\begin{blocks}
Then divide from the m to the b in two parts, and the one part gives you the
fifth fret. Mark it with a dot and put the number 5 there.
\end{blocks}
Starting from the twelfth fret, Gerle divides the distance into two parts and takes the
first part from the nut as the mark for the fret. The letter m is our octave fret, the
exact midpoint of the string. Since he is now dividing that again in half, this
produces a Pythagorean interval of the fourth with a ratio of 4:3.

% Gerle's sixth fret
Now Gerle turns his attention to the tritone, or sixth fret:
\begin{blocks}
Then put the sixth fret in the middle of the fifth and seventh frets. Make it
with a dot and put the number 6 there.
\end{blocks}
If we assume that Gerle is using ``middle'' to denote the arithmetic mean, as opposed
to the geometric mean, we can calculate the middle distance between the
fifth and seventh frets and get a vibrating ratio of 24:17. However, an alternative
explanation is that Gerle is being intentionally vague here as if to tell the player to
put the fret somewhere between the two frets, and not necessarily exactly in the
middle. If we assume this, then it seems reasonable to infer that the player
could also alter the placement until it produced a good result.

Regardless of our interpretation of Gerle's instruction, the tritone is neither Pythagorean nor sixth-comma meantone,
but it may be close enough to be considered equally tempered. In a sixth-comma meantone with a 55-division octave, the
tritone contains either 27 or 28 dieses depending on whether the semitone above the fourth is chromatic or diatonic.
For a lute tuned in G, this would be the difference between a C$\sharp$ and a D$\flat$ on the instrument's first course.
Calculating the decimal equivalents for these two semitones shows us that Gerle's tritone falls somewhere between the
chromatic and diatonic semitones. Although this fret is not identical numerically to the equally-tempered tritone, if
we look at a scaled drawing depicting the location of the fret in relation to the other intervals, Gerle's fret and the
location of the equally-tempered fret look nearly identical. We must note that Gerle places the fret in the middle of
the other two frets without the use of a compass. Furthermore, while Gerle's sixth fret may come close enough to equal
temperament that we could call it an equally tempered interval, he is not using equal temperament in the modern sense.
In fact, Gerle was using meantone in its original sense, the way in which Aron would have tuned.

Meantone temperaments that did not use equal division had to approximate some of their semitones by dividing the
wholetone differently. For example, Aron's original instructions did not consider every semitone in the scale, it only
covered a set of thirds that were tuned pure leaving some of the other pitches indeterminate. Semitones that were not
included in the scheme were divided, splitting a wholetone into two different parts. Since this was probably done by ear
and without the aid of exact measurements, the calculation of this fret is an approximation and we should discount the
use of arithmetic to determine the quality of the semitone between the fifth and seventh frets.

In order to better visualize frets that are approximations instead of exact calculations, we can can look at a scaled
drawing comparing the placement of the different semitones. I will be using many such scaled drawings and each will use
the same sample mensur length of 70 centimeters. The purpose of the drawing is not to show the exact placement of frets
that would normally be approximate, but to demonstrate how one type of inexact fret may be similar to another type of
fret that is exact in its placement by showing how close they are to another. The choice of mensur length is based on
what best appears on the page. While 70 centimeters is unusual for a lute, a more common mensur length of 60
centimeters is smaller on the page and does not exhibit the differences of semitone size as clearly.

In turning to Gerle's sixth fret, we can look at one of these scaled drawings comparing
his fret with the known placement of the chromatic sixth-comma semitone, the diatonic
sixth-comma semitone, and the equally tempered semitone. These frets are shown in red
while Gerle's frets are in black.
\begin{figure}[ht]
\centering
\setlength{\unitlength}{1mm}
\begin{picture}(60,68.3)
% Draw fingerboard edges
\color{black}
\linethickness{0.075mm}
\put(0,0){\line(0,1){68.3}}
\put(60,0){\line(0,1){68.3}}

% Draw strings
% 6th course
\color{strings}
\linethickness{0.5mm}
\put(5,0){\line(0,1){68.3}}
\linethickness{0.25mm}
\put(7,0){\line(0,1){68.3}}
% 5th course
\put(15,0){\line(0,1){68.3}}
\put(17,0){\line(0,1){68.3}}
% 4th course
\put(25,0){\line(0,1){68.3}}
\put(27,0){\line(0,1){68.3}}
% 3rd course
\put(35,0){\line(0,1){68.3}}
\put(37,0){\line(0,1){68.3}}
% 2nd course
\put(45,0){\line(0,1){68.3}}
\put(47,0){\line(0,1){68.3}}
% 1st course
\put(56,0){\line(0,1){68.3}}
\color{markers}
\linethickness{0.5mm}
\put(0,30.2){\line(1,0){60}}
\color{black}
\put(60,29.6){\tiny{\textemdash diatonic sixth-comma semitone}}
\color{markers}
\linethickness{0.5mm}
\put(0,33.3){\line(1,0){60}}
\color{black}
\put(60,32.7){\tiny{\textemdash equal semitone}}
\color{markers}
\linethickness{0.5mm}
\put(0,36.4){\line(1,0){60}}
\color{black}
\put(60,35.8){\tiny{\textemdash chromatic sixth-comma semitone}}
\color{black}
\linethickness{1mm}
\put(0,34.1){\line(1,0){60}}
\color{black}
\linethickness{1mm}
\put(0,63.3){\line(1,0){60}}
\color{black}
\put(60,62.3){\small{\textemdash 5th fret}}
\color{black}
\linethickness{1mm}
\put(0,5){\line(1,0){60}}
\color{black}
\put(60,4){\small{\textemdash 7th fret}}
\end{picture}
\caption{Scale drawing of Gerle's sixth fret}
\label{fig:gerle-6}
\end{figure}

Gerle's fret favors the chromatic side, although given its proximity, his fret is most similar to
an equally tempered semitone. Another point to bear in mind is that as the mensur length decreases,
so will the distances between these frets. Conversely, the distances will increase as the mensur
increases. This will have more bearing on the theorbo, which can have a mensur length of 80 centimeters
or more, and is discussed in later chapters where I focus on executing different types of fret
placements. 

Gerle next instructs us where to place the third fret by taking a portion of the 
distance from the nut to the first fret.
\begin{blocks}
Then divide from the number 1 to the b [in] three parts, and when you have the
three parts then go with the compass unaltered down from the number 1 again five
spans; that gives you the third fret. Mark it with a dot and put the number 3
there.
\end{blocks}
Gerle's uses the term ``span'' to denote one of the three parts in the distance from
the first fret to the nut. Therefore, the total spans from the nut (b) to the third fret 
is eight: the five spans
from the first fret to the third, plus the original three spans from the nut to
the first fret. The resulting 99:83 ratio is closest to a diatonic sixth-comma
meantone minor third.
\begin{figure}[ht]
\centering
\setlength{\unitlength}{1mm}
\begin{picture}(60,45.4)
% Draw fingerboard edges
\color{black}
\linethickness{0.075mm}
\put(0,0){\line(0,1){45.4}}
\put(60,0){\line(0,1){45.4}}

% Draw strings
% 6th course
\color{strings}
\linethickness{0.5mm}
\put(5,0){\line(0,1){45.4}}
\linethickness{0.25mm}
\put(7,0){\line(0,1){45.4}}
% 5th course
\put(15,0){\line(0,1){45.4}}
\put(17,0){\line(0,1){45.4}}
% 4th course
\put(25,0){\line(0,1){45.4}}
\put(27,0){\line(0,1){45.4}}
% 3rd course
\put(35,0){\line(0,1){45.4}}
\put(37,0){\line(0,1){45.4}}
% 2nd course
\put(45,0){\line(0,1){45.4}}
\put(47,0){\line(0,1){45.4}}
% 1st course
\put(56,0){\line(0,1){45.4}}
\color{markers}
\linethickness{0.5mm}
\put(0,5){\line(1,0){60}}
\color{black}
\put(60,4.4){\tiny{\textemdash diatonic sixth-comma semitone}}
\color{markers}
\linethickness{0.5mm}
\put(0,6.8){\line(1,0){60}}
\color{black}
\put(60,6.2){\tiny{\textemdash equal semitone}}
\color{black}
\linethickness{1mm}
\put(0,40.4){\line(1,0){60}}
\color{black}
\put(60,39.4){\small{\textemdash 2nd fret}}
\color{black}
\linethickness{1mm}
\put(0,5.09999999999999){\line(1,0){60}}
\end{picture}
\caption{Scale drawing of Gerle's third fret}
\label{fig:gerle-3}
\end{figure}

In looking at the scale drawing, the black line representing the fret completely
occludes the red line of the diatonic sixth-comma semitone, indicating that for all
practical purposes the two semitones are the same. Longer mensur lengths might exhibit
a slight difference in distance but it would be minimal.

The last fret is the fourth which makes the major third. Similar to the sixth fret, he
averages the distance between the third and fifth.
\begin{blocks}
Then put the fourth fret between the third and the fifth frets. Mark it with a
dot and put the number 4 there.
\end{blocks}
Averaging the distance between the two frets results in the ratio 792:629, which comes
closest to the equal tempered major third. Referring to a drawing of the fourth fret,
Gerle's major third is flatter than a true equally tempered third, but sharper than a
major third in a sixth-comma temperament.
\begin{figure}[ht]
\centering
\setlength{\unitlength}{1mm}
\begin{picture}(60,71.9)
% Draw fingerboard edges
\color{black}
\linethickness{0.075mm}
\put(0,0){\line(0,1){71.9}}
\put(60,0){\line(0,1){71.9}}

% Draw strings
% 6th course
\color{strings}
\linethickness{0.5mm}
\put(5,0){\line(0,1){71.9}}
\linethickness{0.25mm}
\put(7,0){\line(0,1){71.9}}
% 5th course
\put(15,0){\line(0,1){71.9}}
\put(17,0){\line(0,1){71.9}}
% 4th course
\put(25,0){\line(0,1){71.9}}
\put(27,0){\line(0,1){71.9}}
% 3rd course
\put(35,0){\line(0,1){71.9}}
\put(37,0){\line(0,1){71.9}}
% 2nd course
\put(45,0){\line(0,1){71.9}}
\put(47,0){\line(0,1){71.9}}
% 1st course
\put(56,0){\line(0,1){71.9}}
\color{markers}
\linethickness{0.5mm}
\put(0,37.9){\line(1,0){60}}
\color{black}
\put(60,37.3){\tiny{\textemdash chromatic sixth-comma semitone}}
\color{markers}
\linethickness{0.5mm}
\put(0,35.6){\line(1,0){60}}
\color{black}
\put(60,35){\tiny{\textemdash equal semitone}}
\color{markers}
\linethickness{0.5mm}
\put(0,30.9){\line(1,0){60}}
\color{black}
\put(60,30.3){\tiny{\textemdash diatonic sixth-comma semitone}}
\color{black}
\linethickness{1mm}
\put(0,35.9){\line(1,0){60}}
\color{black}
\linethickness{1mm}
\put(0,5){\line(1,0){60}}
\color{black}
\put(60,4){\small{\textemdash 5th fret}}
\color{black}
\linethickness{1mm}
\put(0,66.9){\line(1,0){60}}
\color{black}
\put(60,65.9){\small{\textemdash 3rd fret}}
\end{picture}
\caption{Scale drawing of Gerle's fourth fret}
\label{fig:gerle-4}
\end{figure}

With our sample mensur length of 70 centimeters, the actual difference
between Gerle's major third and a sixth-comma meantone major third is about 2
millimeters. The difference between Gerle's third and a quarter comma meantone third
is 4 millimeters. So when speaking of comparisons that account for a few thousandths of
a decimal, the numbers themselves are small, but when translated into a real
instrument, the differences become large.

Gerle limits his fret placement calculations to the seventh fret only; however,
he does provide instructions for placing an eighth, but does not leave us any
geometrical calculations to do so.
\begin{blocks}
But if on the lute one wants eight frets, then let him make the eighth fret a
little closer to the seventh fret than the sixth is.
\end{blocks}
Taken at face value, we cannot really determine what Gerle exactly means. The
distance between frets would naturally decrease as they ascended the fretboard.
At this point, we are to assume that the player may use his or her ear to adjust
the fret accordingly. This same rule would hold true for the rest of the frets
as well.

Taken as a whole, Gerle's fretting instructions are a mixed assortment of different
intervals. He uses Pythagorean ratios for the basic intervals, but relies on tempered
ones for the others. These are either sixth-comma meantone in nature, or completely
unique. His approach seems to have taken hold with other players because we see the
same kind of heterogeneous techniques used for many years afterwards.

%
% John Dowland
%
\section{John Dowland's Fretting Instructions}

Gerle's fretting instructions seemed to have survived for quite some time after his death because they appear in an
almost identical form in an an instruction method by the English lutenist John Dowland. One of the best known lutenists
of his day, Dowland was an important composer of songs and solo music for the lute and remains so to this day. During
his lifetime, he published five books for lute and voice, as well as music for viol consort with lute accompaniment.
His solo works survive primarily in manuscript, with the notable exception of the \textit{Varietie of Lute-Lessons}
which was published by his son, Robert Dowland, in 1610 and contained music by John Dowland and some of his
contemporaries.

As the title suggests, the \textit{Varietie} was intended as an instruction book, however the music in it demands a high
level of ability. The book begins with two prefaces, one written by John Baptiste Besard and a second by the elder
Dowland. Besard's preface, ``Necessarie Observations Belonging to the Lute and Lute-playing,'' was published three times
elsewhere: First, in Latin, in Besard's own \textit{Thesaurus Harmonicus}, a compendium of lute music published in 1603,
where it was titled ``De modo in testudine studendi libellus.'' The second time was in 1617, where it was revised
slightly and published in his \textit{Novus Partus} with the title ``Ad artem Testudinis brevi.'' Lastly, it was
published again that same year, in German, in a pamphlet entitled \textit{Isagoge in artem institutionem}. Each version
contained minor differences, but focused on technique rather than tuning or temperament.

Dowland's own instructions, entitled ``Other Necessary Observations belonging to the Lute'' offered additional advice on
choosing lute strings and a method for setting the frets on the instrument that followed Gerle's very closely. While his
instructions postdate Gerle by more than 70 years, it is obvious that Dowland must have known of Gerle's book because of
the similarity of their instructions. It is also possible that they consulted a similar source on geometry and
measurement and reached the same conclusions, but it is more likely that Dowland simply knew about Gerle's instructions
either through his book directly or from common practice and was repeating them in his own writing.

After giving the usual historical account of Pythagoras discovering harmony by listening to the hammers of blacksmiths,
Dowland instructs us to procure a piece of ``whitish'' wood that is just as long as the distance from the inward side of
the nut to the inward side of the bridge on the lute. From there, his fretting instructions are exactly the same as
Gerle's except that Dowland specifies letters instead of numbers since he was using the French style tablature system.

If we account for several printing errors which appear in Dowland's instructions, they
are very straightforward.
\begin{blocks}
Wherefore take a thinne flat ruler of whitish woode, and make it just as long and
straight as from the inward side of the Nut to the inward side of the Bridge, then note
that wnd [\textit{sic}] which you meane to the Bridge with some small marke, and the other end with
the letter A, because you may know which belongeth to the one and to the other. then
lay the ruler upon a Table, and take a payre of compasses and seeke out the just middle
of the Ruler: that note with a pricke, and set the letter N. upon it, which is a
Diapason from the A. as appeareth by the striking of the string open.
\end{blocks}
After describing the same manner of using a ruler in place of the mensur length of the
string, Dowland places the twelfth fret, fret N, at the midpoint of the string creating
a pure 2:1 octave.
\begin{blocks}
Secondly, part the distances from N. to D. in three parts, then the first part gives
you the seaventh fret from the Nut, making a Diapente: in that place also set a pricke,
and upon it the letter H.
\end{blocks}
Dowland makes the first of many typographical errors and uses the letter D instead of
A. If we allow for that mistake and read the passage ``N. to A.'' and not ``N. to D.'', it
is obvious that Dowland is marking a perfect 3:2 fifth at the letter H. When he sets
the first and second frets, he repeats Gerle's measurements almost verbatim, yielding a
33:31 diatonic sixth-comma semitone and a 9:8 Pythagorean wholetone.
\begin{blocks}
Thirdly, devide [\textit{sic}] the distance from the letter H. to the letter A. in eleaven parts: two
of which parts from A. gives the first fret, note that with a pricke, and set the
letter B. thereon, which maketh a Semitone. Fourthly, divide the distance from H. to
the letter A. in three parts, one of which parts from A. upward sheweth the second
fret, note that with a pricke, and set the letter D. upon it, which maketh a wholetone
from A.
\end{blocks}
When he gets to the fifth fret there is another error, mistakenly indicating the first
fret instead of the fifth, but otherwise the same 4:3 pure fourth that Gerle uses.
\begin{blocks}
Fifthly, divide the distance from N. to A. into two parts, there the first part sheweth
you the first [ie. fifth and not first] fret, sounding a Diastessaron: in that place
also set a pricke, and upon it the letter F.
\end{blocks}
The sixth fret is placed the same as Gerle's, as the mean distance between the fifth and
seventh, giving us the same ratio of 24:17.
\begin{blocks}
The sixt fret with is a G. must be placed just in the middle betwixt F. and H. which
maketh a Semidiapente.
\end{blocks}

When Dowland reaches the third fret, he departs from Gerle's geometrical instructions:
\begin{blocks}
Seventhly, divide the distance from the letter B. to A. in three parts, which being
done, measure from the B. upwards foure times and a halfe, and that wil give you the
third fret, sounding a Semiditone: mark that also with a prick, \& set thereon the
letter D.
\end{blocks}
Whereas Gerle would measure five of his spans from the first fret, Dowland only
uses four and a half. This results in a ratio of 198:168 which in our sample diagram
appears the same as the sixth-comma chromatic semitone.
\begin{figure}[ht]
\centering
\setlength{\unitlength}{1mm}
\begin{picture}(60,45.4)
% Draw fingerboard edges
\color{black}
\linethickness{0.075mm}
\put(0,0){\line(0,1){45.4}}
\put(60,0){\line(0,1){45.4}}

% Draw strings
% 6th course
\color{strings}
\linethickness{0.5mm}
\put(5,0){\line(0,1){45.4}}
\linethickness{0.25mm}
\put(7,0){\line(0,1){45.4}}
% 5th course
\put(15,0){\line(0,1){45.4}}
\put(17,0){\line(0,1){45.4}}
% 4th course
\put(25,0){\line(0,1){45.4}}
\put(27,0){\line(0,1){45.4}}
% 3rd course
\put(35,0){\line(0,1){45.4}}
\put(37,0){\line(0,1){45.4}}
% 2nd course
\put(45,0){\line(0,1){45.4}}
\put(47,0){\line(0,1){45.4}}
% 1st course
\put(56,0){\line(0,1){45.4}}
\color{markers}
\linethickness{0.5mm}
\put(0,12.4){\line(1,0){60}}
\color{black}
\put(60,11.8){\tiny{\textemdash chromatic sixth-comma semitone}}
\color{markers}
\linethickness{0.5mm}
\put(0,5){\line(1,0){60}}
\color{black}
\put(60,4.4){\tiny{\textemdash diatonic sixth-comma semitone}}
\color{markers}
\linethickness{0.5mm}
\put(0,6.8){\line(1,0){60}}
\color{black}
\put(60,6.2){\tiny{\textemdash equal semitone}}
\color{black}
\linethickness{1mm}
\put(0,40.4){\line(1,0){60}}
\color{black}
\put(60,39.4){\small{\textemdash 2nd fret}}
\color{black}
\linethickness{1mm}
\put(0,12.1){\line(1,0){60}}
\end{picture}
\caption{Scale drawing of Dowland's third fret}
\label{fig:dowland-3}
\end{figure}

Dowland is intentionally vague here. Instead of using whole numbers as everyone else
does when making geometrical divisions with a compass, he uses fractions. If we wanted
to be exact, we could find the mean of the original span that is used to
find the distance to the third fret and have a true ``four and a half spans,'' but
Dowland does not instruct us to do that.

There are several possible explanations for Dowland's odd calculation: First, Dowland could have intended to leave the
method of calculation to the player. Someone able to execute Dowland's geometrical calculations could have found the
exact number of spans necessary, if he or she wished. Perhaps Dowland simply did not want give the extra detail. A
second explanation is that Dowland knew he preferred a chromatic semitone and used the specification of 4.5 spans as an
approximation. The compelling case for this explanation is that the 4.5 span measurement is so close to a chromatic
semitone in sixth-comma meantone, the same kind of regular meantone temperament that Gerle was using. A third
explanation is that Dowland did not know he was expressing the difference between the chromatic and diatonic semitone
and arrived at the interval ``by ear.'' Either by experimentation or estimation, he found a way to represent the fret
placement mathematically with an approximate measure of 4.5 spans.

% Dowland's fourth fret
As Dowland moves on to the fourth fret, he places it using the same arithmetic mean that
Gerle used in his instructions.
\begin{blocks}
then set the fourth fret just in the middle, the which wil[l] be a perfect
\emph{ditone}:
\end{blocks}
Since Dowland's third fret was slightly smaller than Gerle's, this will make the
placement of his fourth fret different from Gerle's as well, coming to a ratio of
1584:1266, which is very close to a quarter-comma chromatic semitone.
\begin{figure}[ht]
\centering
\setlength{\unitlength}{1mm}
\begin{picture}(60,78.9)
% Draw fingerboard edges
\color{black}
\linethickness{0.075mm}
\put(0,0){\line(0,1){78.9}}
\put(60,0){\line(0,1){78.9}}

% Draw strings
% 6th course
\color{strings}
\linethickness{0.5mm}
\put(5,0){\line(0,1){78.9}}
\linethickness{0.25mm}
\put(7,0){\line(0,1){78.9}}
% 5th course
\put(15,0){\line(0,1){78.9}}
\put(17,0){\line(0,1){78.9}}
% 4th course
\put(25,0){\line(0,1){78.9}}
\put(27,0){\line(0,1){78.9}}
% 3rd course
\put(35,0){\line(0,1){78.9}}
\put(37,0){\line(0,1){78.9}}
% 2nd course
\put(45,0){\line(0,1){78.9}}
\put(47,0){\line(0,1){78.9}}
% 1st course
\put(56,0){\line(0,1){78.9}}
\color{markers}
\linethickness{0.5mm}
\put(0,39.7){\line(1,0){60}}
\color{black}
\put(60,39.1){\tiny{\textemdash chromatic quarter-comma semitone}}
\color{markers}
\linethickness{0.5mm}
\put(0,37.9){\line(1,0){60}}
\color{black}
\put(60,37.3){\tiny{\textemdash chromatic sixth-comma semitone}}
\color{markers}
\linethickness{0.5mm}
\put(0,35.6){\line(1,0){60}}
\color{black}
\put(60,35){\tiny{\textemdash equal semitone}}
\color{markers}
\linethickness{0.5mm}
\put(0,30.9){\line(1,0){60}}
\color{black}
\put(60,30.3){\tiny{\textemdash diatonic sixth-comma semitone}}
\color{black}
\linethickness{1mm}
\put(0,5){\line(1,0){60}}
\color{black}
\put(60,4){\small{\textemdash 5th fret}}
\color{black}
\linethickness{1mm}
\put(0,73.9){\line(1,0){60}}
\color{black}
\put(60,72.9){\small{\textemdash 3rd fret}}
\color{black}
\linethickness{1mm}
\put(0,39.5){\line(1,0){60}}
\end{picture}
\caption{Scale drawing of Dowland's fourth fret}
\label{fig:dowland-4}
\end{figure}

Although this could be an argument in support of quarter-comma meantone on
fretted instruments, it is the only interval in this scale that appears to be tempered
this way. The other intervals are still a mix of Pythagorean and sixth-comma meantone
intervals.

% Dowland's other frets
Dowland goes on to give us instructions for placing the eighth, ninth and tenth frets
instead of stopping at the seventh as Gerle does; however, it is at this point that his
calculations go awry. He places the remaining three frets using the same process:
\begin{blocks}
then take one third part from B. to the Bridge, and that third part from B. maketh I.
which soundeth \emph{Semitonium cum Diapente}, then take a third part from the Bridge
to C. [N.B. He means C. to the Bridge and not the other way around] and that third part
maketh E. [N.B. Another misprint here, he means the ninth fret, K.] which soundeth
\emph{Tonus cum diapente}, or an \emph{Hexachordio maior}. Then take one third part
from D. to the Bridge, and that third part from D. maketh L. which soundeth
\emph{Ditonus cum Diapente}.
\end{blocks}
Apart from the errors with the ninth fret, Dowland is repeating the same pattern for
each remaining fret. He seems to be calculating their placement by dividing the
total distance between one fret and the bridge into three parts and then setting the
fret at one-third of the distance. In looking at the mathematical result of
this process, we find the ratios of the higher frets are actually decreasing when they
should be increasing. Either we are misinterpreting Dowland's instructions because of
further errors in the printed source, or Dowland has simply got his calculations wrong.

\begin{table}[h!]
  \begin{center}
  \begin{tabular}{ r l l }
    99:68 ratio  & $ \frac{99}{68}  $ & = 1.4559 \\
    27:19 ratio  & $ \frac{27}{19}  $ & = 1.4210 \\
    594:426 ratio & $ \frac{594}{426} $ & = 1.3944 \\
  \end{tabular}
  \end{center}
  \caption{Decimal ratios of Dowland's eighth, ninth and tenth frets}
\end{table}

If we return to the calculations that we obtained following Gerle's scheme, the only differences between those and
Dowland's are the third and fourth fret. However, in looking at the semantic instructions themselves the real difference
lies with the third fret. Gerle required 5 spans from the first to the third fret while Dowland required slightly less,
4.5. The semantic instructions for the fourth fret were the same in both sources: place the fourth fret in the middle,
between the other two. The ratio of the fourth fret was only different because the distance to the third fret was
shorter.  If we then entertain the possibility that Dowland was essentially using Gerle's original instructions
directly, but with only a slight modification to the third fret, perhaps due to personal preference, we could then see
that the eighth, ninth and tenth frets were of Dowland's own doing.

The last three frets of Dowland's scheme had no precedent, if we assume that Dowland was using Gerle as his primary
source. Dowland was unique in his procedure for obtaining the right proportional distance for those last frets. Given
the calculations above, we cannot arrive at a usable placement for the last three frets of his scheme, nor can we use
Dowland's own fretting instructions as complete source because there is no mention of the eleventh and twelfth frets
which figure prominently in his music.

\section{Ganassi's \textit{Regola Rubertina}}

Because Dowland and Gerle's fretting instructions are similar, we can make a strong case for using them as a single unit
for comparison, since all but one of their frets align very closely. An important source that would contrast with their
methods is Silvestro Ganassi's instruction method for viola da gamba entitled, \textit{Regola Rubertina}. Although he
was a gamba player and his method is intended for that instrument and not lute, it does contain fretting instructions
that are valuable when applied to the lute.

Ganassi approaches his subject slightly differently than Gerle and Dowland. In the fourth chapter of his treatise, he
specifies the locations of each fret using a compass and string division, just as Gerle and Dowland do; however, in the
fifth chapter, he adjusts some of these frets to different positions by comparing unisons from one string on the
instrument to another. The first, second, third, sixth, and eight frets are adjusted flatter than their initial
placements, while the fourth fret is adjusted sharper. The adjustment of the fourth fret is markedly different from
other treatises of the time; although, the frets that make the wholetone, fourth, and fifth, are all kept to their
Pythagorean ratios.

For example, Ganassi opens his fourth chapter with the second fret, and uses the same
Pythagorean wholetone ratio that everyone else has used at this time:
\begin{blocks}
Please note that the proportion \emph{sesquioctava} produces pitches expressed
by these two number, 9:8. This proportion determines the location of the second
fret. If you divide the string, beginning from the nut on the fingerboard and
ending at the bridge, where the bow is drawn, into nine parts, the first of the
nine parts sets the boundary of the second fret.
\end{blocks}
He also uses the pure 4:3 ratio of the
perfect fourth because he says the open strings that are played against it must
be in unison.
\begin{blocks}
Then, you divide the string into four parts; the first of these four part will
set the location of the fifth fret, which produces the consonance of a fourth,
which is created by the proportion of sesquitertia indicated by a ratio of 4:3
[ \ldots ] This produces the consonance of the perfect fourth, because if
one then plays the open string, which is at the end of the fourth part of the
string length, one achieves the opposite of the 4:3 ratio
\end{blocks}
Most players would have tuned their open strings using the fifth fret of the
adjacent string. In this case, Ganassi is merely underscoring the fact that the fifth
fret should be at the ratio of a pure fourth because the interval of a fourth between
the adjacent strings of the viol should be pure also. The seventh fret is kept to a pure
ratio as well:
\begin{blocks}
Then you divide the string length into three parts. The first of the three parts will
be the end of the seventh fret, thereby producing the consonance of the perfect fifth,
of \textit{diapente}, which is formed with the proportion of \textit{sesquialtera}
indicated by the ratio of 3:2.
\end{blocks}

So for these frets, Ganassi is reiterating what most other theorists believed about tuning the wholetone, fourth, and
fifth by maintaining their Pythagorean ratios. In chapter five, ``Method of Adjusting the Frets'', Ganassi verifies the
positions of these three frets by checking the unisons against the various open strings of the instrument. This ensures
that he wants all the intervals between open strings to be pure Pythagorean ratios. It is important to note here that
this precludes the use of any meantone temperament since the fifths would need to be tempered flatter than pure. We may
also say the same thing about Dowland and Gerle's methods since their fourths and fifths are pure as well.

Once Ganassi has completed setting those frets, he does not revisit them. However, this is not the case for the other
frets of the instrument. It is the other intervals that distinguish Ganassi's ideas about fret placement from some his
contemporaries. He departs from certain meantone and Pythagorean ratios in favor of his own that are produced by
comparing tones from one string and fret to the tones of a different string or fret.

\subsection{Ganassi's Non-Pythagorean Frets}

Ganassi's first fret differs in size from both Gerle's and Dowland's. He begins by having
us place it in a Pythagorean ratio, but eventually moves it to a different
kind of ratio. In chapter four, he writes:
\begin{blocks}
After you have positioned the second fret in the manner described above, the
first fret should be set halfway between the major and minor semi-tones by
their respective proportions. In order not to go into this at length, however,
I believe that I have chosen a similar method for finding the first fret which
produces a minor semitone, quite easily.\autocite[106]{RB:2}
\end{blocks}
Another translation of the same passage from Mark Lindley's books is slightly different:
\begin{blocks}
\begin{center}
\begin{tabular}{r l}
Dapoi che hauerai trouato \&      & Now when you have found and \\
terminato il ditto sec\={o}do tasto  & located the second fret \\
c\={o} il modo ditto di sopra     & by the method given above, \\
il tasto primo sera terminato al meza & the first fret is located halfway \\
tra il scagneleto del manico al    & between the nut and the \\
secondo tasto ma de piu zoe      & second fret, but more, i.e. \\
batt\={e}do di fora la mita della   & down the neck by half the \\
grosseza del tasto . . . \& in questo & fret's width. In this regard \\
ti haueria possuto resonar il     & I could have calculated for you the \\
partimento del semiton maior al minor & division of the major and minor semitones. \\
. . . il primo tasto elqual fa leffeto & The first fret gives the effect \\
del semiton minor . . .        & of the minor semitone. \\
\end{tabular}
\end{center}
\end{blocks}

Taking both translations into account, the placement of the first fret appears to be
halfway between the \textit{scagneleto del manico} or nut and the second fret. When
the Pythagorean wholetone (9:8) is divided in half, the resulting ratio is 18:17.
This is very the close to an equal semitone.
\begin{table}[h!]
  \begin{center}
  \begin{tabular}{ r l l }
    18:17 semitone:       & $ \frac{18}{17} $ & = 1.0588 \\
    Equal temperament semitone: & $ 2^\frac{1}{12} $ & = 1.0595 \\
  \end{tabular}
  \end{center}
  \caption{Comparison of the 18:17 and equally tempered semitones}
\end{table}
It is not exactly the same as a true equally tempered semitone, but is close enough
that many theorists and players advocated using a system of semitones based solely on a
series of 18:17 divisions.

In the sixth chapter, Ganassi changes the position of the first fret yet again,
using the fifth fret of the fourth course as the reference:
\begin{blocks}
Having tuned the open fourth string and having adjusted the fourth fret, you should
then check the fifth fret of the third string. Its pitch will provide the means of
regulating the first fret of the fourth string. \autocite[114]{RB:2}
\end{blocks}
Ganassi numbers the courses of his instrument in the reverse order that we do
today: his ``first string'' is our sixth course, and his ``sixth string'' is our
first course. On a lute in Renaissance G tuning, the above description would
correspond to the unision B$\flat$ found between the fifth fret of the fourth
course and the first fret of the third course. Because this unison will not be
in tune if the first fret is a 18:17 ratio, the fret must be moved towards the
nut, rendering it slightly flat. We can determine the decimal equivalent of
this new adjusted first fret by referring to a chart of frets and their
placements from Richard Bodig's article on Ganassi's \textit{Regola}.
\autocite[67]{RB:3} He calculates new values for each fret using Ganassi's
instructions and determining where the frets are moved after they are initially placed. The
resulting decimal values occur once the player moves the frets to where they are
exactly in unison with one another according to Ganassi's instructions.

Using Bodig's calculations of Ganassi's fret adjustments, and comparing them to the
fret placements we have seen thus far with Gerle and Dowland, Ganassi's first fret is
much flatter than either Dowland or Gerle. Curiously, Ganassi's adjusted first fret is
almost exactly the same as the chromatic sixth-comma semitone.
\begin{figure}[ht]
\centering
\setlength{\unitlength}{1mm}
\begin{picture}(60,53.4)
% Draw fingerboard edges
\color{black}
\linethickness{0.075mm}
\put(0,0){\line(0,1){53.4}}
\put(60,0){\line(0,1){53.4}}

% Draw strings
% 6th course
\color{strings}
\linethickness{0.5mm}
\put(5,0){\line(0,1){53.4}}
\linethickness{0.25mm}
\put(7,0){\line(0,1){53.4}}
% 5th course
\put(15,0){\line(0,1){53.4}}
\put(17,0){\line(0,1){53.4}}
% 4th course
\put(25,0){\line(0,1){53.4}}
\put(27,0){\line(0,1){53.4}}
% 3rd course
\put(35,0){\line(0,1){53.4}}
\put(37,0){\line(0,1){53.4}}
% 2nd course
\put(45,0){\line(0,1){53.4}}
\put(47,0){\line(0,1){53.4}}
% 1st course
\put(56,0){\line(0,1){53.4}}
\color{markers}
\linethickness{0.5mm}
\put(0,48.4){\line(1,0){60}}
\color{black}
\put(60,47.8){\tiny{\textemdash chromatic sixth-comma semitone}}
\color{markers}
\linethickness{0.5mm}
\put(0,40.4){\line(1,0){60}}
\color{black}
\put(60,39.8){\tiny{\textemdash Gerle/Dowland/diatonic sixth-comma semitone}}
\color{markers}
\linethickness{0.5mm}
\put(0,43.5){\line(1,0){60}}
\color{black}
\put(60,42.9){\tiny{\textemdash equal semitone}}
\color{black}
\linethickness{1mm}
\put(0,47.3){\line(1,0){60}}
\color{black}
\linethickness{1mm}
\put(0,5){\line(1,0){60}}
\color{black}
\put(60,4){\small{\textemdash 2nd fret}}
\end{picture}
\caption{Comparison of Ganassi's first fret}
\label{fig:ganassi-1}
\end{figure}

This differs from Dowland and Gerle whose first frets come to match the diatonic
sixth-comma semitone. For a lute tuned in G, this would be the difference between a
G$\sharp$ and an A$\flat$ on the first string. While Ganassi's adjustment renders
the first fret in unison with the fourth, it is not a practical solution for a lute
because the pitches at the first fret are now G$\sharp$ for the first course, as well
as A$\sharp$ for the third.

Ganassi's third fret undergoes a similar transformation. In the fourth chapter, he
sets the fret at a pure minor third. Later in the sixth chapter, he adjusts this
and tunes it to the octave formed between the first fret of the third course and the
third fret of the sixth course. Returning to Bodig's calculations, this results in
lowering Ganassi's third fret from its initial pure minor third to something that is
flatter than an equally tempered minor third.
\begin{figure}[ht]
\centering
\setlength{\unitlength}{1mm}
\begin{picture}(60,45.3)
% Draw fingerboard edges
\color{black}
\linethickness{0.075mm}
\put(0,0){\line(0,1){45.3}}
\put(60,0){\line(0,1){45.3}}

% Draw strings
% 6th course
\color{strings}
\linethickness{0.5mm}
\put(5,0){\line(0,1){45.3}}
\linethickness{0.25mm}
\put(7,0){\line(0,1){45.3}}
% 5th course
\put(15,0){\line(0,1){45.3}}
\put(17,0){\line(0,1){45.3}}
% 4th course
\put(25,0){\line(0,1){45.3}}
\put(27,0){\line(0,1){45.3}}
% 3rd course
\put(35,0){\line(0,1){45.3}}
\put(37,0){\line(0,1){45.3}}
% 2nd course
\put(45,0){\line(0,1){45.3}}
\put(47,0){\line(0,1){45.3}}
% 1st course
\put(56,0){\line(0,1){45.3}}
\color{markers}
\linethickness{0.5mm}
\put(0,6.7){\line(1,0){60}}
\color{black}
\put(60,6.1){\tiny{\textemdash equal}}
\color{markers}
\linethickness{0.5mm}
\put(0,5){\line(1,0){60}}
\color{black}
\put(60,4.4){\tiny{\textemdash Gerle/diatonic sixth-comma}}
\color{markers}
\linethickness{0.5mm}
\put(0,12){\line(1,0){60}}
\color{black}
\put(60,11.4){\tiny{\textemdash Dowland/chromatic sixth-comma}}
\color{black}
\linethickness{1mm}
\put(0,40.3){\line(1,0){60}}
\color{black}
\put(60,39.3){\small{\textemdash 2nd fret}}
\color{black}
\linethickness{1mm}
\put(0,8.70000000000002){\line(1,0){60}}
\end{picture}
\caption{Comparison of Ganassi's third fret}
\label{fig:gnassi-3}
\end{figure}

The drawing for this fret shows that it is approaching the chromatic sixth-comma semitone
of Dowland, but not close enough to be considered equal to it, nor to any other kind of
semitone that we have seen thus far.

Ganassi's initial setting of the fourth fret is halfway between the third and
fifth frets, before the third is adjusted. This gives us a very strange ratio of
48:38. Later, he adjusts the position of the fret so the octave between the second
fret of the sixth string and the fourth fret of the fourth string are in tune.
\begin{figure}[ht]
\centering
\setlength{\unitlength}{1mm}
\begin{picture}(60,44.5)
% Draw fingerboard edges
\color{black}
\linethickness{0.075mm}
\put(0,0){\line(0,1){44.5}}
\put(60,0){\line(0,1){44.5}}

% Draw strings
% 6th course
\color{strings}
\linethickness{0.5mm}
\put(5,0){\line(0,1){44.5}}
\linethickness{0.25mm}
\put(7,0){\line(0,1){44.5}}
% 5th course
\put(15,0){\line(0,1){44.5}}
\put(17,0){\line(0,1){44.5}}
% 4th course
\put(25,0){\line(0,1){44.5}}
\put(27,0){\line(0,1){44.5}}
% 3rd course
\put(35,0){\line(0,1){44.5}}
\put(37,0){\line(0,1){44.5}}
% 2nd course
\put(45,0){\line(0,1){44.5}}
\put(47,0){\line(0,1){44.5}}
% 1st course
\put(56,0){\line(0,1){44.5}}
\color{markers}
\linethickness{0.5mm}
\put(0,37.9){\line(1,0){60}}
\color{black}
\put(60,37.3){\tiny{\textemdash diatonic sixth-comma semitone}}
\color{markers}
\linethickness{0.5mm}
\put(0,35.9){\line(1,0){60}}
\color{black}
\put(60,35.3){\tiny{\textemdash Gerle/equal semitone}}
\color{markers}
\linethickness{0.5mm}
\put(0,39.5){\line(1,0){60}}
\color{black}
\put(60,38.9){\tiny{\textemdash Dowland}}
\color{black}
\linethickness{1mm}
\put(0,33.1){\line(1,0){60}}
\color{black}
\linethickness{1mm}
\put(0,5){\line(1,0){60}}
\color{black}
\put(60,4){\small{\textemdash 5th fret}}
\end{picture}
\caption{Scale drawing of Ganassi's fourth fret}
\label{fig:gnassi-4}
\end{figure}

As the drawing indicates, his fourth fret is almost unusable since it is yielding a
major third that is even sharper than equal temperament. Ganassi provides no further
explanation for this, but we must assume that most players would change this to
something more suitable and dismiss his instructions for this particular fret.

The initial placement for the remaining sixth and eighth frets is similar to the
fourth, and like his contemporaries, Ganassi does not provide any information
about frets beyond that. His sixth is the same as Gerle's, the arithmetical
mean of the distance between the fifth and seventh, or a ratio of 24:17. His
eighth is the most specific we have seen in any source thus far, a 8:5 ratio.
\begin{blocks}
Then the sixth fret is set at the midpoint of the space between the fifth and seventh
frets but somewhat less, that is so that the thickness of the fret is within the
compass of the distance; that will set its position. The eighth fret will be located
so as to have the same spacing as from the fifth to the sixth frets.
\end{blocks}
Ganassi as well as other sources at this time such as Dowland, mention the thickness
of the gut string used as the fret and allude to the fact that this would have
some impact on its placement. For example, a particularly thick fret might offset
calculations by as much as a millimeter. For modern lute players, this would be an
issue for the lower frets, where gut strings of a millimeter or more are often used, but
the thickness decreases to less than a millimeter for higher frets. No
information exists about the specific thicknesses of the gut strings used for frets on
instruments during this time. Ganassi's mention of it here at least indicates that it
may have been an issue. His instruction is simply to ensure that no matter how thick
the fret is, its entire thickness is \textit{within} the span of the compass.

The final adjustments to Ganassi's fretting scheme include the sixth and eighth frets,
which are moved slightly flatter than where they were originally placed during chapter
four. The unison between the sixth fret of the third string and the first fret of the
second is used to adjust the sixth fret as needed in order to make that sound in tune. The
eighth fret is then adjusted so that the octave formed between the open fourth string
and the eighth fret of the third string is in tune. Using Bodig's measurements, we can
see that the final placement of the sixth fret moves it just slightly past the mark for
the chromatic sixth-comma semitone, or C$\sharp$ on an instrument tuned to a relative
pitch of G.
\begin{figure}[ht]
\centering
\setlength{\unitlength}{1mm}
\begin{picture}(60,43.1)
% Draw fingerboard edges
\color{black}
\linethickness{0.075mm}
\put(0,0){\line(0,1){43.1}}
\put(60,0){\line(0,1){43.1}}

% Draw strings
% 6th course
\color{strings}
\linethickness{0.5mm}
\put(5,0){\line(0,1){43.1}}
\linethickness{0.25mm}
\put(7,0){\line(0,1){43.1}}
% 5th course
\put(15,0){\line(0,1){43.1}}
\put(17,0){\line(0,1){43.1}}
% 4th course
\put(25,0){\line(0,1){43.1}}
\put(27,0){\line(0,1){43.1}}
% 3rd course
\put(35,0){\line(0,1){43.1}}
\put(37,0){\line(0,1){43.1}}
% 2nd course
\put(45,0){\line(0,1){43.1}}
\put(47,0){\line(0,1){43.1}}
% 1st course
\put(56,0){\line(0,1){43.1}}
\color{markers}
\linethickness{0.5mm}
\put(0,8.89999999999998){\line(1,0){60}}
\color{black}
\put(60,8.29999999999998){\tiny{\textemdash Gerle/Dowland}}
\color{markers}
\linethickness{0.5mm}
\put(0,8.09999999999999){\line(1,0){60}}
\color{black}
\put(60,7.49999999999999){\tiny{\textemdash equal temperament}}
\color{markers}
\linethickness{0.5mm}
\put(0,5){\line(1,0){60}}
\color{black}
\put(60,4.4){\tiny{\textemdash diatonic sixth-comma semitone}}
\color{markers}
\linethickness{0.5mm}
\put(0,11.2){\line(1,0){60}}
\color{black}
\put(60,10.6){\tiny{\textemdash chromatic sixth-comma semitone}}
\color{black}
\linethickness{1mm}
\put(0,11.4){\line(1,0){60}}
\color{black}
\linethickness{1mm}
\put(0,38.1){\line(1,0){60}}
\color{black}
\put(60,37.1){\small{\textemdash 5th fret}}
\end{picture}
\caption{Scale drawing of Ganassi's sixth fret}
\label{fig:gnassi-6}
\end{figure}

It is hard to say if this was intentional or simply a coincidence of Ganassi's
adjustment procedures. This is perhaps more evident in the placement of the
eighth fret, which is moved into a position that is not in any one particular
temperament, making it unique.
\begin{figure}[ht]
\centering
\setlength{\unitlength}{1mm}
\begin{picture}(60,37.6)
% Draw fingerboard edges
\color{black}
\linethickness{0.075mm}
\put(0,0){\line(0,1){37.6}}
\put(60,0){\line(0,1){37.6}}

% Draw strings
% 6th course
\color{strings}
\linethickness{0.5mm}
\put(5,0){\line(0,1){37.6}}
\linethickness{0.25mm}
\put(7,0){\line(0,1){37.6}}
% 5th course
\put(15,0){\line(0,1){37.6}}
\put(17,0){\line(0,1){37.6}}
% 4th course
\put(25,0){\line(0,1){37.6}}
\put(27,0){\line(0,1){37.6}}
% 3rd course
\put(35,0){\line(0,1){37.6}}
\put(37,0){\line(0,1){37.6}}
% 2nd course
\put(45,0){\line(0,1){37.6}}
\put(47,0){\line(0,1){37.6}}
% 1st course
\put(56,0){\line(0,1){37.6}}
\color{markers}
\linethickness{0.5mm}
\put(0,10.6){\line(1,0){60}}
\color{black}
\put(60,9.99999999999997){\tiny{\textemdash  chromatic sixth-comma semitone}}
\color{markers}
\linethickness{0.5mm}
\put(0,5){\line(1,0){60}}
\color{black}
\put(60,4.4){\tiny{\textemdash  diatonic sixth-comma semitone}}
\color{markers}
\linethickness{0.5mm}
\put(0,6.89999999999998){\line(1,0){60}}
\color{black}
\put(60,6.29999999999998){\tiny{\textemdash  equal semitone}}
\color{black}
\linethickness{1mm}
\put(0,32.6){\line(1,0){60}}
\color{black}
\put(60,31.6){\small{\textemdash 7th fret}}
\color{black}
\linethickness{1mm}
\put(0,8.89999999999998){\line(1,0){60}}
\end{picture}
\caption{Scale drawing of Ganassi's eighth fret}
\label{fig:gnassi-8}
\end{figure}


Even though Ganassi spends an entire chapter of this treatise on placing the frets according to geometrical
calculations, it is not until the sixth chapter that we see the complete picture of his scheme. Dividing the process as
he does seems to acknowledge the dichotomy between a theoretical system of string division and a practical one that
works in the realm of performance. Beyond this Platonic differentiation in tuning, we really cannot know what his
motivations are.

Ganassi holds many similarities with other sources. The second, fifth, and seventh frets are the same as Gerle and
Dowland. His first fret, however, is noticeably flatter than the Gerle/Dowland model, making it closer to a sixth-comma
meantone chromatic semitone instead of a diatonic one which Gerle and Dowland favor. Perhaps the most important feature
that Ganassi's fretting scheme shares with others from this period is that it combines different types of semitones from
different temperaments. This further indicates that lute temperaments were treated differently than temperaments on
other instruments.

\section{Spanish Vihuela Sources}

We can examine sixteenth-century sources on the vihuela for additional information about fretting.
The vihuela shared many similar characteristics with the lute and it was evident that players of the
vihuela grappled with the same fretting issues that lutenists did. Despite the two instruments'
differences, Spanish vihuelists knew of the lute's repertory and technique. Miguel Fuenllana
discusses the right-handed ``foreign style'' of playing in his \textit{Orphenica lyra} of
1554. This refers to the thumb and index alternation of playing practiced by lutenists at the time.

Information about setting frets on the vihuela come from two sources. The first, and
earliest of the two is Luis Milan's \textit{El Maestro}. Milan was a well-known vihuela
composer and poet who seems to have spent most of his active life around Valencia.
Although his birth and death dates are not known, it is believed he died sometime
after 1561, due to a reference in one of the composer's books of poetry, \textit{El
Cortesano}.\autocite[6]{LG:1} Milan is best described as both a poet and bard who
would sing or recite his poetry to his own music. He was active mostly at the court of
Valencia where he made his living working for the dukes of Calabria. \textit{El
Maestro} is his only musical publication but remains one of the most important vihuela
publications we have, not only for its musical content but for its information about
performance practice such as tempo and interpretive elements that are absent from
other contemporary treatises.

While \textit{El Maestro} contains only a few references to fret placement, they are
very significant. Milan's own remarks were not as exact as the others we have
seen. He gave no measurements with which to mark frets on a straightedge, but he
did give some clues as to which frets he wanted adjusted to achieve a different
quality for certain notes. In Gasser's monograph on performance practices in Milan's
music, he translates the following section from \textit{El Maestro} where Milan
discusses changing the position of the fourth fret when performing Fantasia 14.
\begin{blocks}
Whenever you play the fourth and third tones in those places through which the fantasia
moves, raise the fourth fret of the vihuela a little, so that the note of the fret
becomes strong and not weak.
[ Siempre que ta\~{n}erais el cuatro y tercero tono por estos t\`{e}rminos que est
fantas\`{i}a anda, alzar\`{e}is un poco el cuarto traste de la vihuela para que el
punto del dicho traste sea fuerte y no flaco ]\autocite[156]{LG:1}
\end{blocks}
Additionally, there is another piece of information concerning the same fret before the
beginning of the music for \textit{Con pavor record\'{o} el moro}.
\begin{blocks}
Playing in these pieces on the vihuela, you have to raise the fourth fret a bit toward
the pegs.
[ Ta\~{n}do por estas partes en la vihuela hab\'{e}is de alzar un poco el
cuarto traste hacia las clavijas de la vihuela ]\autocite[156]{LG:1}
\end{blocks}
Both references concern the fourth fret, and both offer an adjustment that raises the
fret closer to the pegs, making the interval smaller. This would equate to a narrower
major third; however, it also assumes that the fret as been placed in some manner prior
to being moved. It is not possible to know by what means Milan would have first placed
his fourth fret, but it seems that for certain pieces, he preferred it
to be slightly flatter than its initial placement. This could corroborate Dowland's
use of a quarter-comma chromatic semitone for his fourth fret, which is substantially
flatter than either an equally-tempered or Pythagorean semitone. However, the problem
here is that Dowland achieves a flatter fourth fret by making his third fret flat
as well. Milan mentions no other fret to adjust, so we might assume that all his frets
were placed according to standard Pythagorean proportions. Still, both Dowland and
Milan do show a preference for flatter fourth frets.

Perhaps the most complete discussion on vihuela fretting appears in Juan Bermudo's
\textit{De ta\~{n}er vihuela}, published in 1555. Bermudo's publication is one of the
most extensive on vihuela performance practice, including several chapters on
intabulation for the vihuela, and fret placement. Bermudo provides a total of three
different fretting schemes. The first of these, which appears in chapter 77, is
Pythagorean. The second scheme attempts to correct the semitones in the first scheme
by adjusting the first, sixth, and eighth frets while leaving the others in their
original Pythagorean form. The third and last of Bermudo's schemes is unrelated to
either of the previous two and is the closest to today's equal temperament.

Dawn Espinosa translated Bermudo's work, with the original Spanish printed alongside
the English translation, and prefaced her translation with an informative discussion.
In it, she provides a table of Bermudo's fretting schemes and the ratios she has
calculated for each fret. For my analysis, I have used these to compare each of
Bermudo's fretting schemes with the others that we have seen so far.

Similar to Ganassi, Bermudo gives us a simple Pythagorean-based fretting system but
then offers two alternative systems as well. This attests to the changes in
tuning that were occurring during the time. Prior to the sixteenth century, Pythagorean tuning
had been the dominant system, but it had its inadequacies. By starting with a
Pythagorean system and then providing alternatives, writers could now acknowledge its
importance and at the same time move past its insufficiencies. For example, Bermudo's second
fretting system presents a solution that corrects some of the problems with
a Pythagorean system and arrives at the exact same placement for the third fret as
Ganassi does.
\begin{figure}[ht]
\centering
\setlength{\unitlength}{1mm}
\begin{picture}(60,45.3)
% Draw fingerboard edges
\color{black}
\linethickness{0.075mm}
\put(0,0){\line(0,1){45.3}}
\put(60,0){\line(0,1){45.3}}

% Draw strings
% 6th course
\color{strings}
\linethickness{0.5mm}
\put(5,0){\line(0,1){45.3}}
\linethickness{0.25mm}
\put(7,0){\line(0,1){45.3}}
% 5th course
\put(15,0){\line(0,1){45.3}}
\put(17,0){\line(0,1){45.3}}
% 4th course
\put(25,0){\line(0,1){45.3}}
\put(27,0){\line(0,1){45.3}}
% 3rd course
\put(35,0){\line(0,1){45.3}}
\put(37,0){\line(0,1){45.3}}
% 2nd course
\put(45,0){\line(0,1){45.3}}
\put(47,0){\line(0,1){45.3}}
% 1st course
\put(56,0){\line(0,1){45.3}}
\color{markers}
\linethickness{0.5mm}
\put(0,6.7){\line(1,0){60}}
\color{black}
\put(60,6.1){\tiny{\textemdash equal}}
\color{markers}
\linethickness{0.5mm}
\put(0,5){\line(1,0){60}}
\color{black}
\put(60,4.4){\tiny{\textemdash Gerle/diatonic sixth-comma}}
\color{markers}
\linethickness{0.5mm}
\put(0,12){\line(1,0){60}}
\color{black}
\put(60,11.4){\tiny{\textemdash Dowland/chromatic sixth-comma}}
\color{black}
\linethickness{1mm}
\put(0,40.3){\line(1,0){60}}
\color{black}
\put(60,39.3){\small{\textemdash 2nd fret}}
\color{black}
\linethickness{1mm}
\put(0,8.70000000000002){\line(1,0){60}}
\end{picture}
\caption{Bermudo's third fret}
\label{fig:bermudo-3}
\end{figure}

Also similar to many of his contemporaries, his solution for the first fret clearly
advocates the diatonic semitone.
\begin{figure}[ht]
\centering
\setlength{\unitlength}{1mm}
\begin{picture}(60,53.4)
% Draw fingerboard edges
\color{black}
\linethickness{0.075mm}
\put(0,0){\line(0,1){53.4}}
\put(60,0){\line(0,1){53.4}}

% Draw strings
% 6th course
\color{strings}
\linethickness{0.5mm}
\put(5,0){\line(0,1){53.4}}
\linethickness{0.25mm}
\put(7,0){\line(0,1){53.4}}
% 5th course
\put(15,0){\line(0,1){53.4}}
\put(17,0){\line(0,1){53.4}}
% 4th course
\put(25,0){\line(0,1){53.4}}
\put(27,0){\line(0,1){53.4}}
% 3rd course
\put(35,0){\line(0,1){53.4}}
\put(37,0){\line(0,1){53.4}}
% 2nd course
\put(45,0){\line(0,1){53.4}}
\put(47,0){\line(0,1){53.4}}
% 1st course
\put(56,0){\line(0,1){53.4}}
\color{markers}
\linethickness{0.5mm}
\put(0,37.4){\line(1,0){60}}
\color{black}
\put(60,36.8){\tiny{\textemdash diatonic quarter-comma semitone}}
\color{markers}
\linethickness{0.5mm}
\put(0,48.4){\line(1,0){60}}
\color{black}
\put(60,47.8){\tiny{\textemdash chromatic sixth-comma semitone}}
\color{markers}
\linethickness{0.5mm}
\put(0,40.4){\line(1,0){60}}
\color{black}
\put(60,39.8){\tiny{\textemdash Gerle/Dowland/diatonic sixth-comma semitone}}
\color{markers}
\linethickness{0.5mm}
\put(0,43.5){\line(1,0){60}}
\color{black}
\put(60,42.9){\tiny{\textemdash equal semitone}}
\color{black}
\linethickness{1mm}
\put(0,38.3){\line(1,0){60}}
\color{black}
\linethickness{1mm}
\put(0,5){\line(1,0){60}}
\color{black}
\put(60,4){\small{\textemdash 2nd fret}}
\end{picture}
\caption{Comparison of Bermudo's first fret}
\label{fig:bermudo-1}
\end{figure}

However, it is slightly sharper than the sixth-comma diatonic semitone of Gerle and
Dowland.

Bermudo refers to the ``faults'' found in fretting systems between notes that are
\textit{mi} or \textit{fa}. In terms of sixteenth-century theory, this was the
modern equivalent of a sharp note versus a flat note. After setting the frets in
his first scheme, the Pythagorean one, he discusses the problems found on the eighth
fret:
\begin{blocks}
On the eighth fret there seem to be three faults. This fret should be \text{fa} for the
seventh, fourth and first strings, but is it [made] \text{mi} for all the strings.
\autocite[95]{DE:1}
\end{blocks}
Bermudo is highlighting the central problem about how to set the frets: the distance of one
semitone on one string is the same for all the other semitones on that string. Depending
on the temperament and semitone, a minor semitone might be required for one string while
a major semitone would be needed for another.

One of Bermudo's solutions, which we also see with Milan, is to adjust the frets
according to the key of the piece. Milan, for example, would adjust the fourth fret
when playing the third and fourth tones, while Bermudo describes players who move their
frets according to their ears:
\begin{blocks}
We have seen players who, with their frets set for the sixth, want to play the fourth
mode, but are unable to do it without moving the frets as their good ear tells them.
What I intend to do here is to give the measurements by which to place the frets, so
that those who are not [such good] musicians will be able to place them with ease and
exactitude, and thus the vihuela with be more perfect.
\autocite[78]{DE:1}
\end{blocks}
Bermudo seems to suggest that good players are capable of moving their frets
for different modes, while inexperienced or lesser skilled players should
rely on Bermudo's calculations to place the frets for them.

For the latter, Bermudo devised a scheme that approximates modern equal temperament,
essentially splitting the wholetone into two equal halves, although the
composer himself acknowledges that it is not his aim to do so:
\begin{blocks}
All the [theorists] agree that the wholetone cannot be divided into two equal
semitones, but that is what we are presuming to do. The above being the most agreed
and true thing among theorists, on the vihuela we find the opposite in practice.
\autocite[xx]{DE:1}
\end{blocks}
For Bermudo, the issue was a question of theory versus practice. Espinosa summarizes
this issue succinctly: ``for him, theory has more authority than practice, but he
concedes that practice precedes theory.'' \autocite[xx]{DE:1}

In table~\ref{bermudo:differences}, the difference between each of Bermudo's frets and
modern equal temperament is expressed in both cents and in three different mensur
lengths: 600mm, 700mm, and 800mm. In this instance, we use millimeters for the mensur length because they
can express the differences with more accuracy. 700mm is the same as 70cm, which
has been the reference mensur length in our other diagrams. Cents, a logarithmic
measurement, are used as another point of comparison. Cents are often used with
pitches in equal temperament, where the octave is divided into 1200 cents giving
each semitone an equal amount of 100 cents each.
Subjectively speaking, the human ear can notice differences of a few cents, but near or
less than one cent would be difficult.
\begin{table}[h!]
  \begin{center}
  \begin{tabular}{ r r| r r r }
   \textit{Fret} & \textit{In cents} & \multicolumn{3}{c}{\textit{At different mensur lengths in mm}} \\
    & & \textit{600mm} & \textit{700mm} & \textit{800mm} \\
   \hline
       1 &     0.50	&     0.16	 &     0.19 &     0.22 \\
       2 &     0.31	&     0.09	 &     0.11 &     0.13 \\
   \textbf{3} & \textbf{-5.87}	& \textbf{-1.71} & \textbf{-2.00} & \textbf{-2.28} \\
       4 &     0.62	&     0.17	 &     0.20 &     0.23 \\
       5 &     -1.96	&     -0.51	 &     -0.59 &     -0.68 \\
       6 &     -1.46	&     -0.36	 &     -0.42 &     -0.48 \\
       7 &     -1.65	&     -0.38	 &     -0.44 &     -0.51 \\
   \textbf{8} & \textbf{-7.82}	& \textbf{-1.72} & \textbf{-2.00} & \textbf{-2.28} \\
       9 &     -1.33	&     -0.27	 &     -0.32 &     -0.37 \\
       10 &     -3.91	&     -.76	 &     -0.89 &     -1.02 \\
  \end{tabular}
  \end{center}
  \caption{Differences bewteen Bermudo's third scheme and true equal temperament}
  \label{bermudo:differences}
\end{table}
The negative numbers in the table show a fret that is flatter than equal temperament,
while a positive number indicates a fret sharper than equal temperament.

As the table shows, all but two of these frets are within a few millimeters of an equally-tempered semitone, in a
variety of different mensur lengths. The only exceptions are at the third and eighth frets which are flatter than equal
temperament. When looking at the differences in cents, these frets are quite flat, and most listeners would notice a
difference of 5 to 8 cents; however, the differences of fractions of one cent would be virtually undetectable. For what
reason Bermudo chose to make these particular frets flatter than the others, by comparison, we cannot know. However, an
interesting aspect to Bermudo's scheme is that, after the fourth fret, each one is gradually flatter. This may be to
counteract a problem common to lutes and fretted instruments where pitches at the higher frets tended to sound sharper.
The effect was greater at the highest frets, such as the tenth, which is perhaps why this fret is almost four cents flat
in Bermudo's scheme.

\section{Sources of Equal Fretting}

The last category of fretting systems that we will examine come from the late sixteenth and early seventeenth centuries
and advocated dividing the octave into equal semitones. This is not to say that these systems produced the kind of equal
temperament that we have today. Creating a temperament with semitones that are exactly equal to one another requires a
special class of mathematical functions called logarithms, which were not used in tuning until later in the seventeenth
century. Despite their ability to divide the octave correctly into equal semitones, equal temperament did not become a
true standard until the twentieth century. As Ross Duffin has argued, even during the nineteenth century musicians
still tempered their fifths more than those in equal temperament.

Prior to these advances in mathematics, equal semitones were approximated using other methods. Sources during this time
describe several different systems of measurement that produced temperaments very close to modern equal temperament, but
which were achieved through a means of calculation that did not involve logarithms. The first of these we have already
seen with Bermudo's third method of fretting, which uses standard geometrical string divisions common to most lute
sources at the time. Additional information on fretting in equal semitones comes from Vicenzo Galilei and Marin
Mersenne, as well as numerous other anecdotes collected by Mark Lindley in which contemporary musicians refer to the
lute's ability to tune equally.

One of the simplest ways in which a lute player could achieve equal semitones was to
divide the octave into a series of Pythagorean minor semitones, at a ratio of 18:17.
Mersenne, Galilei, and other sources refer to it as the \textit{18:17 Rule}, and it
could create equal semitones on any instrument, but the technique seems to be solely
reserved for fretted instruments.

The idea originates with the problem of dividing the Pythagorean 9:8 wholetone into 2 geometrically equal parts,
which I discussed in the previous chapter. A series of minor semitones produced fairly acceptable
results, but as the frets progressed further down the fingerboard, the distances became increasingly
smaller, making each fret flatter than the next.

Looking at table~\ref{18:17rule}, we can see the 18:17 rule compared with equal temperament
when executed with a variety of different mensur lengths as well as how it
compares in terms of cents.
\begin{table}[h!]
  \begin{center}
  \begin{tabular}{ r r| r r r }
   \textit{Fret} & \textit{In cents} & \multicolumn{3}{c}{\textit{At different mensur lengths in mm}} \\
    & & \textit{600mm} & \textit{700mm} & \textit{800mm} \\
   \hline
   1 & -1.05 & -0.34 & -0.40 & -0.46 \\
   2 & -2.09 & -0.65 & -0.75 & -0.86 \\
   3 & -3.14 & -0.91 & -1.07 & -1.22 \\
   4 & -4.18 & -1.15 & -1.34 & -1.54 \\
   5 & -5.23 & -1.36 & -1.59 & -1.81 \\
   6 & -6.27 & -1.54 & -1.80 & -2.05 \\
   7 & -7.32 & -1.70 & -1.98 & -2.26 \\
   8 & -8.36 & -1.83 & -2.14 & -2.44 \\
   9 & -9.41 & -1.94 & -2.27 & -2.59 \\
   10 & -10.45 & -2.04 & -2.38 & -2.72 \\
   11 & -11.50 & -2.12 & -2.47 & -2.82 \\
   12 & -12.54 & -2.18 & -2.55 & -2.91 \\
  \end{tabular}
  \end{center}
  \caption{Comparison of fret placement using the 18:17 system}
  \label{18:17rule}
\end{table}
The differences between the two systems for the first five frets are very slight: only
less than two millimeters between the measurements of these frets,
and at most a five cent difference in pitch. At the higher frets, however, the
differences become larger. By the time we reach the octave, the fret is more than
twelve cents flatter than the pure octave found in equal temperament. This would be
very noticeable to most individuals.

Despite the decrease in distances, using a series of equal minor semitones in the 18:17 rule provided a practical
temperament because its problems were mitigated by two factors: First, most musicians at this time were using some form
of temperament that flattened fifths, and at the seventh fret where the fifth occurs, the fret is already flattened by a
few millimeters. Second, the tension created by stopping a string at a given fret, especially if the fret itself is
quite thick, or doubled as is the case with the viola da gamba, can raise the pitch slightly. \autocite[21]{ML:1} This
effect is compounded by the fact that changes in finger pressure or position have a more noticeable effect on pitch at
high positions. Making the frets gradually flatter as they move to higher positions would help counteract this effect.

Another method of determining equal fret positions used a device called a mesolab, which was an instrument of Greek
origin that could determine mean proportionals, or the average distance between two lines. In 1558, Zarlino published a
fretting scheme for lute that relied on such a device. \autocite[26]{ML:1} There is also a picture of one in the
\textit{Discours non plus melancoliques que diverses, de choses mesmement, qui appartiennent a notre FRANCE: \& a la fin
La maniere de bien \& iustement entoucher les Lucs \& Guiternes}, published in 1556, which was mentioned earlier.
However, use of the mesolab did not appear in any practical fretting guides and seems to have been relegated to music
theory sources where equal temperament was a kind of puzzle to be solved and not really taken seriously in a musical
context.

By the seventeenth century, advances in mathematics employed decimal numbers and logarithms to express musical ratios.
Johannes Faulhaber was the first to use logarithms for a fretting scheme in his 1630 publication. \autocite[21]{ML:1}
In order for it to work, the string had to be divided into 20,000 parts. This idea later found its way into an appendix
to a translation of Rene Descartes' \textit{Musicae compendium}. Mark Lindley has concluded that its translator is
William Brouncker, who also wrote the appendix, which contains several fretting schemes for lute, including our modern $
\sqrt[12]{2} $ method, as well as others of his own devising.

Despite the accuracy of logarithmic calculations in placing frets at equal temperament, the method was largely ignored
except in a few treatises. While many authors spoke of the lute's ability to play with equal semitones, the idea of an
``equal temperament'' was somewhat less certain. From the available evidence, systems approximating equal temperament
and utilizing the idea of equal semitones certainly existed during this period, but they still recognized the difficult
issue of how to reconcile unequal semitones found in the temperaments of all the other instruments. The central crux of
the lute player's struggle was how to navigate between these two poles.

\section{Summary}

Based on this lengthy description of the available sources on fretting from the sixteenth century and seventeenth
centuries, one can see that some of them do agree on the placement of certain frets (see table ~\ref{table:comparison}).
The second, fifth, and seventh are all uniformly Pythagorean in nature, making the fretted intervals of the perfect
fourth and fifth pure. When players tuned their open strings to one another by using the fifth fret of the adjacent
strings as a reference, this made the interval between the two strings a perfect fourth as well. This is corroborated by
every source at this time that contained instructions for tuning the lute's open strings.

\begin{table}[h!]
 \begin{center}
  \begin{tabular}{ l l l l l }
   \textit{Fret} & \textit{Dowland} & \textit{Gerle} & \textit{Ganassi} & \textit{Bermudo II} \\
   \hline
   1 & \multicolumn{2}{|c|}{\cellcolor[gray]{0.9}diatonic $ \frac{1}{6} $} & unique/chromatic $ \frac{1}{6} $ & \cellcolor[gray]{0.9}diatonic $ \frac{1}{4} $ \\
   2 & \multicolumn{4}{|c|}{\cellcolor[gray]{0.9}Pythagorean 9:8 wholetone} \\
   3 & chromatic $ \frac{1}{6} $ & diatonic $ \frac{1}{6} $ & \multicolumn{2}{|c|}{\cellcolor[gray]{0.9}unique} \\
   4 & chromatic $ \frac{1}{4} $ & unique/equal & unique & equal \\
   5 & \multicolumn{4}{|c|}{\cellcolor[gray]{0.9}Pythagorean 3:2 fourth} \\
   6 & \multicolumn{2}{|c|}{\cellcolor[gray]{0.9} unique/equal } & unique/chromatic $ \frac{1}{6} $ & unique \\
   7 & \multicolumn{4}{|c|}{\cellcolor[gray]{0.9}Pythagorean 2:3 fifth} \\
   8 & n/a & n/a & \multicolumn{2}{|c|}{\cellcolor[gray]{0.9}unique} \\
  \end{tabular}
 \end{center}
 \caption{Comparison of fretting schemes}
 \label{table:comparison}
\end{table}

The use of Pythagorean intervals puts these fretting systems at odds with a meantone temperament. The second fret seems
like a good choice at first glance, since it would be used in most scalar passages as the wholetone above the open
strings. However, when used in chords, the second fret often holds the third of the chord. In meantone temperaments, it
would need to be lower in pitch or moved closer to the nut in order to accommodate either a sixth or quarter-comma
meantone temperament. Furthermore, the presence of a fretted pure fourth would create pure fourths between open strings
during the tuning process. This discounts the use of any meantone temperament because of the wide third that would
result between the instrument's third and fourth courses.

It is clear that lutenists were experimenting with meantone intervals for some of their frets.  We can see some general
agreement that the first fret should be diatonic in nature. This would mean that for instruments tuned in either G or
A, their first frets would have to be A$\flat$ or B$\flat$ respectively. However, there is disagreement over the
remaining chromatic frets. For example, each source has its own technique for placing the fourth fret, some of which are
very close to equal temperament, but each is essentially unique to their system. Dowland and Gerle seem to favor sixth-
comma meantone for their third frets, but disagree over the quality. Ganassi and Bermudo, on the other hand, arrive at
the same placement for their third fret, but it does not match any interval in the meantone scheme of temperaments.

Given the issues with the extant fretting resources presented in this chapter, we can conclude that most of the them are
problematic at best. It is evident that players embraced a variety of different techniques when addressing the issue of
fretting, and we should use that as a guiding principle when creating our own, but that means we must inevitably make
our own choices. Lastly, the historical solutions were not uniform and none consisted entirely of Pythagorean intervals
or meantone intervals of one type or another. With that in mind, in the next chapter I shall propose more accurate
solutions for fretting that retain some of the principles found in these historical resources. I will then examine my
solutions in a musical context, using examples from the period.
