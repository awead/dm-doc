\begin{titlepage}
  \begin{center}
    \vspace*{1cm}
    
    LUTE TUNING AND TEMPERAMENT IN THE SIXTEENTH AND SEVENTEENTH CENTURIES
    
    \vspace{1.5cm}
    
    BY\\
    ADAM WEAD\\
    
    \vfill
    
    \begin{centerblocks}
      Submitted to the faculty of the\\
      Jacobs School of Music in partial fulfillment\\
      of the requirements for the degree,\\
      Doctor of Music,\\ 
      Indiana University\\ 
      August, 2014\\
    \end{centerblocks}
    
    
  \end{center}
\end{titlepage}

\null
\vfill

\begin{centerblocks}
Accepted by the faculty of the Jacobs School of Music,\\
Indiana University, in partial fulfillment of the requirements\\
for the degree Doctor of Music.\\
\end{centerblocks}

\null
\vfill

\begin{blocks}
\hfill \line(1,0){250}\\
\hfill Nigel North, Research Director \& Chair\\
\vspace*{2\baselineskip}
\hfill \line(1,0){250}\\
\hfill Stanley Ritchie\\
\vspace*{2\baselineskip}
\hfill \line(1,0){250}\\
\hfill Ayana Smith\\
\vspace*{2\baselineskip}
\hfill \line(1,0){250}\\
\hfill Elisabeth Wright\\
\end{blocks}

\chapter*{Acknowledgments}

First and foremost, my thanks to Nigel North for guiding me through the research process and helping me fully understand
this topic. To my committee members, I thank them for their time and attention to the details of my writing and helping
me to clarify what is I am trying to say. I would also like to thank David Dolata, who read an early draft of this paper
and provided me with many insightful comments and suggestions. I am indebted to my father, George Wead, for many things,
but his translations of the Latin fretting sources I encountered were especially helpful.  Lastly, a big word of thanks
to Anie, for seeing me through this to the very end.

\clearpage
